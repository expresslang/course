% rulesv.tex   EXPRESS Course Rules 
% Created by Peter R Wilson, 1992 -- 2004
%\documentclass[11pt]{article}
%\documentclass{article}
%\usepackage{vgraph}
%\usepackage{headfoot}

%%%%\notestrue
%%%%\notesfalse

%\begin{document}


\pagestyle{empty}
\bodsiz

\vspace*{\beftit}
\begin{center}
{\titsiz \bfseries Information Modeling Using EXPRESS --- \\ RULES}
\end{center}
\vspace{\beftit}
\vspace{\beftit}
\begin{center}
%%%\textbf{Peter R. Wilson}
\end{center}

\clearpage
\ifnotes
  \pagestyle{plain}
\else
  \pagestyle{empty}
%%%  \pagefooter{}{}{{\small Peter R. Wilson}}
\fi

\begin{remarks}
\remintro
NB. NOTES ARE ON EVEN NUMBERED (LEFT HAND) PAGES AND THE PRESENTATION ON
ODD NUMBERED (RIGHT HAND) PAGES.

\remtitle{The Principles}

    The basic premise is that global RULEs are bad.

    It's also important that there is no guarantee as to when
any constraints will be checked. This is entirely dependent on
the particular implementation of an EXPRESS-defined 
\emph{object base} (which could be a Part 21 ASCII data file or a
sophisticated shared database).

\remend
\end{remarks}

\vtitle{THE PRINCIPLES}

\begin{enumerate}

\item EXPRESS is permissive --- anything that is not prohibited is allowed.

\item The rule about rules is very simple:

\begin{itemize}
\item Don't use global rules
\end{itemize}

(except as a last resort).

\item EXPRESS does not define the time (e.g. instance creation, 
instance useage) at which rules are tested.

\end{enumerate}

\begin{remarks}
\remintro
\remtitle{Kinds of rules}

\emph{Local} rules (WHERE) clauses in entities apply to each instance.

\emph{Global} rules (RULEs) apply to (combinations of) (some) entity
instances.

    Both are used to specify constraints. Other constraint methods
are available.

\remend
\end{remarks}

\vtitle{Kinds of rules}

There are two kinds of rule:

\begin{description}
\item[Local rules] These are defined in entities or types and apply
to each and every instance.
\item[Global rules] These are defined outside of entities and apply to
specific entity instances, or combinations of entity instances.
\end{description}

Constraints can also be specified by UNIQUE clauses, INVERSE attributes,
subtyping with attribute redeclaration, importation (USE, REFERENCE,
subtype and select pruning).
Sometimes DERIVED attributes can be considered as constraints.


\begin{remarks}
\remintro
\remtitle{The NYM Principle}

    This is a major guiding principle, particularly as much
of the understanding of a model comes from the names used.

\remend
\end{remarks}

\vtitle{The NYM Principle}

\begin{itemize}
\item If things are the same, then they should have the same name.

\item If things are not the same, then they are different.

\item Different things should have different names.
\end{itemize}

In general, `One name, one meaning, one definition'. Synonyms and
homonyms in a model are a fruitful and never-ending source of confusion.


\begin{remarks}
\remintro
\remtitle{RULES and the Nym principle}

    This is an example that goes over several pages.

We start with this model which is about --- what??? The schema
names imply something --- \texttt{main} and \texttt{sub}(sidiary),
but the entity names\ldots

\remend
\end{remarks}

\vtitle{Global RULES and the Nym principle}

    Global rules should not change the meaning or potential for 
interoperability of the entity to which they are applied.

\begin{verbatim}
SCHEMA main;
  ENTITY fred;
    a : LIST OF joe;
  END_ENTITY;
...
END_SCHEMA; -- main

SCHEMA sub;
  USE FROM main (fred);
...
\end{verbatim}



\begin{remarks}
\remintro
\remtitle{(cont)}

And (still within the \texttt{sub} schema) there is a constraint 
against \texttt{fred}.

We can, but should not, use a RULE to express it. This requires
an algorithmic approach which is quite possible to get wrong.

\remend
\end{remarks}

\clearpage

There must be at least 5 \texttt{joe}s in the list
for attribute \texttt{a}.


\begin{verbatim}
RULE bad_rule FOR (fred);
LOCAL
  good : LOGICAL := TRUE;
END_LOCAL;
REPEAT i := 1 TO SIZEOF(fred);
  IF SIZEOF(fred[i].a) < 5
  THEN good := FALSE; SKIP;
  END_IF;
END_REPEAT;
WHERE
 w1: good;
END_RULE;

END_SCHEMA; -- sub
\end{verbatim}



\begin{remarks}
\remintro
\remtitle{(cont)}

Here are two better approaches, both involving subtyping. The second
is the better as it involves less ``checking'' and is clearer
to the reader.

\remend
\end{remarks}

\clearpage

The constrained version of \texttt{fred} is better modeled as:

\begin{verbatim}
SCHEMA sub;
  REFERENCE FROM main (fred);

ENTITY sub_fred
  SUBTYPE OF (fred);
WHERE
 w1: SIZEOF(SELF\fred.a) >= 5;
END_ENTITY;
\end{verbatim}

Or, perhaps even better as:
\begin{verbatim}
ENTITY sub_fred
  SUBTYPE OF (fred);
  SELF\fred.a : LIST [5:?] OF joe;
END_ENTITY;
\end{verbatim}


\begin{remarks}
\remintro
\remtitle{Local (WHERE) Rules}

These are one or more LOGICAL expressions.

There are problems with an instance if any of these is FALSE but
no problems if all are TRUE. A mixture of TRUE and UNKNOWN leaves
the instance in limbo regarding problems.

\remend
\end{remarks}

\vtitle{Local (WHERE) Rules}

    Are defined within the definition of a type or entity and apply to 
each and every instance of the type or entity.

\begin{verbatim}
WHERE
  label_1: logical_expression_1 ;
  label_2: logical_expression_2 ;
  ....
END_
\end{verbatim}


\begin{itemize}
\item An instance is non-conforming if any logical expression evaluates 
      to FALSE.
\item An instance is conforming if all the logical expressions evaluate 
      to TRUE.
\item An instance is considered to be not non-conforming if some or all the 
      logical expressions evaluate to UNKNOWN and the remainder evaluate 
      to TRUE.
\end{itemize}


\begin{remarks}
\remintro
\remtitle{`Logical' Rule}

If the \texttt{z} attribute has no value (represented as `?') the
expression evaluates to UNKNOWN.

\remend
\end{remarks}

\vtitle{`Logical' Rule}

    This domain rule may evaluate to FALSE, UNKNOWN or TRUE.

\begin{verbatim}
ENTITY vector;
  x, y : REAL;
  z    : OPTIONAL REAL;
WHERE
  w1: x**2 + y**2 + z**2 = 1.0;
END_ENTITY;
\end{verbatim}


\begin{remarks}
\remintro
\remtitle{`Boolean' Rule}

The NVL function returns its first argument if it is valued otherwise
(i.e., when it is ?) it returns its second argument. Now the
expression will be either TRUE or FALSE.

\remend
\end{remarks}

\vtitle{`Boolean' Rule}

    This domain rule will only evaluate to FALSE or TRUE.

\begin{verbatim}
ENTITY vector;
  x, y : REAL;
  z    : OPTIONAL REAL;
WHERE
  w1: x**2 + y**2 + NVL(z, 0.0)**2 = 1.0;
END_ENTITY;
\end{verbatim}

NOTE: If \texttt{x} or \texttt{y} does not have a value in a particular
instance of \texttt{vector}, then the instance is non-conforming 
by definition.


\begin{remarks}
\remintro
\remtitle{`Function' Rule}

    For non-trivial WHERE rules you can use a FUNCTION that returns
a LOGICAL or BOOLEAN result. (I think it also makes for a cleaner, i.e.,
less cluttered, and therefore more understandable model).

\remend
\end{remarks}

\vtitle{`Function' Rule}

    A rule can be described using a logical (or boolean) function.

\begin{verbatim}
ENTITY vector;
  x, y : REAL;
  z    : OPTIONAL REAL;
WHERE
  w1: unit_vector(SELF);
END_ENTITY;

FUNCTION unit_vector(v:vector):BOOLEAN;
  RETURN(v.x**2 + v.y**2 + 
         NVL(v.z, 0.0)**2 = 1.0);
END_FUNCTION;
\end{verbatim}
%%\clearpage

\begin{remarks}
\remintro
\remtitle{(cont)}

    For non-trivial WHERE rules you can use a FUNCTION that returns
a LOGICAL or BOOLEAN result. (I think it also makes for a cleaner, i.e.,
less cluttered, and therefore more understandable model).

\remend
\end{remarks}

\clearpage

    Functions are of most use when it is difficult to express the constraint
as a single logical expression. They are also beneficial when the same
constraint applies to different kinds of things.

\begin{verbatim}
ENTITY vector;
  x, y : REAL;
  z    : OPTIONAL REAL;
WHERE
  w1: unit_vector(x,y,z);
END_ENTITY;

FUNCTION unit_vector(u,v,w:REAL):LOGICAL;
  IF (NOT EXISTS(w)) THEN
    IF (NOT EXISTS(v)) THEN
      RETURN(u**2 = 1.0);
    END_IF;
    RETURN(u**2 + v**2 = 1.0);
  END_IF;
  RETURN(u**2 + v**2 + w**2 = 1.0);
END_FUNCTION;
\end{verbatim}


\begin{remarks}
\remintro
\remtitle{UNIQUE}

    The next few examples illustrate how UNIQUE constraints
may be used.

   A \texttt{circle}, defined via the location of its center 
and its radius, is used throughout.

\remend
\end{remarks}

\vtitle{UNIQUE}

\begin{verbatim}
ENTITY circle;
  centre : point;
  radius : positive_number;
END_ENTITY;
\end{verbatim}

There can be any number of circles in the object base with identical centres
and/or radii.

\begin{remarks}
\remintro
\remtitle{(cont)}

    The \texttt{center} of each \texttt{circle} must be unique.

\remend
\end{remarks}

\clearpage

\begin{verbatim}
ENTITY circle;
  centre : point;
  radius : positive_number;
UNIQUE
  not_concentric : centre;
END_ENTITY;
\end{verbatim}

There can be any number of circles in the object base with identical 
radii but none with identical centres.
(No circles are concentric but some may have the same size.)

\begin{remarks}
\remintro
\remtitle{(cont)}

    Each \texttt{radius} must be unique.

\remend
\end{remarks}

\clearpage

\begin{verbatim}
ENTITY circle;
  centre : point;
  radius : positive_number;
UNIQUE
  different_sizes : radius;
END_ENTITY;
\end{verbatim}

There can be any number of circles in the object base with identical 
centres but none with identical radii.
(No circles have the same size but some may be concentric.)

\begin{remarks}
\remintro
\remtitle{(cont)}

    Each \texttt{center} must be unique.

    Separately, each \texttt{radius} must be unique.

    This is probably not a realistic real-life requirement.

\remend
\end{remarks}

\clearpage

\begin{verbatim}
ENTITY circle;
  centre : point;
  radius : positive_number;
UNIQUE
  not_concentric : centre
  different_sizes : radius;
END_ENTITY;
\end{verbatim}

There can be no circles in the object base with identical 
centres and no circles with identical radii.
(Every circle is a different size and differently located.)

\begin{remarks}
\remintro
\remtitle{(cont)}

    The combination of \texttt{center} and \texttt{radius} must be unique.

    This is probably the effect that was sought after by the previous example.

\remend
\end{remarks}

\clearpage

\begin{verbatim}
ENTITY circle;
  centre : point;
  radius : positive_number;
UNIQUE
  all_different : centre, radius;
END_ENTITY;
\end{verbatim}

There can be no circles in the object base with the identical 
combination of centre and radius.
(No circles represent the same `point set'.)


\begin{remarks}
\remintro
\remtitle{Instance and Value}

    Every entity instance has a unique `object identifier' or `oid'.
Two instances may have the same attribute values but are distinguished
by their oids. (EXPRESS leaves it up to an object base implementation
to decide what an oid is).

    Everything else is, in some sense, anonymous.

    For comparisons \verb?:<>:? and \verb?:=:? are instance (un)equal,
while \verb?<>? and \verb?=? are value (un)equal.

\texttt{VALUE\_UNIQUE} is a built-in EXPRESS function.

\remend
\end{remarks}

\vtitle{Instance and Value}

    In EXPRESS comparisons for uniqueness 
are performed on the `object-id' for entity instances, and on values 
for `anonymous' types (e.g. REAL). Thus,
\begin{verbatim}
TYPE pair = SET [2:2] OF point;
END_TYPE;
\end{verbatim}
requires that \verb?pair[1] :<>: pair[2]? is TRUE, but \verb?pair[1] = pair[2]?
may be TRUE or FALSE.

    For value uniqueness, do something like:
\begin{verbatim}
TYPE vpair = SET [2:2] OF point;
WHERE
 vun: VALUE_UNIQUE(SELF);
END_TYPE;
\end{verbatim}
which requires \verb|vpair[1] = vpair[2]| to be FALSE.


\begin{remarks}
\remintro
\remtitle{Joint value uniqueness}

UNIQUE applied to entity instances is oid-based.

\remend
\end{remarks}

\vtitle{Joint value uniqueness}

\begin{verbatim}
ENTITY e;
 a1 : a;
 a2 : b;
 a3 : c
UNIQUE
  ju : a1, a2;
END_ENTITY;
\end{verbatim}
The values of the attributes \texttt{a1} and \texttt{a2} are constrained 
to be jointly \emph{instance unique}. 

    If they are further required to be jointly \emph{value unique,}
use a global rule of the following kind to specify this additional
constraint.

\begin{remarks}
\remintro
\remtitle{(cont)}

\texttt{temp} is an ENTITY (local to the RULE) whose only attributes are
those involved in the value uniqueness constraint.

The REPEAT loop creates an instance of \texttt{temp} for each instance
of \texttt{e} and collects them into the SET \texttt{s}. Now, if each
member of \texttt{s} is value unique, then the \texttt{e} instances
are also value unique on the attribute pair.

\remend
\end{remarks}

\clearpage

\begin{verbatim}
RULE vu FOR (e);
  ENTITY temp;
    a1 : a;
    a2 : b;
  END_ENTITY;
LOCAL
  s : SET OF temp := [];
END_LOCAL;
REPEAT i := 1 TO SIZEOF(e);
  s := s + temp(e[i].a1, e[i].a2);
END_REPEAT;
WHERE
  jvu: VALUE_UNIQUE(s);
END_RULE;
\end{verbatim}

    Note the use of an ENTITY definition local to the rule, and the
use of the entity constructor for instances of this entity type.


\begin{remarks}
\remintro
\remtitle{GLOBAL RULES}

    RULEs apply to (combinations) of entity instances.

\remend
\end{remarks}

\vtitle{GLOBAL RULES}

    Are defined outside entities and only apply to entities.
Every instance of the specified entity(s) is examined. The
entity instances are conforming the WHERE rules all evaluate to TRUE.

\begin{verbatim}
RULE rname FOR (ent1, ent2, ...);
  body of rule (code)
WHERE
  label_1: logical_expression_1 ;
   ...
END_RULE;
\end{verbatim}

    All instances of entities of the given type(s) are examined during
rule execution (combinatorial explosion?).


\begin{remarks}
\remintro
\remtitle{Global Rule Usage}

Do your best to avoid using RULEs, but sometimes this is not possible.

\remend
\end{remarks}

\vtitle{Global Rule Usage}

    Use a global rule when:
\begin{enumerate}
\item A combination of different entity types must be constrained; or
\item A constraint only applies to some, but not all, instances of
      a particular entity type; or
\item The number of instances is to be constrained.
\end{enumerate}


\begin{remarks}
\remintro
\remtitle{Person Example}

    There now follows a sequence of models of a person.

    This is the initial model. What odd things does it allow?
How can it be brought closer to reality?

\remend
\end{remarks}

\vtitle{Person Example}

\begin{verbatim}
ENTITY person;
  name   : STRING;
  ss_no  : INTEGER;
  sex    : gender;
  spouse : OPTIONAL person;
UNIQUE
  un1: ss_no;
END_ENTITY;
\end{verbatim}


\begin{remarks}
\remintro
\remtitle{Person Example}

The intent of the WHERE rule is not particularly obvious.
Is it correct?

\remend
\end{remarks}

\vtitle{Person Example}

\begin{verbatim}
ENTITY person;
  name   : STRING;
  ss_no  : INTEGER;
  gender : sex;
  spouse : OPTIONAL person;
UNIQUE
  un1: ss_no;
WHERE
  w1: (EXISTS(spouse) AND 
       gender <> spouse.gender)
      XOR (NOT EXISTS(spouse));
END_ENTITY;
\end{verbatim}


\begin{remarks}
\remintro
\remtitle{Person Example}

    This eliminates the WHERE rule, making the model easier to
understand. Are there any problems with this?

\remend
\end{remarks}

\vtitle{Person Example}

\begin{verbatim}
ENTITY person;
  name  : STRING;
  ss_no : INTEGER;
UNIQUE
  un1: ss_no;
END_ENTITY;

ENTITY male
  SUBTYPE OF (person);
  wife : OPTIONAL female;
END_ENTITY;

ENTITY female
  SUBTYPE OF (person);
  husband : OPTIONAL male;
END_ENTITY;
\end{verbatim}


\begin{remarks}
\remintro
\remtitle{Person Example}

    This model eliminates hermaphrodites. 
Is all well now?

\remend
\end{remarks}

\vtitle{Person Example}

\begin{verbatim}
ENTITY person
  SUPERTYPE OF (ONEOF(male,female));
  name  : STRING;
  ss_no : INTEGER;
UNIQUE
  un1: ss_no;
END_ENTITY;

ENTITY male
  SUBTYPE OF (person);
  wife : OPTIONAL female;
END_ENTITY;

ENTITY female
  SUBTYPE OF (person);
  husband : OPTIONAL male;
END_ENTITY;
\end{verbatim}


\begin{remarks}
\remintro
\remtitle{Example --- Married rule}

    The RULE (if it is coded properly) checks that
husbands and wives are married to each other.

\remend
\end{remarks}

\vtitle{Example --- Married rule}

\begin{verbatim}
RULE married FOR (male, female);
  LOCAL
    ok1, ok2 : BOOLEAN := TRUE;
  END_LOCAL;
  IF (EXISTS(male.wife) AND 
      male :<>: male.wife.husband) THEN
    ok1 := FALSE;
  END_IF;
  IF (EXISTS(female.husband) AND
      female :<>: female.husband.wife) THEN
    ok2 := FALSE;
  END_IF;
WHERE
  r1: ok1;
  r2: ok2;
END_RULE;
\end{verbatim}


\begin{remarks}
\remintro
\remtitle{Example --- Married entity}

A simple model, and also one of broader applicability ---
in many cases someone's marital status is irrelevent.
We could also SUBTYPE \texttt{married} if it was necessary to
record further information about that (e.g., when it started).

\remend
\end{remarks}

\vtitle{Example -- Married entity}

\begin{verbatim}
ENTITY male SUBTYPE OF (person);
END_ENTITY;

ENTITY female SUBTYPE OF (person);
END_ENTITY;

ENTITY married;
  husband : male;
  wife    : female;
UNIQUE
  no_bigamy: husband;
  no_polyandry: wife;
END_ENTITY;
\end{verbatim}


\begin{remarks}
\remintro
\remtitle{Limit instances}

    A RULE has to be used if only a certain number of instances are
required or allowed.

\remend
\end{remarks}

\vtitle{Limit instances}

\begin{verbatim}
CONSTANT
max_scj : INTEGER := 9;
END_CONSTANT;

ENTITY scj SUBTYPE OF (person);
END_ENTITY;

RULE max_no FOR (scj);
WHERE
  r1: SIZEOF(scj) <= max_scj;
END_RULE;
\end{verbatim}

This rule says that there shall be no more than \verb|max_scj| \texttt{scj}s
(Supreme Court Justices).

\begin{remarks}
\remintro
\remtitle{(cont)}

A similar restriction on numbers of instances.

\remend
\end{remarks}

\clearpage


The following RULE states that there shall be one and only one point at the
origin in the object-base.

\begin{verbatim}
RULE unique_origin FOR (point);
LOCAL
  origin : SET OF point;
END_LOCAL;
  origin := QUERY(temp <* point |
                  (temp.x = 0.0) AND
                  (temp.y = 0.0));
WHERE
  r1: SIZEOF(origin) = 1;
END_RULE;
\end{verbatim}


\begin{remarks}
\remintro
\remtitle{Recursion --- Entity}

    Recursion is when something (apparently) applies itself to itself.

An ENTITY attribute may refer to the ENTITY (as a type).
I have called this `type recursive' and it is a regular part of
modeling. (A person may have a child, who is of course a person).

In the first model an instance of a \texttt{node} may list itself
among its \texttt{children}. This is almost certainly incorrect.

In the second model an instance of a \texttt{node} cannot list
itself among its children, but could be listed among its
grandchildren. This is probably incorrect.

\remend
\end{remarks}

\vtitle{Recursion --- Entity}

    This \texttt{node} entity is `type recursive' and may be 
`instance recursive'

\begin{verbatim}
ENTITY node;
  local_data : data;
  children : LIST OF UNIQUE node;
END_ENTITY;
\end{verbatim} 

    This \texttt{node} entity is `type recursive' and not `self instance
recursive' but may be `globally instance recursive'.

\begin{verbatim}
ENTITY node;
  local_data : data;
  children : LIST OF UNIQUE node;
WHERE
 all_unique : NOT (SELF IN SELF.children);
END_ENTITY;
\end{verbatim} 


\begin{remarks}
\remintro
\remtitle{Recursion --- Function}

A function can call itself, but at some point there must be a condition
that prevents this (in order to prevent an infinite recursion).

The \texttt{NodeSet} function generates the SET consisting of the
input \texttt{node} and all its descendents.

The \texttt{NodeBag} function generates the BAG consisting of the
input \texttt{node} and all its descendents.

\remend
\end{remarks}

\vtitle{Recursion --- Function}

\begin{verbatim}
FUNCTION NodeSet(input: node): SET OF node;
LOCAL
  result : SET OF node := [];
END_LOCAL;
REPEAT i := 1 TO SIZEOF(input.children);
  result := result + NodeSet(input.children[i]);
END_REPEAT;
RETURN(result + input);
END_FUNCTION;
\end{verbatim}

\begin{verbatim}
FUNCTION NodeBag(input: node): BAG OF node;
LOCAL
  result : BAG OF node := [];
END_LOCAL;
REPEAT i := 1 TO SIZEOF(input.children);
  result := result + NodeBag(input.children[i]);
END_REPEAT;
RETURN(result + input);
END_FUNCTION;
\end{verbatim}


\begin{remarks}
\remintro
\remtitle{RULE with recursive functions}

    This RULE checks that any node is not also a descendent of itself.
(\texttt{NodeBag} lists all descendent nodes, including duplicates, 
and \texttt{NodeSet} 
does the same but excludes duplicates).

\remend
\end{remarks}

\vtitle{RULE with recursive functions}

    A tree of nodes must be acyclic. That is, a given node instance must only
appear once in the tree.
\begin{verbatim}
RULE acyclic_tree FOR (node);
LOCAL
  result : LOGICAL;
END_LOCAL;
REPEAT i := 1 TO SIZEOF(node);
  result := SIZEOF(NodeSet(node[i])) = 
            SIZEOF(NodeBag(node[i]));
  IF (result = FALSE)
  THEN SKIP;
  END_IF;
END_REPEAT;
WHERE
  acyclic: result;
END_RULE;
\end{verbatim}


\begin{remarks}
\remintro
\remtitle{Or}

      This does the same, but more concisely and less understandably.
The QUERY returns a BAG of nodes where the SIZEOF the \texttt{NodeSet}s
and \texttt{NodeBag}s are not the same.

The SIZEOF is the number of nodes in the QUERY's BAG, which should be zero.

\remend
\clearpage
\end{remarks}

\clearpage

Or
\begin{verbatim}
RULE acyclic_tree FOR (node);
WHERE
  acyclic: SIZEOF(QUERY(t <* node | 
                  SIZEOF(NodeSet(t)) <> 
                  SIZEOF(NodeBag(t)))
                 ) = 0;
END_RULE;
\end{verbatim}


\begin{remarks}
\remintro
\remtitle{More Recursion}

The next example is taken from the International STEP Standard.

The constraint on \texttt{relationship} instances is that the
\texttt{parent} / \texttt{child} graph is acyclic. Equivalently
ancestors and descendents must unique.

\remend
\end{remarks}

\vtitle{More Recursion}

    This can be used to describe a relationship between two \texttt{obj}
(Part 41, Annex D).
\begin{verbatim}
ENTITY relationship;
  description : STRING;
  parent      : obj;
  child       : obj;
END_ENTITY;
\end{verbatim}
In turn, the \texttt{obj} that is a child in one of these may be the
parent in another \texttt{relationship}, and so on.
    Often it is required that a string of \texttt{relationship} be acyclic. 
More simply, a child cannot be its own ancestor, or equivalently a parent
cannot be its own descendent.

    Use a function in a WHERE rule as:
\begin{verbatim}
WHERE
w1: acyclic(SELF,[SELF.parent],'...');
\end{verbatim}


\begin{remarks}
\remintro
\remtitle{(cont)}

This is a (helper) function that converts an AGGREGATE 
(ARRAY, LIST, BAG or SET) to a SET.

\remend
\end{remarks}

\clearpage

    Convert an AGGREGATE to a SET.
\begin{verbatim}
FUNCTION Agg2Set(agg: AGGREGATE OF GENERIC:a):
                 SET OF GENERIC:a;
LOCAL
  result : SET OF GENERIC:a := [];
END_LOCAL;
REPEAT i := LOINDEX(agg) TO HIINDEX(agg);
  result := result + agg[i];
END_REPEAT;
RETURN(result);
END_FUNCTION;
\end{verbatim}

\begin{remarks}
\remintro
\remtitle{(cont)}

This is the \texttt{acyclic} function defined in STEP. Does it do
what it is meant to? 

An immediate answer is: Who knows?

Seriously, it takes some time to work out if it works.

\remend
\end{remarks}

\clearpage

Does the following (Part 41 p 156) work?
\begin{verbatim}
FUNCTION acyclic(rel: relationship;
                 relatives: SET [1:?] OF obj;
                 subtyp: STRING): LOGICAL;
LOCAL
  x     : SET [1:?] OF relationship;
  close : SET [1:?] OF obj;
END_LOCAL;
REPEAT i := 1 TO HIINDEX(relatives);
  IF rel.parent :=: relatives[i]
  THEN RETURN(FALSE); END_IF;
END_REPEAT;
x := Agg2Set(USEDIN(rel.parent, subtyp));
close := relatives + rel.parent;
REPEAT i := 1 TO SIZEOF(x);
  IF NOT acyclic(x[i],close,subtyp)
    THEN RETURN(FALSE); END_IF;
END_REPEAT;
RETURN(TRUE);
END_FUNCTION;
\end{verbatim}

\begin{comment}

\begin{remarks}
\remintro
\remtitle{Rem}

\remend
\end{remarks}

\clearpage

From Part 43, pp 10 to 12, a rewrite of \texttt{mapped\_item}:
\begin{verbatim}
ENTITY rep;
  items : SET [1:?] OF ri;
  ...
END_ENTITY;

ENTITY rm;
  map    : rep;
  origin : ri;
INVERSE
  usage : SET [1:?] OF mi FOR source;
END_ENTITY;

ENTITY ri;
  name : STRING;
WHERE
 ...
END_ENTITY;

ENTITY mi
  SUBTYPE OF (ri);
  source : rm;
  target : ri;
WHERE
  AcyclicMr(UsingReps(SELF), [SELF]);
END_ENTITY;
\end{verbatim}
Where the function \texttt{UsingReps} returns the set of \texttt{rep} which
reference a given \texttt{ri} (or \texttt{mi}).


\begin{remarks}
\remintro
\remtitle{Rem}

\remend
\end{remarks}

\clearpage

\begin{verbatim}
FUNCTION AcyclicMr(parents : SET OF rep;
                   children : SET OF ri):
         BOOLEAN;
LOCAL
  x, y : SET OF ri;
END_LOCAL;
-- subset of children that are mi
x := QUERY(z <* children |
           'SN.MI' IN TYPEOF(z));
-- check each element
REPEAT i := 1 TO SIZEOF(x);
-- FALSE if element maps a rep in parent set
  IF x[i]\mi.source.map IN parents
  THEN RETURN(FALSE); END_IF;
-- recursive check on the mr elements
  IF NOT AcyclicMr(
    parents + x[i]\mi.source.mr,
    x[i]\mi.source.map.items)
  THEN RETURN(FALSE); END_IF;
END_REPEAT;
-- subset of children that are not mi
x := children - x;
-- check each element
REPEAT i := 1 TO SIZEOF(x);
-- get set of ri referenced
  y := QUERY(z <* Agg2Set(USEDIN(x[i], '')) |
             'SN.RI' IN TYPEOF(z));
-- recursively check for offending mi
  IF NOT AcyclicMr(parents, y)
  THEN RETURN(FALSE); END_IF;
END_REPEAT;
-- no cycles
RETURN(TRUE);
END_FUNCTION;
\end{verbatim}

%% above takes 2 pages
\clearpage

\end{comment}

\begin{remarks}
\remintro
\remtitle{TYPEOF function}

    One of the EXPRESS built-in functions.

    Typically used to check if a variable is of a particular type.

\remend
\end{remarks}

\vtitle{TYPEOF function}

\verb|TYPEOF(V: GENERIC): SET OF STRING;| returns the set of uppercase
strings holding the fully qualified names of the types of which the
value (instance) \texttt{V} could be a value of. That is, the result is
the set of potential uses of \texttt{V}, not the actual usage.

\begin{verbatim}
SCHEMA s;

TYPE mylist = LIST OF REAL; END_TYPE;
...
LOCAL lst : mylist; END_LOCAL;

TYPEOF(lst) = ['S.MYLIST', 'LIST']; -- TRUE
\end{verbatim}
Note that given a subtype instance, the returned set will include the
subtype and all its supertypes, but it excludes subtypes lower in the tree.


\begin{remarks}
\remintro
\remtitle{SIZEOF function}

    One of the EXPRESS built-in functions.

    It returns the number of items in an aggregate.

    In the example, all that it is used for is checking that the
two lists have the same number of entries --- it has nothing to do with
whether or not the third, say, item in each list go together.

    A better model follows for correlating students and marks.

\remend
\end{remarks}

\vtitle{SIZEOF function}

   \verb|SIZEOF(agg)| returns the number of element instances in 
the (aggregate) instance \verb|agg|.

    Usually used for controlling an iteration or for comparing the actual
sizes of two aggregates.


\begin{verbatim}
ENTITY PoorExamMarks;
  course   : STRING;
  students : LIST OF UNIQUE person;
  marks    : LIST OF INTEGER;
WHERE
  matched_lists : SIZEOF(students) = 
                  SIZEOF(marks);
END_ENTITY;
\end{verbatim}

    This has been used as an attempt to specify that there is a
one-to-one correlation between the elements in the two lists.


\begin{remarks}
\remintro
\remtitle{Correlated aggregates}

If a student and a mark go together, then define an ENTITY to
capture this, as in \texttt{BetterExamMarks} and \texttt{StudentMark}.

    This, of course, solves one problem only to create another.

    The new problem is solved by \texttt{BestExamMarks}, and the
function \texttt{UniqueStudents}.

\remend
\end{remarks}

\vtitle{Correlated aggregates}

\begin{verbatim}
ENTITY BetterExamMarks;
  course : STRING;
  results : LIST OF StudentMark;
END_ENTITY;

ENTITY StudentMark;
  student : person;
  mark    : INTEGER;
END_ENTITY;
\end{verbatim}
But what about student uniqueness in \texttt{BetterExamMarks}?

\begin{verbatim}
ENTITY BestExamMarks;
  course : STRING;
  results : LIST OF StudentMark;
WHERE
  wr1: UniqueStudents(results);
END_ENTITY;
\end{verbatim}


\begin{remarks}
\remintro
\remtitle{UniqueStudents}

The function takes a bunch of \texttt{StudentMark} and creates a BAG
of all the students. It also creates a SET of the students and checks
if the BAG and SET are the same size.

\remend
\end{remarks}

\vtitle{UniqueStudents}

\begin{verbatim}
FUNCTION UniqueStudents
         (input: AGGREGATE OF StudentMark): 
         LOGICAL;
LOCAL
  aBag : BAG OF person := [];
END_LOCAL;
REPEAT i := 1 TO SIZEOF(input);
  aBag := aBag + input[i].student;
END_REPEAT;
RETURN (SIZEOF(aBag) = 
        SIZEOF(Agg2Set(aBag)));
END_FUNCTION;
\end{verbatim}


\begin{remarks}
\remintro
\remtitle{QUERY function}

    One of the EXPRESS built-in functions.

    Given an aggregate, it tests every element against a logical condition, 
and puts each element that passes the test into a returned aggregate 
(of the same kind as the input one).


\remend
\end{remarks}

\vtitle{QUERY function}

\verb?QUERY(v <* InAgg | Lexp(v)): OutAgg? \\
applies the logical expression \texttt{Lexp(v)}
to each element of the aggregate \texttt{InAgg}. Each element for which 
\texttt{Lexp} is TRUE is added to the returned aggregate \texttt{OutAgg}, 
which is of the same type as \texttt{InAgg}.
    It is equivalent to the following pseudo-EXPRESS.
\begin{verbatim}
FUNCTION query(input: AGGREGATE OF GENERIC:GEN; 
               LEXP): 
              AGGREGATE OF GENERIC:GEN;
LOCAL
  result : AGGREGATE OF GENERIC:GEN := [];
END_LOCAL;
REPEAT i := LOINDEX(input) TO HIINDEX(input);
  IF Lexp(input[i]) = TRUE
  THEN  result := result + input[i];
  END_IF;
END_REPEAT;
RETURN(result);
END_FUNCTION;
\end{verbatim}


\begin{remarks}
\remintro
\remtitle{Example}

    This model just uses SIZEOF. The next one uses QUERY.

\remend
\end{remarks}

\vtitle{Example}

    A school party must have at least one adult for every 10 children 
and shall not be larger than 50 in total.
\begin{verbatim}
ENTITY SchoolParty;
  adults, children : SET OF person;
WHERE
  w1: 10*SIZEOF(adults) >= SIZEOF(children);
  w2: SIZEOF(adults) + SIZEOF(children) <= 50;
END_ENTITY;
\end{verbatim}


\begin{remarks}
\remintro
\remtitle{(cont)}

    This model uses both SIZEOF and QUERY.

    The assumption here is that a \texttt{person} entity has
an \texttt{age} attribute. The first QUERY grabs all the adults 
and the second grabs all the children.

\remend
\end{remarks}

\clearpage

Or, reformulating the entity and using the QUERY function:

\begin{verbatim}
ENTITY SchoolParty;
  group : SET [2:50] OF person;
WHERE
w1: 10*SIZEOF(QUERY(p <* group | p.age >= 21))
    >=
    SIZEOF(QUERY(p <* group | p.age <= 18));
END_ENTITY;
\end{verbatim}


\begin{remarks}
\remintro
\remtitle{QUERY and SIZEOF}

   These two are often combined. The names of the functions in the
example are meant to indicate the kind of result the QUERY returns.

\begin{itemize}
\item There shall be no bad p's.

\item At most one bad p.

\item At least one ...

\item Between 2 and 5 ...

\item Every one

\end{itemize}

\remend
\end{remarks}

\vtitle{QUERY and SIZEOF}

    \verb|QUERY| and \verb|SIZEOF| functions are often combined.
\begin{verbatim}
SIZEOF(QUERY(p <* e | Bad(p)=TRUE)) = 0;
 
SIZEOF(QUERY(p <* e | MaxOneBad(p)=TRUE)) <= 1;
 
SIZEOF(QUERY(p <* e | AtLeastOne(p)=TRUE)) >0;

{2 <= 
  SIZEOF(QUERY(p <* e | Two2Five(p)=TRUE))
<= 5};

SIZEOF(QUERY(p <* e | AllGood(p)=TRUE))
= SIZEOF(e);
\end{verbatim}


\begin{remarks}
\remintro
\remtitle{USEDIN function}

    One of the EXPRESS built-in functions.

    There is an implied directionality in EXPRESS entities. From an entity
you can `see' what its attributes are but you can't `see' where it is used
as an attribute.

    The USEDIN function returns entity instances where a particular
entity instance is used as a particular attribute.

    You could get the same information from an INVERSE attribute, if there
was one, but USEDIN can be used even if there isn't.
    
\remend
\end{remarks}

\vtitle{USEDIN function}

\texttt{USEDIN(T:GENERIC; R:STRING): BAG OF GENERIC;} returns the BAG of 
entity instances that uses instance \texttt{T} in role \texttt{R}.

   If \texttt{T} plays no roles and/or role \texttt{R} is not found, 
the returned BAG is empty.

    If \texttt{R} is an empty string, every usage of instance \texttt{T} 
is reported.

    Note that the \texttt{USEDIN} function examines instances in an 
object-base. That is, it looks at actual data rather than the potential 
kinds (types) of data.


\begin{remarks}
\remintro
\remtitle{USEDIN example}

    It is not all that asy to work out what a USEDIN is trying to
discover. It's at least doubly difficult if it is part of
a QUERY (which often is embedded in a SIZEOF).

\remend
\end{remarks}

\vtitle{USEDIN example}

\begin{verbatim}
ENTITY PoorEnt;
  attr : PoorColour;
END_ENTITY;

ENTITY PoorColour;
  hue        : fraction;
  saturation : fraction;
  intensity  : fraction;
WHERE
  wr1: SIZEOF(QUERY(x <* 
              USEDIN(SELF, 'POORENT.ATTR') |
       (x.attr.intensity > 0.5))) = 0;
END_ENTITY;
\end{verbatim}
Says that when an instance of \texttt{PoorColour} is used as the 
\texttt{attr} of the entity \texttt{PoorEnt}, then its value for 
\texttt{intensity} shall be not more than half.


\begin{remarks}
\remintro
\remtitle{(cont)}

With a little bit or rework,the model is much cleaner and understandable.
(Why should a constraint by the user be put into the used?)

\remend
\end{remarks}

\clearpage


    This model is better written as:
\begin{verbatim}
ENTITY Ent;
  attr : Colour;
WHERE
  wr1: attr.intensity <= 0.5;
END_ENTITY;

ENTITY Colour;
  hue        : fraction;
  saturation : fraction;
  intensity  : fraction;
END_ENTITY;
\end{verbatim}



\begin{remarks}
\remintro
\remtitle{(cont)}

    An INVERSE could be used instead of the USEDIN, but this again obscures
the intent.

\remend
\end{remarks}

\clearpage

Or, it could be rewritten using an inverse.

\begin{verbatim}
ENTITY Ent;
  attr : Colour;
END_ENTITY;

ENTITY Colour;
  hue        : fraction;
  saturation : fraction;
  intensity  : fraction;
INVERSE
  low : BAG OF Ent FOR attr;
WHERE
  w1: (SIZEOF(low) > 0 AND 
       intensity <= 0.5) XOR
      (SIZEOF(low) = 0);
END_ENTITY;
\end{verbatim}


\begin{remarks}
\remintro
\remtitle{USEDIN example}

   This kind of thing is scattered throughout STEP 
(and encouraged to boot).

    The RULE is intended to say that \texttt{ent} cannot be
independently instantiated --- it is a second-class entity.

\remend
\end{remarks}

\vtitle{USEDIN example}

\begin{verbatim}
RULE SecondClass FOR (ent);
WHERE
  wr1: SIZEOF(QUERY(e <* ent |
              NOT (SIZEOF(USEDIN(e,'')) >= 1 )))
       = 0;
END_RULE;
\end{verbatim}
states that \texttt{ent} shall not be independently instantiated.
\begin{itemize}
\item \verb|USEDIN(e,'')| gives entities that reference instance 
      \texttt{e} of entity type \texttt{ent}
\item \verb|SIZEOF(USEDIN(e,'')) >= 1| gives number of entities 
      referencing \texttt{e}
\item \verb|NOT (SIZEOF...)| gives an \texttt{e} that is not referenced
\item and there should be none of these.
\end{itemize}

\begin{remarks}
\remintro
\remtitle{(cont)}

    There is no need for the RULE as it is exactly the semantics
of REFERENCE import into a SCHEMA.

\remend
\end{remarks}

\clearpage

    The semantics of this rule are exactly the same as the EXPRESS REFERENCE
construct.

\begin{verbatim}
SCHEMA good;        SCHEMA ap;
REFERENCE FROM sub    ENTITY ent;
          (ent);        ...
  ...                   ...
END_SCHEMA;           END_ENTITY;
SCHEMA sub;
ENTITY ent;           RULE SecondClass FOR
   ...                                 (ent);
END_ENTITY;             ...
...
END_SCHEMA;           END_SCHEMA;
\end{verbatim}


\begin{remarks}
\remintro
\remtitle{ROLESOF function}

    One of the EXPRESS built-in functions.

    Another of the functions that examine the object base. Given
an entity instance, it returns the names of the
entities, and the atribute names, where it is used as an attribute.


    The model is the basis for an example which follows.

\remend
\end{remarks}

\vtitle{ROLESOF function}

\verb|ROLESOF(V:GENERIC): SET OF STRING;| returns the set of roles that the
instance \texttt{V} plays in the object base.

\begin{verbatim}
SCHEMA uk;
ENTITY judge;
  office_holder : person;
  court         : STRING;
END_ENTITY;

ENTITY criminal;
  prisoner : person;
  gaol     : address;
  crime    : ...
END_ENTITY;
\end{verbatim}

\begin{remarks}
\remintro
\remtitle{(cont)}

    Quite sensibly, in the UK a judge must not in jail. (This
model would be incorrect in (parts of) the United States).

    There must be no instance where a \texttt{person} simultaneously
plays the role of \texttt{office\_holder} in \texttt{judge} and
the role of \texttt{prisoner} in \texttt{criminal}.

\remend
\end{remarks}

\clearpage

    In the UK schema, a person who is a judge shall not be a prisoner in gaol.
\begin{verbatim}
RULE NoCriminalJudge FOR (person);
WHERE
wr1: SIZEOF(QUERY(p <* person | 
      'UK.CRIMINAL.PRISONER' IN ROLESOF(p) 
      AND
      'UK.JUDGE.OFFICE_HOLDER' IN ROLESOF(p))
     ) = 0;
END_RULE;
\end{verbatim}


\begin{remarks}
\remintro
\remtitle{Required Optional Attributes}

    Now two examples about putting constraints on the presence
or absence of values for optional attributes.

    An example of how to specify that at least one among several optional
attributes must be present.

\remend
\end{remarks}

\vtitle{Required Optional Attributes}

  At least one of the optional attributes must have a value:

\begin{verbatim}
ENTITY ent;
  attr1 : OPTIONAL ...;
  attr2 : OPTIONAL ...;
WHERE
 at_least_one : EXISTS(attr1) OR
                EXISTS(attr2);
END_ENTITY;
\end{verbatim}


\begin{remarks}
\remintro
\remtitle{(cont)}

    Only one of the attributes must have a value.

\remend
\end{remarks}

\clearpage

  One and only one of the optional attributes must have a value:

\begin{verbatim}
ENTITY ent;
  attr1 : OPTIONAL ...;
  attr2 : OPTIONAL ...;
WHERE
 only_one : EXISTS(attr1) XOR
            EXISTS(attr2);
END_ENTITY;
\end{verbatim}


\begin{remarks}
\remintro
\remtitle{Attribute Redeclaration}

    A SUBTYPE can specialise inherited attrubutes (i.e., limit
the potential kinds and/or numbers of values).

\remend
\end{remarks}

\vtitle{Attribute Redeclaration}

\begin{verbatim}
ENTITY t;
  a : LIST OF d;
  b : NUMBER;
END_ENTITY;

ENTITY sub
  SUBTYPE OF (t);
  SELF\t.a : LIST [1:4] OF UNIQUE e;
  SELF\t.b : INTEGER;
END_ENTITY;

ENTITY e SUBTYPE OF d;
...
END_ENTITY;
\end{verbatim}


\begin{remarks}
\remintro
\remtitle{(cont)}

    On the other hand, if you want to confuse your readers you could
do this.

\remend
\end{remarks}

\clearpage

    Instead of:

\begin{verbatim}
ENTITY sub
  SUBTYPE OF (t);
WHERE
  w1: 'INTEGER' IN TYPEOF(SELF\t.b);
  w2: {1 <= SIZEOF(SELF\t.a) <= 4};
  w3: SIZEOF(SELF\t.a) = 
      SIZEOF(Agg2Set(SELF\t.a));
--  w4: subtyping of list elements
END_ENTITY;
\end{verbatim}


\begin{remarks}
\remintro
\remtitle{SUMMARY}

\remend
\end{remarks}

\vtitle{SUMMARY}

\begin{itemize}
\item An EXPRESS information model is permissive (i.e. what is not explicitly
prohibited is permissable).

\item Minimise constraints (enhances re-useability).

\item Add all necessary constraints --- a model is as much about the
limitations of objects as about the objects themselves.

\item Specify constraints by the following ordered preferences:

\begin{enumerate}
\item Model structure
\item Local constraints
\item Global rules
\end{enumerate}

\end{itemize}

\endinput



