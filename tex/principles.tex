% principles.tex   EXPRESS Course Modeling Principles
% Created by Peter R Wilson, 1992 -- 2004
%\documentclass[11pt]{article}
%\usepackage{vgraph}
%\usepackage{headfoot}

%%%%\notestrue
%%%%\notesfalse

%\begin{document}

\pagestyle{empty}
\bodsiz

\vspace*{\beftit}
\begin{center}
{\titsiz\bfseries Information Modeling Using EXPRESS --- \\ Modeling Principles }
\end{center}
\vspace{\beftit}
\vspace{\beftit}
\begin{center}
%%%\textbf{Peter R. Wilson}
\end{center}

\clearpage

\ifnotes
  \pagestyle{plain}
\else
  \pagestyle{empty}
%%%  \pagefooter{}{}{{\small Peter R. Wilson}}
\fi

\vtitle{SUMMARY}

\begin{itemize}
\item Organize the team who will
\item Develop the model while
\item Applying the principles.
\end{itemize}

\vtitle{THE TEAM}

    The modeling team consists of three groups:
\begin{enumerate}
\item Technical experts who know the model domain.
\item Modeling experts who help the technical experts develop the model.
\item Reviewers, who are domain experts, to `keep the developers honest'
\end{enumerate}

\vtitle{Remember ---}

    `A theory has only the alternative of being right or wrong. A model has
a third possibility: it may be right but irrelevant.'

\begin{normalsize}
Manfred Eigen in \textit{The Physicists Conception of Nature}
(ed. Jagdish Mehra) Reidel, 1973
\end{normalsize}

\vtitle{PROCESS}

\begin{itemize}
\item Define the scope
\item `Find the objects'
\item `Sketch' the structure
\item Generalisation and specialisation
\item Refine the structure
\item Eliminate synonyms and homonyms
\item Cardinality constraints
\item Local constraints
\item Global constraints
\item Partition
\end{itemize}

\vtitle{PRINCIPLES}

\begin{itemize}
\item Scoping
\item Readability
\item Simple Types
\item Implementation Independence
\item Partitioning
\item Schema Independence
\item Nym Principle
\item Context Independence
\item Constraints
\item Invariance
\item Redundancy
\item Inheritance
\end{itemize}

\vtitle{Readability}

    An information model, although if defined via EXPRESS is computer
interpretable, should primarily be human understandable.

    Modeling constructs should be chosen to aid the reader rather
than obfuscate understanding by using complex, intertwined or
opaque definitional relationships.

\vtitle{Scoping}

    A model must have a well defined scope (otherwise how do you know
when you are finished?).

    Before modeling:
\begin{itemize}
\item Define the intended model scope.
\item Document the assumptions.
\end{itemize}

\vtitle{Simple Types}

    The EXPRESS simple types (i.e Integer, String, etc) carry virtually no
semantics.

    As far as possible, avoid the use of simple types as entity attributes.
Instead, use Types to provide the semantics.
\begin{verbatim}
ENTITY antique_object;
  produced : date;
  maker    : name;
DERIVE
  years_old : age := elapsed(produced);
END_ENTITY;

TYPE name = STRING; END_TYPE;
TYPE age = INTEGER; END_TYPE;
\end{verbatim}


\vtitle{Implementation Independence}

    An information model defines the \textbf{information} necessary for some
purpose; \emph{it does not define the data structuring}.

\begin{itemize}
\item Model objects, not syntax.
\item Model the `real world' not a technology-based representation.
\item Do not define `precision' of data types.
\item Do not define `default' data.
\end{itemize}

\vtitle{Partitioning}

\begin{itemize}
\item A \verb?SCHEMA? defines a scope and a context.
\item A model is often partitioned into a set of schemas (e.g to increase
readability, to partition the development work, etc.)
\item Minimise the interaction between schemas.
\item Within a schema, minimise the constraints on the objects in question (to
      promote re-usability).
\end{itemize}


\vtitle{Schema Independence}

    In EXPRESS, each Schema defines a scope; definition names need only be
unique within a Schema.

    Attempt to maintain name uniqueness across all schemas in a model (see the
Nym Principle). This will assist when restructuring a model, if necessary,
by modifying the schema boundaries.

\vtitle{Nym Principle}

    `If things are the same, then they should have the same name.'

    `If things are not the same, then they are different.'

    `Different things should have different names.'

    In general, `one name, one meaning, one definition'. Synonyms and homonyms
in a model are a fruitful and never-ending source of confusion.

\vtitle{Context Independence}

    Each entity exists in a context in which it may be used. This may vary from
extremely broad to highly specific. An entity definition should be as context
independent as possible, yet as context specific as required.

\begin{itemize}
\item Only apply the minimum necessary number of constraints.
\item Use Subtyping to get more specificity.
\end{itemize}


\vtitle{Invariance}

    The meaning of an entity should not be dependent on the values of its
attributes. Do not use `flags' to change meanings.
\begin{verbatim}
ENTITY poor_person_model;
  sex : enumeration_of_male_female;
  ... -- gender related attributes
  ... -- non-gender attributes
END_ENTITY;

ENTITY good_person_model
  SUPERTYPE OF (ONEOF(male, female));
  ... -- non-gender attributes
END_ENTITY;
  -- gender related attributes put into
  -- the relevant subtypes
\end{verbatim}

\vtitle{Constraints}

    An EXPRESS information model is \emph{permissive}
(i.e what is not explicitly prohibited is permissable).

    Add all necessary constraints --- a model is as much about the limitations
of objects as it is about the objects themselves.

\begin{verbatim}
TYPE age = INTEGER;
WHERE
  non_negative : SELF >= 0;
END_TYPE;
\end{verbatim}

\vtitle{Constraint ordering}

    Specify constraints by the following ordered preferences:

\begin{enumerate}
\item Model structure
\item Local constraints
\item Global rules
\end{enumerate}

\vtitle{Constraint (global rule)}

\begin{verbatim}
ENTITY male SUBTYPE OF (person);
  wife : OPTIONAL female;
  -- other attributes
END_ENTITY;

ENTITY female SUBTYPE OF (person);
  husband : OPTIONAL male;
  -- other attributes
END_ENTITY;

RULE married FOR (male, female);
  -- check declared husbands
  -- and wives match each other
END_RULE;
\end{verbatim}

\vtitle{Constraint (local)}


\begin{verbatim}
ENTITY male SUBTYPE OF (person);
  wife : OPTIONAL female;
  -- other attributes
WHERE
  -- check wife says she is
  -- married to me
END_ENTITY;

ENTITY female SUBTYPE OF (person);
  husband : OPTIONAL male;
  -- other attributes
WHERE
  -- check husband says he is
  -- married to me
END_ENTITY;
\end{verbatim}

\vtitle{Constraint (structural)}


\begin{verbatim}
ENTITY male SUBTYPE OF (person);
  -- other attributes
END_ENTITY;

ENTITY female SUBTYPE OF (person);
  -- other attributes
END_ENTITY;

ENTITY married;
  husband : male;
  wife    : female;
UNIQUE
  no_bigamy    : husband;
  no_polyandry : wife;
END_ENTITY;
\end{verbatim}

\vtitle{Redundancy}

    A model should not contain redundant information; redundancy leads to
the possibility of data inconsistencies.

\begin{verbatim}
ENTITY circle;
  center : point;
  radius : REAL;
DERIVE
  perimeter : REAL := 2.0*PI*radius;
  diameter : REAL := 2.0*radius;
END_ENTITY;
\end{verbatim}

\vtitle{Inheritance}

    A Subtype inherits all the properties of its Supertype.

    For readability it may appear desirable to migrate the common properties
down to the leaves of the supertype tree. This, however, implies that the
common properties are semantically different.

    All common properties should be moved as close to the root of the Supertype
tree as possible. This demonstrates that they ARE common.

\vmintitle{SCHEMA INTERFACING}

\begin{verbatim}
SCHEMA first;
  ENTITY aaa;
    -- attributes
  END_ENTITY;

  ENTITY original;
    attr : NUMBER;
  END_ENTITY;
END_SCHEMA; -- first

SCHEMA second;
  USE FROM first (aaa AS bbb);
  REFERENCE FROM first (original);

  ENTITY constrained
    SUBTYPE OF (original);
    attr : INTEGER(7);
    WHERE
      positive : attr > 0;
    END_ENTITY;
END_SCHEMA; -- second
\end{verbatim}

\endinput


