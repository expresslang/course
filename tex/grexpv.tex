% grexpv.tex               EXPRESS-G VIEWGRAPHS
\documentclass[11pt]{article}
\usepackage{vgraph}
\usepackage{headfoot}
\usepackage{ifpdf}
\usepackage{graphicx}
% \DeclareGraphicsRule{*}{mps}{*}{}

%%%%\notestrue
\notesfalse

\begin{document}

% grexp.tex               EXPRESS-G Course
% Created by Peter R Wilson, 1992 -- 2004
%\documentclass[11pt]{article}
%\usepackage{vgraph}
%\usepackage{headfoot}
%\usepackage{graphicx}
%\usepackage{ifpdf}
%\ifpdf
%  \DeclareGraphicsRule{*}{mps}{*}{}
%\fi

%%%%\notestrue
%%%%\notesfalse
%
%\begin{document}


\pagestyle{empty}
\bodsiz

\vspace*{\beftit}
\begin{center}
{\titsiz \textbf{THE EXPRESS-G LANGUAGE}}
\end{center}
\vspace{\beftit}
\vspace{\beftit}
\begin{center}
%%%\textbf{Peter R. Wilson}
\end{center}

\clearpage

\ifnotes
  \pagestyle{plain}
\else
  \pagestyle{empty}
%%%  \pagefooter{}{}{{\small Peter R. Wilson}}
\fi

\begin{remarks}
\remintro
NB. NOTES ARE ON EVEN NUMBERED (LEFT HAND) PAGES AND THE PRESENTATION ON
ODD NUMBERED (RIGHT HAND) PAGES.

\remtitle{EXPRESS-G}

This part of the course covers the elements of EXPRESS-G:
\begin{itemize}
\item History
\item Design goals
\item Language elements and examples
\end{itemize}

    There were three main proposals for a graphical version of
EXPRESS before the EXPRESS Committee started work in earnest. This meant
that they had some user experiences to guide them.

\remend
\end{remarks}

\vtitle{EXPRESS-G}

    EXPRESS-G is a graphical subset of EXPRESS lexical language and
has evolved from a variety of sources.

\begin{description}
\item[10/88]: EXPRESSAM (Bruce Lownsbery, Lawrence Livermore Lab)
              based on NIAM.
\item[2/89]: GREXP (Peter Wilson, GE) BLA  (boxes, lines, annotation) with
             multiple levels of abstraction and any amount of text.
\item[11/89]: ExpressGraph (Curt Parks, NIST \& John Zimmerman, Allied Bendix)
              simplified version of EXPRESSAM.
\item[2/90]: EXPRESS-G (EXPRESS Committee)
\end{description}

\begin{remarks}
\remintro
\remtitle{EXPRESS-G}

    The final result is yet another member of the BLA (boxes,
lines, and annotations) family. Unlike practically any other BLA
language it was based on an existing lexical language --- the others
were stand-alone.

    EXPRESS-G is limited to the structural parts of EXPRESS --- no
representation of constraints, no FUNCTIONs.

\remend
\end{remarks}


\vtitle{EXPRESS-G}

\begin{itemize}
\item Drafted by EXPRESS Committee in January 1990
\item A combination of ExpressGraph and GREXP.
\item A \emph{subset} of EXPRESS.
\item Modeling of Entity, Attribute, Cardinality, Supertype, Type and Schema.
\item Limited levels of abstraction.
\item No ISO ballot issues in mid 1990 (Informative Annex).
\item February 1991 --- Normative Annex to EXPRESS document and
extended.
\item Same standardization status as EXPRESS.
\end{itemize}

\begin{remarks}
\remintro
\remtitle{EXPRESS-G design Goals}

    As there was a full lexical language available EXPRESS-G was limited
to those things that could be easily represented graphically.

    It was hoped that the diagrams could be understood with (practically)
no training. (Hopefully even non-technical managers would feel comfortable).

    Different levels of abstraction (detail) would be available.

    Importantly, it was known that EXPRESS-G models would be published
in reports on normal sized paper, so there had to be a means of splitting
a large diagram across a number of pages.

    The icons had to be simple so simple drawing programs could be used
(at one time diagrams were drawn using ASCII art on lineprinters).

    It would be useful to be able to develop software to go between
EXPRESS and EXPRESS-G modeling.


\remend
\end{remarks}

\vtitle{EXPRESS-G Design Goals}

\begin{itemize}
\item Subset of EXPRESS.
\item Intuitive understanding of diagrams.
\item Support different levels of abstraction.
\item Mechanism for multi-page models.
\item Minimal graphic capabilities needed.
\item Potential automatic EXPRESS-G display of EXPRESS models.
\end{itemize}


\begin{remarks}
\remintro
\remtitle{EXPRESS-G Example Model}

This is meant to be understandable by the audience --- at least the general
tenor of the model.

    It is a `2 page' model even though it is shown on one sheet of paper.

\normalsize
\begin{verbatim}
  TYPE date = ARAY [1:3] OF INTEGER; END_TYPE;

  TYPE hair_type = ENUMERATION OF
    (blonde, brown, black, red, white, bald);
  END_TYPE;

  ENTITY person;
    first_name : STRING;
    last_name : STRING;
    nickname : STRING;
    birth_date : date;
    children : SET [0:?] OF person;
    hair : hair_type;
  DERIVE
    age : INTEGER := years(birth_date);
  INVERSE
    parents : SET [0:2] OF person FOR children;
  END_ENTITY;

  SUBTYPE_CONSTRAINT sc_person FOR person;
    ONEOF(male, female);
  END_SUBTYPE_CONSTRAINT;

  ENTITY female SUBTYPE OF (person);
  INVERSE
    husband : SET [0:1] OF male FOR wife;
  END_ENTITY;

  ENTITY male SUBTYPE OF (person);
    wife : OPTIONAL female;
  END_ENTITY;

  FUNCTION years(past : date): INTEGER
    (* returns number of years between past
       and curent date *)
  END_FUNCTION;
\end{verbatim}

\remend
\end{remarks}

\vtitle{EXPRESS-G Example Model}

\begin{figure}[hp]
\normalsize
\centering
\includegraphics{models.1}

\vspace{\afttit}

PAGE 1 OF 2

\vspace{\afttit}
\vspace{\afttit}

\includegraphics{models.2}

\vspace{\afttit}

PAGE 2 OF 2
\end{figure}

\begin{remarks}
\remintro
\remtitle{Definition Symbols}

These are the symbols for representing the structural elements of EXPRESS,
principally TYPE, ENTITY and SCHEMA.

SUBTYPE\_CONSTRAINT is a late addition.

\remend
\end{remarks}

%\normalsize
%\bodsiz
\ifnotes
  \vspace*{-2\beftit}
\fi
\vtoptitle{Definition Symbols}


\begin{figure}[hp]
\centering
\includegraphics{symbols.1}

\ifnotes
  \vspace{\baselineskip}
\else
  \vspace{\afttit}
\fi

BASE TYPES

\ifnotes
  \vspace{\baselineskip}
\else
  \vspace{\afttit}
\fi

\includegraphics{symbols.4}

\ifnotes
  \vspace{\baselineskip}
\else
  \vspace{\afttit}
\fi

\includegraphics{symbols.8}

\ifnotes
  \vspace{\baselineskip}
\else
  \vspace{\afttit}
\fi

DEFINED TYPES

\ifnotes
  \vspace{\baselineskip}
\else
  \vspace{\afttit}
\fi

\includegraphics{symbols.9}

\ifnotes
  \vspace{\baselineskip}
\else
  \vspace{\afttit}
\fi

ENTITY

\ifnotes
  \vspace{\baselineskip}
\else
  \vspace{\afttit}
\fi

\includegraphics{symbols.17}

\ifnotes
  \vspace{\baselineskip}
\else
  \vspace{\afttit}
\fi

SUBTYPE\_CONSTRAINT

\ifnotes
  \vspace{\baselineskip}
\else
  \vspace{\afttit}
\fi

\includegraphics{symbols.13}

\ifnotes
  \vspace{\baselineskip}
\else
  \vspace{\afttit}
\fi

SCHEMA
\end{figure}


\begin{remarks}
\remintro
\remtitle{Relationship Symbols}

    Lines are used to indicate relationships between definitions.

    The thickness of the line is meant to be indicative of the strength
of the relationship.

\begin{itemize}
\item Thick lines for supertype/subtype relationship
\item Dashed line for an optional attribute of an ENTITY.
\end{itemize}

\remend
\end{remarks}

\vtitle{Relationship Symbols}

\begin{figure}[hp]
\centering
\includegraphics{symbols.14}

\vspace{\afttit}
LINE STYLES

\end{figure}


\begin{remarks}
\remintro
\remtitle{Composition Symbols}

There are 2 kinds of compostion symbol
\begin{itemize}
\item Page connectors, where a relationship line crosses to or from
another page.
\item Inter-schema references where something is defined in some other
schema than the current one.
\end{itemize}

\remend
\end{remarks}

\vtitle{Composition Symbols}

\begin{figure}[hp]
\centering
\includegraphics{symbols.15}

\vspace{\afttit}

PAGE REFERENCES

\vspace{\afttit}

\includegraphics{symbols.16}

\vspace{\afttit}

INTER-SCHEMA REFERENCES

\end{figure}


\begin{remarks}
\remintro
\remtitle{A Supertype Tree}
\begin{verbatim}
SCHEMA simple_tree;

ENTITY super; END_ENTITY;

ENTITY sub1 SUBTYPE OF (super); END_ENTITY;

ENTITY sub2 SUBTYPE OF (super); END_ENTITY;

SUBTYPE_CONSTRAINT sc_sub2 FOR sub2;
  ABSTRACT;
  ONEOF(sub3, sub4);
END_SUBTYPE_CONSTRAINT;

ENTITY sub5 SUBTYPE OF (super); END_ENTITY;

ENTITY sub3 SUBTYPE OF (sub2); END_ENTITY;

ENTITY sub4 SUBTYPE OF (sub2); END_ENTITY;

END_SCHEMA; -- simple_tree
\end{verbatim}

\remend
\end{remarks}

\vtitle{A Supertype Tree}

\begin{figure}[hp]
\centering
\includegraphics{models.5}

\vspace{\afttit}

\end{figure}


\begin{remarks}
\remintro
\remtitle{Retyping attributes}

\begin{verbatim}
ENTITY sup_a;
  attr : sub_b;
END_ENTITY;

ENTITY sub_a SUBTYPE OF (sup_a);
  SELF\sup_a.sub_b : sub_b;
END_ENTITY;

ENTITY sup_b;
  num : OPTIONAL NUMBER;
END_ENTIY;

ENTITY sub_b SUBTYPE OF (sup_b);
  SELF\sup_b.num : REAL;
END_ENTIY;
\end{verbatim}


\remend
\end{remarks}

\vtitle{Retyping attributes}

\begin{figure}[hp]
\centering
\includegraphics{models.6}

\vspace{\afttit}

\end{figure}



\begin{remarks}
\remintro
\remtitle{Partial and Complete Entity Models}

\begin{verbatim}
ENTITY super; END_ENTITY;

ENTITY sub_1 SUBTYPE OF (super);
  attr : from_ent;
END_ENTITY;

ENTITY sub_2 SUBTYPE OF (super);
  pick : choice;
END_ENTITY;

ENTITY an_ent;
  int : INTEGER;
END_ENTITY;

ENTITY from_ent;
  description : OPTIONAL to_ent;
  values      : ARRAY [1:3] OF UNIQUE REAL;
END_ENTITY;

ENTITY to_net;
  text : strings;
END_ENTITY;

TYPE choice = SELECT
  (an_ent, name);
END_TYPE;

TYPE name = STRING; END_TYPE;

TYPE strings  LIST [1:?] OF STRING; END_TYPE;
\end{verbatim}

\remend
\end{remarks}

\vmintitle{Partial and Complete Entity Models}

\begingroup

\normalsize
\ifnotes
%  \enlargethispage{1in}
\fi

%\begin{figure}[hp]
\centering
\includegraphics{models.3}

\vspace{\afttit}

PARTIAL ENTITY LEVEL MODEL

\vspace{\afttit}

%\end{figure}

%\begin{figure}[hp]
\centering
\includegraphics{models.4}

\vspace{\afttit}

COMPLETE ENTITY LEVEL MODEL

\endgroup

%\end{figure}

\begin{remarks}
\remintro
\remtitle{Schema and Entity Models}

\normalsize
\begin{verbatim}
SCHEMA top;
  USE FROM geom (curve, point AS node);
  REFERENCE FROM geom (surface);

  ENTITY face;
    bounds : LIST [1:?] OF loop;
    loc    : surface;
  END_ENTITY;

  ENTITY loop; END_ENTITY;

  SUBTYPE_CONSTRAINT sc_loop FOR loop;
    ABSTRACT;
    ONEOF(eloop, vloop);
  END_SUBTYPE_CONSTRAINT;

  ENTITY eloop SUBTYPE OF (loop);
    bound : LIST [1:?] OF edge;
  END_ENTITY;

  ENTITY vloop SUBTYPE OF (loop);
    bound : vertex;
  END_ENTITY;

  ENTITY edge;
    start, end : vertex;
    loc   : curve;
  END_ENTITY;

  ENTITY vertex;
    loc : node;
  END_ENTITY;
END_SCHEMA; -- top

SCHEMA geom;
  ENTITY lcs; END_ENTITY;
  ENTITY surface; END_ENTITY;
  ENTITY curve; END_ENTITY;
  ENTITY point; END_ENTITY;
END_SCHEMA; -- geom
\end{verbatim}

\remend
\end{remarks}


\vtoptitle{Schema and Entity Models}

\begin{figure}[hp]
\centering
\includegraphics{models.8}

\vspace{\afttit}

SCHEMA LEVEL MODEL

\vspace{\afttit}

\includegraphics{models.7}

\vspace{\afttit}

ENTITY LEVEL MODEL

\end{figure}

\begin{remarks}
\remintro
\remtitle{Subtype Constraints}

\begin{verbatim}
ENTITY p; END_ENTTY;  -- person

SUBTYPE_CONSTRAINT p_subs FOR p;
  TOTAL_OVER(m, f);
  ONEOF(m, f) AND ONEOF(c, a);
END_SUBTYPE_CONSTRAINT;

ENTITY m SUBTYPE OF (p); END_ENTITY; -- male

ENTITY f SUBTYPE OF (p); END_ENTITY; -- female

ENTITY c SUBTYPE OF (p); END_ENTITY; -- citizen

ENTITY a SUBTYPE OF (p); END_ENTITY; -- alien

SUBTYPE_CONSTRAINT no_li FOR a;
  ABSTRACT SUPERTYPE;
  ONEOF(l, i);
END_SUBTYPE_CONSTRAINT;

ENTITY l SUBTYPE OF (a); END_ENTITY; -- legal

ENTITY i SUBTYPE OF (a); END_ENTITY; -- illegal
\end{verbatim}

\remend
\end{remarks}


\vtitle{Subtype constraints}

\begin{figure}[hp]
\centering
\includegraphics{models.10}

\vspace{\afttit}

\end{figure}


\begin{remarks}
\remintro
\remtitle{Usage}

    May be used anywhere EXPRESS is used.

    Can be used alone in its own right, but more usually employed with
EXPRESS lexical to fill in any missing details.

\remend
\end{remarks}

\vtitle{Usage}

%\bodsiz
\begin{itemize}
\item Graphical display of EXPRESS models.
\item Stand-alone information modeling iconic language.
\item Model display at varying levels of abstraction.
\item Model display at varying levels of granularity.
\end{itemize}

\endinput






\end{document}

\pagestyle{empty}
\bodsiz

\vspace*{\beftit}
\begin{center}
{\titsiz \textbf{THE EXPRESS-G LANGUAGE}}
\end{center}
\vspace{\beftit}
\vspace{\beftit}
\begin{center}
\textbf{Peter R. Wilson}
\end{center}

\clearpage

\ifnotes
  \pagestyle{plain}
\else
  \pagefooter{}{}{{\small Peter R. Wilson}}
\fi

\begin{remarks}
\remintro
NB. NOTES ARE ON EVEN NUMBERED (LEFT HAND) PAGES AND THE PRESENTATION ON
ODD NUMBERED (RIGHT HAND) PAGES.

\remtitle{EXPRESS-G}

This part of the course covers the elements of EXPRESS-G:
\begin{itemize}
\item History
\item Design goals
\item Language elements and examples
\end{itemize}

    There were three main proposals for a graphical version of
EXPRESS before the EXPRESS Committee started work in earnest. This meant
that they had some user experiences to guide them.

\remend
\end{remarks}

\vtitle{EXPRESS-G}

    EXPRESS-G is a graphical subset of EXPRESS lexical language and
has evolved from a variety of sources.

\begin{description}
\item[10/88]: EXPRESSAM (Bruce Lownsbery, Lawrence Livermore Lab)
              based on NIAM.
\item[2/89]: GREXP (Peter Wilson, GE) BLA  (boxes, lines, annotation) with
             multiple levels of abstraction and any amount of text.
\item[11/89]: ExpressGraph (Curt Parks, NIST \& John Zimmerman, Allied Bendix)
              simplified version of EXPRESSAM.
\item[2/90]: EXPRESS-G (EXPRESS Committee)
\end{description}

\begin{remarks}
\remintro
\remtitle{EXPRESS-G}

    The final result is yet another member of the BLA (boxes,
lines, and annotations) family. Unlike practically any other BLA
language it was based on an existing lexical language --- the others
were stand-alone.

    EXPRESS-G is limited to the structural parts of EXPRESS --- no
representation of constraints, no FUNCTIONs.

\remend
\end{remarks}


\vtitle{EXPRESS-G}

\begin{itemize}
\item Drafted by EXPRESS Committee in January 1990
\item A combination of ExpressGraph and GREXP.
\item A \emph{subset} of EXPRESS.
\item Modeling of Entity, Attribute, Cardinality, Supertype, Type and Schema.
\item Limited levels of abstraction.
\item No ISO ballot issues in mid 1990 (Informative Annex).
\item February 1991 --- Normative Annex to EXPRESS document and
extended.
\item Same standardization status as EXPRESS.
\end{itemize}

\begin{remarks}
\remintro
\remtitle{EXPRESS-G design Goals}

    As there was a full lexical language available EXPRESS-G was limited
to those things that could be easily represented graphically.

    It was hoped that the diagrams could be understood with (practically)
no training. (Hopefully even non-technical managers would feel comfortable).

    Different levels of abstraction (detail) would be available.

    Importantly, it was known that EXPRESS-G models would be published
in reports on normal sized paper, so there had to be a means of splitting
a large diagram across a number of pages.

    The icons had to be simple so simple drawing programs could be used
(at one time diagrams were drawn using ASCII art on lineprinters).

    It would be useful to be able to develop software to go between
EXPRESS and EXPRESS-G modeling.


\remend
\end{remarks}

\vtitle{EXPRESS-G Design Goals}

\begin{itemize}
\item Subset of EXPRESS.
\item Intuitive understanding of diagrams.
\item Support different levels of abstraction.
\item Mechanism for multi-page models.
\item Minimal graphic capabilities needed.
\item Potential automatic EXPRESS-G display of EXPRESS models.
\end{itemize}


\begin{remarks}
\remintro
\remtitle{EXPRESS-G Example Model}

This is meant to be understandable by the audience --- at least the general
tenor of the model.

    It is a `2 page' model even though it is shown on one sheet of paper.

\normalsize
\begin{verbatim}
  TYPE date = ARAY [1:3] OF INTEGER; END_TYPE;

  TYPE hair_type = ENUMERATION OF
    (blonde, brown, black, red, white, bald);
  END_TYPE;

  ENTITY person;
    first_name : STRING;
    last_name : STRING;
    nickname : STRING;
    birth_date : date;
    children : SET [0:?] OF person;
    hair : hair_type;
  DERIVE
    age : INTEGER := years(birth_date);
  INVERSE
    parents : SET [0:2] OF person FOR children;
  END_ENTITY;

  SUBTYPE_CONSTRAINT sc_person FOR person;
    ONEOF(male, female);
  END_SUBTYPE_CONSTRAINT;

  ENTITY female SUBTYPE OF (person);
  INVERSE
    husband : SET [0:1] OF male FOR wife;
  END_ENTITY;

  ENTITY male SUBTYPE OF (person);
    wife : OPTIONAL female;
  END_ENTITY;

  FUNCTION years(past : date): INTEGER
    (* returns number of years between past
       and curent date *)
  END_FUNCTION;
\end{verbatim}

\remend
\end{remarks}

\vtitle{EXPRESS-G Example Model}

\begin{figure}[hp]
\normalsize
\centering
\includegraphics{models.1}

\vspace{\afttit}

PAGE 1 OF 2

\vspace{\afttit}
\vspace{\afttit}

\includegraphics{models.2}

\vspace{\afttit}

PAGE 2 OF 2
\end{figure}

\begin{remarks}
\remintro
\remtitle{Definition Symbols}

These are the symbols for representing the structural elements of EXPRESS,
principally TYPE, ENTITY and SCHEMA.

SUBTYPE\_CONSTRAINT is a late addition.

\remend
\end{remarks}

%\normalsize
%\bodsiz
\ifnotes
  \vspace*{-2\beftit}
\fi
\vtoptitle{Definition Symbols}


\begin{figure}[hp]
\centering
\includegraphics{symbols.1}

\ifnotes
  \vspace{\baselineskip}
\else
  \vspace{\afttit}
\fi

BASE TYPES

\ifnotes
  \vspace{\baselineskip}
\else
  \vspace{\afttit}
\fi

\includegraphics{symbols.4}

\ifnotes
  \vspace{\baselineskip}
\else
  \vspace{\afttit}
\fi

\includegraphics{symbols.8}

\ifnotes
  \vspace{\baselineskip}
\else
  \vspace{\afttit}
\fi

DEFINED TYPES

\ifnotes
  \vspace{\baselineskip}
\else
  \vspace{\afttit}
\fi

\includegraphics{symbols.9}

\ifnotes
  \vspace{\baselineskip}
\else
  \vspace{\afttit}
\fi

ENTITY

\ifnotes
  \vspace{\baselineskip}
\else
  \vspace{\afttit}
\fi

\includegraphics{symbols.17}

\ifnotes
  \vspace{\baselineskip}
\else
  \vspace{\afttit}
\fi

SUBTYPE\_CONSTRAINT

\ifnotes
  \vspace{\baselineskip}
\else
  \vspace{\afttit}
\fi

\includegraphics{symbols.13}

\ifnotes
  \vspace{\baselineskip}
\else
  \vspace{\afttit}
\fi

SCHEMA
\end{figure}


\begin{remarks}
\remintro
\remtitle{Relationship Symbols}

    Lines are used to indicate relationships between definitions.

    The thickness of the line is meant to be indicative of the strength
of the relationship.

\begin{itemize}
\item Thick lines for supertype/subtype relationship
\item Dashed line for an optional attribute of an ENTITY.
\end{itemize}

\remend
\end{remarks}

\vtitle{Relationship Symbols}

\begin{figure}[hp]
\centering
\includegraphics{symbols.14}

\vspace{\afttit}
LINE STYLES

\end{figure}


\begin{remarks}
\remintro
\remtitle{Composition Symbols}

There are 2 kinds of compostion symbol
\begin{itemize}
\item Page connectors, where a relationship line crosses to or from
another page.
\item Inter-schema references where something is defined in some other
schema than the current one.
\end{itemize}

\remend
\end{remarks}

\vtitle{Composition Symbols}

\begin{figure}[hp]
\centering
\includegraphics{symbols.15}

\vspace{\afttit}

PAGE REFERENCES

\vspace{\afttit}

\includegraphics{symbols.16}

\vspace{\afttit}

INTER-SCHEMA REFERENCES

\end{figure}


\begin{remarks}
\remintro
\remtitle{A Supertype Tree}
\begin{verbatim}
SCHEMA simple_tree;

ENTITY super; END_ENTITY;

ENTITY sub1 SUBTYPE OF (super); END_ENTITY;

ENTITY sub2 SUBTYPE OF (super); END_ENTITY;

SUBTYPE_CONSTRAINT sc_sub2 FOR sub2;
  ABSTRACT;
  ONEOF(sub3, sub4);
END_SUBTYPE_CONSTRAINT;

ENTITY sub5 SUBTYPE OF (super); END_ENTITY;

ENTITY sub3 SUBTYPE OF (sub2); END_ENTITY;

ENTITY sub4 SUBTYPE OF (sub2); END_ENTITY;

END_SCHEMA; -- simple_tree
\end{verbatim}

\remend
\end{remarks}

\vtitle{A Supertype Tree}

\begin{figure}[hp]
\centering
\includegraphics{models.5}

\vspace{\afttit}

\end{figure}


\begin{remarks}
\remintro
\remtitle{Retyping attributes}

\begin{verbatim}
ENTITY sup_a;
  attr : sub_b;
END_ENTITY;

ENTITY sub_a SUBTYPE OF (sup_a);
  SELF\sup_a.sub_b : sub_b;
END_ENTITY;

ENTITY sup_b;
  num : OPTIONAL NUMBER;
END_ENTIY;

ENTITY sub_b SUBTYPE OF (sup_b);
  SELF\sup_b.num : REAL;
END_ENTIY;
\end{verbatim}


\remend
\end{remarks}

\vtitle{Retyping attributes}

\begin{figure}[hp]
\centering
\includegraphics{models.6}

\vspace{\afttit}

\end{figure}



\begin{remarks}
\remintro
\remtitle{Partial and Complete Entity Models}

\begin{verbatim}
ENTITY super; END_ENTITY;

ENTITY sub_1 SUBTYPE OF (super);
  attr : from_ent;
END_ENTITY;

ENTITY sub_2 SUBTYPE OF (super);
  pick : choice;
END_ENTITY;

ENTITY an_ent;
  int : INTEGER;
END_ENTITY;

ENTITY from_ent;
  description : OPTIONAL to_ent;
  values      : ARRAY [1:3] OF UNIQUE REAL;
END_ENTITY;

ENTITY to_net;
  text : strings;
END_ENTITY;

TYPE choice = SELECT
  (an_ent, name);
END_TYPE;

TYPE name = STRING; END_TYPE;

TYPE strings  LIST [1:?] OF STRING; END_TYPE;
\end{verbatim}

\remend
\end{remarks}

\vmintitle{Partial and Complete Entity Models}

\begingroup

\normalsize
\ifnotes
%  \enlargethispage{1in}
\fi

%\begin{figure}[hp]
\centering
\includegraphics{models.3}

\vspace{\afttit}

PARTIAL ENTITY LEVEL MODEL

\vspace{\afttit}

%\end{figure}

%\begin{figure}[hp]
\centering
\includegraphics{models.4}

\vspace{\afttit}

COMPLETE ENTITY LEVEL MODEL

\endgroup

%\end{figure}

\begin{remarks}
\remintro
\remtitle{Schema and Entity Models}

\normalsize
\begin{verbatim}
SCHEMA top;
  USE FROM geom (curve, point AS node);
  REFERENCE FROM geom (surface);

  ENTITY face;
    bounds : LIST [1:?] OF loop;
    loc    : surface;
  END_ENTITY;

  ENTITY loop; END_ENTITY;

  SUBTYPE_CONSTRAINT sc_loop FOR loop;
    ABSTRACT;
    ONEOF(eloop, vloop);
  END_SUBTYPE_CONSTRAINT;

  ENTITY eloop SUBTYPE OF (loop);
    bound : LIST [1:?] OF edge;
  END_ENTITY;

  ENTITY vloop SUBTYPE OF (loop);
    bound : vertex;
  END_ENTITY;

  ENTITY edge;
    start, end : vertex;
    loc   : curve;
  END_ENTITY;

  ENTITY vertex;
    loc : node;
  END_ENTITY;
END_SCHEMA; -- top

SCHEMA geom;
  ENTITY lcs; END_ENTITY;
  ENTITY surface; END_ENTITY;
  ENTITY curve; END_ENTITY;
  ENTITY point; END_ENTITY;
END_SCHEMA; -- geom
\end{verbatim}

\remend
\end{remarks}


\vtoptitle{Schema and Entity Models}

\begin{figure}[hp]
\centering
\includegraphics{models.8}

\vspace{\afttit}

SCHEMA LEVEL MODEL

\vspace{\afttit}

\includegraphics{models.7}

\vspace{\afttit}

ENTITY LEVEL MODEL

\end{figure}

\begin{remarks}
\remintro
\remtitle{Subtype Constraints}

\begin{verbatim}
ENTITY p; END_ENTTY;  -- person

SUBTYPE_CONSTRAINT p_subs FOR p;
  TOTAL_OVER(m, f);
  ONEOF(m, f) AND ONEOF(c, a);
END_SUBTYPE_CONSTRAINT;

ENTITY m SUBTYPE OF (p); END_ENTITY; -- male

ENTITY f SUBTYPE OF (p); END_ENTITY; -- female

ENTITY c SUBTYPE OF (p); END_ENTITY; -- citizen

ENTITY a SUBTYPE OF (p); END_ENTITY; -- alien

SUBTYPE_CONSTRAINT no_li FOR a;
  ABSTRACT SUPERTYPE;
  ONEOF(l, i);
END_SUBTYPE_CONSTRAINT;

ENTITY l SUBTYPE OF (a); END_ENTITY; -- legal

ENTITY i SUBTYPE OF (a); END_ENTITY; -- illegal
\end{verbatim}

\remend
\end{remarks}


\vtitle{Subtype constraints}

\begin{figure}[hp]
\centering
\includegraphics{models.10}

\vspace{\afttit}

\end{figure}


\begin{remarks}
\remintro
\remtitle{Usage}

    May be used anywhere EXPRESS is used.

    Can be used alone in its own right, but more usually employed with
EXPRESS lexical to fill in any missing details.

\remend
\end{remarks}

\vtitle{Usage}

%\bodsiz
\begin{itemize}
\item Graphical display of EXPRESS models.
\item Stand-alone information modeling iconic language.
\item Model display at varying levels of abstraction.
\item Model display at varying levels of granularity.
\end{itemize}


\end{document}


