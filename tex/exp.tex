% exp.tex   EXPRESS Course "Info Modeling using EXP --- Introduction"
% Created by Peter R Wilson, 1992 -- 2004
%\documentclass[11pt]{article}
%\usepackage{vgraph}
%\usepackage{headfoot}
%\usepackage{ifpdf}
%
%\begin{document}
%

\pagestyle{empty}
\bodsiz

\vspace*{\beftit}
\begin{center}
{\titsiz \bfseries Information Modeling Using EXPRESS --- \\ An Introduction }
\end{center}
\vspace{\beftit}
\vspace{\beftit}
\begin{center}
%%\textbf{Peter R. Wilson}
\end{center}

\clearpage

\ifnotes
  \pagestyle{plain}
\else
  \pagestyle{empty}
%%%  \pagefooter{}{}{{\small Peter R. Wilson}}
\fi


\begin{remarks}
\remintro
NB. NOTES ARE ON EVEN NUMBERED (LEFT HAND) PAGES AND THE PRESENTATION ON
ODD NUMBERED (RIGHT HAND) PAGES.

\remtitle{Information?}


    This first part of the EXPRESS course covers the following topics:
\begin{itemize}
\item What is information, why do we need information models, and
      what is an information model?
\item EXPRESS background
\item STEP architecture
\item TYPE
\item ENTITY
\item FUNCTION
\item SCHEMA
\end{itemize}

    We are basically concerned with storing and transmitting information
in computer systems. In particular information about engineering
data and processes. 

\remend
\end{remarks}

\vtitle{Information?}

\begin{description}
\item[Definition:] \emph{Information is knowledge of ideas, facts and/or 
                        processes.}
\end{description}

\begin{itemize}
\item Information can be exchanged between partners.
\item Exchange may be real time or delayed.
\item Information storage is a special case of communication.
\end{itemize}

\begin{remarks}
\remintro
\remtitle{Information Model?}

    Because we are dealing with computer systems we need a precise
definition of the information of interest. One goal is to be able to
exchange information between Computer Aided Design (CAD), Computer Aided
Engineering (CAE), Computer Aided Manufacturing (CAM), etc., systems
so they can interoperate.

    The model may describe things in general, for example an animal has
legs, and there can be data corresponding to the model about particular
things --- my cat Jessie has three legs (but she can still catch birds).

\remend
\end{remarks}

\vtitle{Information Model?}

\begin{description}
\item[Definition:] \emph{An information model is a formal description of types 
(classes) of ideas, facts and processes which together form a model of a 
portion of interest of the real world.}
\end{description}

\begin{itemize}
\item If a model is written in EXPRESS or any other computer sensible 
representation, it has the additional property of being \emph{computer 
processible}.
\item An information model may be \emph{instantiated} or \emph{populated} to
represent \textbf{particular} ideas, facts and processes.
\item An information model that is specialized to take account of a particular
instantiation method is a \emph{concrete} model. One that is implementation
independent is a \emph{conceptual} model.
\end{itemize}

\begin{remarks}
\remintro

\remtitle{Why Formal Models?}

    In everday conversation, and writing, there can be many ambiguities
and we are good at deciding which meaning is meant. Computers are
notoriously bad at this.

    We tend to use the surrounding context to help in disambiguation
and if this is missing we have difficulties.

\begin{itemize}
\item     The first two example sentences are not ambiguous (to us) but
looking at the pair \ldots

\item     The third sentence is ambiguous just by itself.

\item    Even a single word (braces) can be ambiguous.

\item     Numerical data (1-3-91) can be ambiguous.
\end{itemize}

    There are some exercises about ambiguities.

\remend
\end{remarks}

\vtitle{Why Formal Models?}

\begin{itemize}
\item Time flies like an arrow.
\item Fruit flies like a banana.
\item Joe saw the man with the telescope.
\item Braces
  \begin{description}
  \item[UK] Hold up one's trousers.
  \item[USA] Adjust one's teeth.
  \end{description}
\item The date 1--3--91
  \begin{description}
  \item[UK] 1st March 1991 (or is it 1891?)
  \item[USA] 3rd January 1991
  \end{description}
\end{itemize}

\begin{remarks}
\remintro

\remtitle{Information Model}

    This is a somewhat formal definition, but then we are going to
be spending a lot of time with formal models and modeling.

    It is intended to enable unambiguous communication about a limited
topic (or range of topics).

\remend
\end{remarks}

\vtitle{Information Model}

\begin{description}
\item[Purpose:] To identify clearly the objects in a `Universe of Discourse'
in order to enable people (and potentially computer systems) to communicate
precisely about a domain of common interest.

\item[Comprises:] An information model is composed of 
  \begin{itemize}
  \item objects
  \item relationships
  \item constraints
  \end{itemize}
which, taken together, provide a complete and unambiguous formal 
representation of a Universe of Discourse.
\end{description}

\begin{remarks}
\remintro

\remtitle{IM is NOT}

   We are interested in computer based/processible information models.

   A model uses, or is associated with, various more common computer
techniques, but it is essentially for human consumption.

\remend
\end{remarks}

\vtitle{IM is NOT}

An Information Model is 
\begin{itemize}
\item NOT a database definition (even though terms such as \emph{schema} are 
      common.
\item NOT a data structure definition (even though data instances of the model
      could be structured)
\item NOT a program (even though procedural code and algorithms may be in
      the model)
\end{itemize}

    A populated instance of an IM may be maintained using DB or similar 
technologies. IM constraints are often implemented via programatic code.

\begin{remarks}
\remintro

\remtitle{IM Description Methods}

    Historically, formal information models have been specified using
either a written (lexical) language or using a graphical (drawings) 
language. 

    The graphic constructs are usually boxes and lines connecting the boxes, 
together with some annotations on the diagram. 

    A graphical model can easily be the size of a wall, which might cause
difficulties if you ant to put one in a report.


\remend
\end{remarks}

\vtitle{IM Description Methods}

An Information Model may be described:
\begin{description}
\item[Textually] using a formally defined lexical language. Examples include
EXPRESS, IISyCL (Integrated Information Systems Constraint Language), VDM
(Vienna Development Method), etc.
\item[Graphically] using an iconic or diagramatic language such as EXPRESS-G,
IDEF1X, OMT, UML, etc.
\end{description}

NOTE: Supplementing textual models with diagrams can help the reader's
understanding. Graphical models nearly always require supplemental text for
completeness.


\begin{remarks}
\remintro

\remtitle{EXPRESS Development}

   EXPRESS has been used, one way or another, for 20 years
or so.

   The requirement was for use in specifying industry and international
standards.

   Other modeling techniques were reviewed but did not have the power
that was felt to be needed, in particular constraint specifications.
Also the languages were basically graphical although there were some
proprietry lexical adjuncts.



\remend
\end{remarks}

\vtitle{EXPRESS Development}

    EXPRESS developed as an information modeling language to meet the needs
of product data exchange model definition.

\begin{itemize}
\item First version, called DSL, developed under the USAF funded PDDI 
      program (early '80s).
\item PDES reviewed NIAM and IDEF1X. Neither had the power needed.
\item PDES started extending EXPRESS.
\item STEP mandated all `Normative' models to be in EXPRESS.
\item Language still evolving.
\end{itemize}

\begin{remarks}
\remintro

\remtitle{REVIEWS}

EXPRESS has been formally approved as an International Standard,
specifically:

 ISO 10303-11 \textit{Industrial automation systems and integration ---
                      Product data representation and exchange ---
                      Part 11: Description method: The EXPRESS language
                      reference manual}

    The first edition was formally approved and published in 1994.

    The second edition should be published during 2004.


\remend
\end{remarks}

\vtitle{REVIEWS}

    The language is subject to ongoing review within STEP and by other users.
Also international public review as part of ISO standardization:

\begin{description}
\item[Early 1989] ISO Draft Proposal ballot % (the Tokyo edition).
%\item[Mid 1990] ISO Committee Draft ballot % (the Gothenburg edition).
\item[Mid 1991] ISO Committee Draft ballot % (the San Diego edition).
\item[Oct 1991] Ballot successful --- Draft International Standard status.
\item[Mid 1993] Approved for registration as an International Standard
                (ISO 10303 Part 11).
\item[End 1994] Published as International Standard ISO 10303-11:1994.
\item[End 2003] Edition 2 approved as an International Standard.
\end{description}

\begin{remarks}
\remintro

\remtitle{Language Comparison}

 Most modeling anguages are graphical, which is inherently limiting.

For data modeling most languages are targeted towards Relational Databases.
Examples include IDEF1X, Shlaer-Mellor, Extended Entity-Relation.

   UML is for modeling an Object Oriented program. EXPRESS is for modeling
data and naturally moved to an OO perspective (it was developed by practising
engineers as user, not by computer scientists).

\remend
\end{remarks}



\vtoptitle{LANGUAGE COMPARISON}

\begin{center}
\begin{tabular}{|l|c|c|} \hline
Characteristic & Others     & EXPRESS \\ \hline
\multicolumn{3}{|c|}{Modeling} \\ \hline
Form           & Graphics   & Programmatic \\  
Flavor         & Relational & OO \\
Objects        &  X         & X \\
Relationships  &  X         & X \\
Attributes     &  X         & X \\
Derived Atts.  &            & X \\
Domain         & Entity     & Entity + Type \\
Sequencing     &            & X \\
Cardinalities  & Limited    & Any \\  \hline
\multicolumn{3}{|c|}{Constraints} \\ \hline
Domain         & Limited    & Any \\
Roles          & Limited    & Any \\
Categorization & Limited    & Broad \\ \hline
\multicolumn{3}{|c|}{Miscellaneous} \\ \hline
Multi-page     & Some       & X \\
Algorithms     &            & X \\
Scoping        &            & X \\ \hline
\end{tabular} 
\end{center}

\begin{remarks}
\remintro

\remtitle{Graphical Models}

    Very good for group work --- sketch on blackboard, but soon run
out of space on the board. I have seen complete models that can
take up a whole wall even with small print.

    It's difficult to check a model except by eyeballing it. It's been
a general experience over several decades of going from flowcharts to
program code that many details get missed.

    It is difficult to formally specify a graphical language. 

\remend
\end{remarks}

\vtitle{Graphical Models}

\begin{itemize}
\item Excellent for group explanations and work.
\item Easy to follow (but can take a lot of wall space).
\item Model development may be superficial (it looks right).
\item Some drawing tools may exist, or can use CAD system.
\item Effectively, not computer processible (What You See Is All You've Got).
\end{itemize}

\begin{remarks}
\remintro

\remtitle{Textual Models}

    Text languages for modeling can be formally defined, both syntax and much
of the semantics. This means that they can be made computer processible and so 
can be automatically checked for correctness (syntax) and completeness.

    They can represent a variety of modeling approaches, from mathematical
or logical schemes to things more readily understood.

    They can include a programming language so constraints can be expressed
in terms of a process as well as in terms of rules and regulations.

   They provide opportunities for models to be manipulated, for example 
automatically developing test cases or checking that data conforms to the
model.   

\remend
\end{remarks}

\vtitle{Textual Models}

\begin{itemize}
\item Good formal definition or mathematical support.
\item May be non-intuitive (e.g logic based methods).
\item Complex constraints and rules.
\item Computer processible.
\item Syntax and semantic checking.
\item Potential for automatic implementation (for model simulation and test).
\end{itemize}

\begin{remarks}
\remintro

\remtitle{EXPRESS is}

    NIAM and IDEF1X are both graphical languages for modeling Relational 
databases.

    EXPRESS started as a single lexical language but has since expanded
into a family of languages.

    It was developed by a small group (about 4 at any given time) for
modeling the kinds of information used in engineering. CAD models, Blueprints,
Mechanisms, Engineering sign-off, and so on.

    There were releases every quarter to a user group of about 50, who were full
of their own suggestions and merrily changed the language in between times.
In the first years there were no compilers (the language was changing too 
rapidly) so there were no technical constraints --- every use of the language
was perfect, no bugs, no complaints!

    One of the strengths of EXPRESS is that it much of it was developed by the 
end users. That is also probaly its major weakness as its initial coherence
sank under the weight.

\remend
\end{remarks}

\vtitle{EXPRESS is:}

\begin{itemize}
\item A language family for representing an information model.
\item Computer processible.
\item Under development since early '80s.
\item Superset of NIAM and IDEF1X representation capabilities.
\item Exhibits an object oriented flavor.
\item Been an ISO standard since 1994 (2nd Edition 2004)
\item Has several aspects (subsets)
\end{itemize}

\begin{remarks}
\remintro

\remtitle{EXPRESS Aspects}

    The principal elements of EXPRESS are for representing things
and the relationships between things (and as far as EXPRESS is concerned,
a relationship is a thing). Groups of strongly related things can be
collected together.

    It includes a Pascal-like programming language for specifying complex
constraints.

    It is a conceptual moeling language, so puts no restrictions on the 
number of characters in a name, and arithmetic is infinitely precise.

   There is a graphical form called EXPRESS-G which is a subset of the lexical
language.

   Another member of the family EXPRESS-I is a lexical language for displaying
data that correspond to the concepts in EXPRESS.

    Much more recently the third lexical language EXPRESS-X has been developed
in which you can specify desired changes to an EXPRESS model and then have
them performed; transformations principally consist of splitting or merging
things and their relationships.

\remend
\end{remarks}

\vtitle{EXPRESS Aspects}

%\begin{description}
%\item[CURRENT]:
  \begin{itemize}
  \item Textual language.
  \item Modeling of things and relationships (implementation independent).
  \item Algorithms for arbitrary constraint specifications.
  \item Modeling of implementation dependent data structures.
  \item Graphical form as a subset of textual form (EXPRESS-G).
  \item An `instantiation' format (EXPRESS-I).
  \item Transformation specification (EXPRESS-X).
  \end{itemize}
%\item[PLANNED]:
%  \begin{itemize}
%  \item Methods
%  \item Model mapping.
%  \item Miscellaneous extensions.
%  \end{itemize}
%\end{description}

%\vtitle{SOFTWARE TOOLS}
%
%    30+ organizations have developed 40+ tools and systems for processing
%EXPRESS.
%\begin{itemize}
%%n\item Canada, Germany, Netherlands, UK, USA, etc
%\item Government, Industry, Academia
%\item Proprietry, Public Domain, License and Commercial.
%\item Categories
%  \begin{itemize}
%  \item Editors
%  \item Parsers
%  \item Compilers
%  \item System building tools
%  \item Modeling systems
%  \end{itemize}
%\end{itemize}


\begin{remarks}
\remintro

\remtitle{EXPRESS Usage}

    EXPRESS is widely used in the Standards community for formal definition
of data-related concepts.

\remend
\end{remarks}

\vtoptitle{EXPRESS Usage}


\begin{itemize}
\item Definition of the STEP models
      (200+ people from 20+ countries)
\item Reverse engineering of a DBMS system
\item Software Specification Document for a CAD geometry processor
%\item DICE information modeling
\item Electronic standards (VHDL, EDIF, CFI etc)
%\item Integral part of EIS specification
\item Many European ESPRIT projects
\item Data Definition Language for OO Database
\item Geological modeling
\item Genome modeling
\end{itemize}

    Other uses are possible, such as using EXPRESS to define the syntax, 
grammer, and semantics of the EXPRESS language.

\begin{remarks}
\remintro

\remtitle{STEP History}

\normalsize

The story starts in the mid 1970's with a small group trying to develop
an ANSI standard for geometry data. At the end of the 70's McAuto (part
of McDonnel Douglas) got a contract from CAM-I (Computer Aided Manufacturing
--- International) to develop a standard for data exchange between solid 
modeling systems; the result was not well received.

    Just after this Boeing (Walt Braithwaite), GE (Phil Kennicott) and
the then National Bureau of Standards (Roger Nagel) produced IGES ---
Initial Graphics Exchange Specification for data exchange between CAD 
(Computer Aided Drawing) systems. This was reluctantly implemented by the
major CAD vendors and rapidly became the ANSI Y14.6M standard (the last
section of which was the McAuto work). Then came a proliferation of standards.

    As IGES was not written in France the French published their SET standard.
CAM-I still wanted a solid model data exchange mechanism and came up with
the XBF (Experimental Boundary File), an extension of IGES, which itself 
was going through several 
expansions. The Germans produced VDAFS specifically for sculptured surfaces
as used for car bodies. The XBF work moved under the IGES umbrella and became
ESP (Experimental Solids Proposal).

    The USAF gave McDonnell Douglas a 2 part contract to (a) for a small 
amount of money determine if IGES met USAF (and industry) requiremnts
and if the did not (b) for a large amount of money develop something that did.
Unsurprisingly they determined that IGES was unsuitable and so came up with
the PDDI standard. There was also yet another effort going on in Europe
called the CAD*I project funded under the ESPRIT program.

    IGES was experiencing growing pains and it seemed sensible to make a fresh
start. Boeing (Kal Brauner and Dave Briggs) proposed PDES --- Product Data 
Exchange Standard based on the best work from the US. In particular they
strongly urged that it should have a formal basis.

    Somehow the international community got together and demanded just one
standard --- STEP, Standard for the Exchange of Product Model Data, to be 
based on the technical work from the PDES group. 

    After a while some countries got upset as they felt that it had become a
US standard (even though most participants were non-US). This dilemma was 
eventually resolved by changing PDES to be --- Product Data Exchange using 
STEP (which some then called Standard for Exchange using PDES).

\remend
\end{remarks}

\cleardoublepage
%\vtitle{STEP history}
\normalsize
%\input{/auto/home/pwilson/Rpi/Figs/figshist}
%\input{figshist}
\begin{figure}
\centering
% pstphist.tex  Standards history on mm grid  fig:shist
%%%\begin{figure}[htbp]
\setlength{\unitlength}{1mm}
%\centering
%%%\begin{picture}(150,200) \input{/rdrc/design/pwilson/Pbooks/Ebook/Bkfigs/gridmml}
\begin{picture}(136,150)
\thicklines

\put(0,0){\framebox(135,7.5){{\bf ISO STEP Standard}}}
\put(75,15){\framebox(15,7.5){IGES-n}}
\put(75,30){\framebox(15,7.5){IGES-3}}
\put(45,45){\framebox(15,7.5){CAD*I}}
\put(60,60){\framebox(15,7.5){PDES}}
\put(90,60){\framebox(15,7.5){ESP}}
\put(30,75){\framebox(15,7.5){PDDI}}
\put(75,75){\framebox(15,7.5){IGES-2}}
\put(15,90){\framebox(15,7.5){VDAFS}}
\put(90,90){\framebox(15,7.5){XBF}}
\put(0,105){\framebox(15,7.5){SET}}
\put(75,105){\framebox(15,7.5){IGES-1}}
\put(105,105){\framebox(30,7.5){ANSI Y14.26M}}
\put(87.5,120){\framebox(20,7.5){McAuto}}
\put(87.5,135){\framebox(20,7.5){ANSI draft}}

%  connecting lines
% draft/McAuto
\put(97.5,135){\vector(0,-1){7.5}}
% McAuto/XBF
\put(97.5,120){\vector(0,-1){11.25}}
\put(97.5,108.75){\vector(0,-1){11.25}}
% IGES/Y14
\put(90,108.75){\vector(1,0){15}}
% Y14/STEP
\put(120,105){\vector(0,-1){97.5}}
% IGES/SET
\put(75,108.75){\vector(-1,0){60}}
% SET/STEP
\put(7.5,105){\vector(0,-1){97.5}}
% IGES1/IGES2
\put(82.5,105){\vector(0,-1){22.5}}
% IGES/VDAFS
\put(82.5,93.75){\vector(-1,0){52.5}}
% VDAFS/STEP
\put(22.5,90){\vector(0,-1){82.5}}
% IGES2/PDDI
\put(75,78.75){\vector(-1,0){30}}
% PDDI/STEP
\put(37.5,75){\vector(0,-1){67.5}}
% XBF/ESP
\put(97.5,90){\vector(0,-1){22.5}}
% IGES2/ESP
\put(90,78.75){\vector(1,0){7.5}}
% IGES2/IGES3
\put(82.5,75){\vector(0,-1){37.5}}
% IGES2/PDES
\put(82.5,63.75){\vector(-1,0){7.5}}
% PDDI/PDES
\put(37.5,63.75){\vector(1,0){22.5}}
% CAD*I/STEP
\put(52.5,45){\vector(0,-1){37.5}}
% PDES/STEP
\put(67.5,60){\vector(0,-1){52.5}}
% CAD*I/PDES
\put(60,18.75){\vector(1,0){7.5}}
\put(60,18.75){\vector(-1,0){7.5}}
% ESP/IGES/PDES
\put(97.5,26.25){\line(0,1){33.75}}
\put(97.5,26.25){\vector(-1,0){15}}
\put(82.5,26.25){\vector(-1,0){15}}
% IGES3/IGESn
\put(82.5,30){\vector(0,-1){7.5}}
% IGESn/STEP
\put(82.5,15){\vector(0,-1){7.5}}
% IGESn/Y14
\put(90,18.75){\vector(1,0){30}}

\end{picture}
%%\caption{The genealogy of the STEP product data transfer standard.}
%%\label{fig:shist}
\setlength{\unitlength}{1pt}
%%\end{figure}

\endinput



%%\caption{The genealogy of the STEP product data transfer standard}
%%\label{fig:shist}
\end{figure}
\bodsiz

\begin{remarks}
\remintro

\remtitle{STEP Documents}

    The STEP standard, ISO 10303, is really a suite of cooperating standards
each member of which is a \emph{Part} of ISO 10303.

    The Parts are grouped into \emph{series}.
\begin{itemize}
\item Parts in the range 11--19 form the \emph{Description Methods} series, 
      which include the EXPRESS family.
\item Parts in the range 21--29 form the \emph{Implemantation Methods} series
      defining how to exchange data that corresponds to an EXPRESS model.
\item Parts in the range 31--39 form the \emph{Conformance and Testing} series
      defining how to test STEP implementations.
\item Parts in the range 41--99 form the \emph{Resources} series which define
      an integrated set of application independent EXPRESS information models 
      for product descriptions.
\item Parts in the range 201+ form the \emph{Application Protocol} (AP) series
      which specify application dependent information models for the purposes
      of data exchange.
\end{itemize}


\remend
\end{remarks}

\clearpage
%\vspace*{\aftit}
%\begin{center}
%{\normalsize \bf STEP DOCUMENTS}
%\end{center}
%\vspace{\afttit}
\normalsize
%\input{[pdes.pdes01.texfig]figstepover}
%\input{/rdrc/design/pwilson/Figs/figstepover}
%\input{/auto/home/pwilson/Rpi/Figs/figstepover}
%\input{figstepover}
\begin{figure}
\centering
%  pstpover.tex  Overview of STEP documentation architecture
%%\begin{figure}[htp]
\setlength{\unitlength}{0.2in}
%%\centering
\begin{picture}(30,33)
\small
%\input{grid02}
\thicklines

\put(0,0){\begin{picture}(6,32)
  \put(0,0){\begin{picture}(1,32)
    \put(0,0){\framebox(1,32){}}
    \put(0.25,16){{\bf \shortstack{D\\ E\\ S\\ C\\ R\\ I\\ P\\ T\\ I\\ O\\ N}}}
    \put(0.1,8.5){{\bf \shortstack{M\\ E\\ T\\ H\\ O\\ D\\ S}}}
    \end{picture}}
  \multiput(0,0)(-0.04,0.04){3}{\framebox(6,32){}}
  \put(1,24){\begin{picture}(5,8)
    \put(0,0){\framebox(5,8){}}
    \put(2.5,4.5){\makebox(0,0){\#11}}
    \put(2.5,3.5){\makebox(0,0){EXPRESS}}
    \put(2.5,2.5){\makebox(0,0){EXPRESS-G}}
    \end{picture}}
  \put(1,16){\begin{picture}(5,8)
    \put(0,0){\framebox(5,8){}}
    \put(2.5,4.5){\makebox(0,0){\#12}}
    \put(2.5,3.5){\makebox(0,0){EXPRESS-I}}
    \end{picture}}
  \put(1,8){\framebox(5,8){}}
  \put(1,0){\framebox(5,8){}}
  \end{picture}}

\put(24,0){\begin{picture}(6,32)
  \put(5,0){\begin{picture}(1,32)
    \put(0,0){\framebox(1,32){}}
    \put(0.1,16.5){{\bf \shortstack{C\\ O\\ N\\ F\\ O\\ R\\ M\\ A\\ N\\ C\\ E}}}
    \put(0.25,13.5){{\bf \&}}
%    \put(0.25,8.5){{\bf \shortstack{T\\ O\\ O\\ L\\ S}}} 
    \put(0.25,6.5){{\bf \shortstack{T\\ E\\ S\\ T\\ I\\ N\\ G}}}
    \end{picture}}
  \multiput(0,0)(0.04,0.04){3}{\framebox(6,32){}}
  \put(0,24){\begin{picture}(5,8)
    \put(0,0){\framebox(5,8){}}
    \put(2.5,5){\makebox(0,0){\#31}}
    \put(2.5,4){\makebox(0,0){Testing}}
    \put(2.5,3){\makebox(0,0){Concepts}}
    \end{picture}}
  \put(0,16){\framebox(5,8){\#32}}
  \put(0,0){\framebox(5,16){}}
  \end{picture}}

\put(7,6){\begin{picture}(16,26)
  \multiput(0,0)(0.04,0.04){3}{\framebox(16,26){}}
  \put(0,21){\begin{picture}(16,5)
    \put(0,4){\framebox(16,1){{\bf APPLICATION PROTOCOLS}}}
    \put(1,0){\framebox(14,4){}}
    \put(1,0){\framebox(4,4){\#201}}
    \put(5,0){\framebox(4,4){\#203}}
    \put(9,2){\framebox(4,2){\#205}}
    \put(9,0){\framebox(2,2){\#202}}
    \put(11,0){\framebox(2,2){\#204}}
    \put(13,0){\framebox(2,4){}}
    \end{picture}}
  \end{picture}}

\put(7,0){\begin{picture}(16,5)
  \put(0,0){\framebox(16,1){{\bf IMPLEMENTATION METHODS}}}
  \multiput(0,0)(-0.04,-0.04){3}{\framebox(16,5){}}
  \put(0,1){\begin{picture}(4,4)
    \put(0,0){\framebox(4,4){}}
    \put(2,1){\makebox(0,0){File}}
    \put(2,2){\makebox(0,0){Physical}}
    \put(2,3){\makebox(0,0){\#21}}
    \end{picture}}
  \put(4,1){\begin{picture}(4,4)
    \put(0,0){\framebox(4,4){}}
    \put(2,1){\makebox(0,0){Form}}
    \put(2,2){\makebox(0,0){Working}}
    \put(2,3){\makebox(0,0){\#22}}
    \end{picture}}
  \put(8,1){\begin{picture}(4,4)
    \put(0,0){\framebox(4,4){}}
    \put(2,1){\makebox(0,0){Base}}
    \put(2,2){\makebox(0,0){Data}}
    \end{picture}}
  \put(12,1){\begin{picture}(4,4)
    \put(0,0){\framebox(4,4){}}
    \put(2,1){\makebox(0,0){Base}}
    \put(2,2){\makebox(0,0){Knowledge}}
    \end{picture}}
  \end{picture}}

\put(8,7.5){\begin{picture}(14,18)
  \put(0,17){\framebox(14,1){{\bf INTEGRATED RESOURCES}}}
  \multiput(0,0)(0.04,0.04){3}{\framebox(14,18){}}
  \put(0,16){\framebox(14,1){{\bf Application Resources}}}
  \put(0,13){\begin{picture}(4,3)
    \put(0,0){\framebox(4,3){}}
    \put(2,2){\makebox(0,0){\#101}}
    \put(2,1){\makebox(0,0){Draughting}}
    \end{picture}}
  \put(4,13){\begin{picture}(4,3)
    \put(0,0){\framebox(4,3){}}
    \put(2,2){\makebox(0,0){\#104}}
    \put(2,1){\makebox(0,0){FEM}}
    \end{picture}}
  \put(8,13){\framebox(3,3){}}
  \put(11,13){\framebox(3,3){}}

  \put(0,12){\framebox(14,1){{\bf General Resources}}}
  \put(0,9){\begin{picture}(5,3)
    \put(0,0){\framebox(5,3){}}
    \put(2.5,2){\makebox(0,0){\#46}}
    \put(2.5,1){\makebox(0,0){Presentation}}
    \end{picture}}
  \put(5,9){\begin{picture}(5,3)
    \put(0,0){\framebox(5,3){}}
    \put(2.5,2){\makebox(0,0){\#44}}
    \put(2.5,1){\makebox(0,0){Configuration}}
    \end{picture}}
  \put(10,9){\framebox(4,3){}}

  \put(0,6){\begin{picture}(8,3)
    \put(0,0){\framebox(8,3){}}
    \put(4,2){\makebox(0,0){\#47}}
    \put(4,1){\makebox(0,0){Tolerances}}
    \end{picture}}
  \put(8,6){\begin{picture}(4,3)
    \put(0,0){\framebox(4,3){}}
    \put(2,2){\makebox(0,0){\#45}}
    \put(2,1){\makebox(0,0){Materials}}
    \end{picture}}

  \put(0,3){\begin{picture}(10,3)
    \put(0,0){\framebox(10,3){}}
    \put(5,2){\makebox(0,0){\#42 \& \#43}}
    \put(5,1){\makebox(0,0){Shape Representation}}
    \end{picture}}
  \put(10,3){\framebox(2,3){}}
  \put(12,3){\framebox(2,6){}}

  \put(0,0){\begin{picture}(11,3)
    \put(0,0){\framebox(11,3){}}
    \put(5.5,2){\makebox(0,0){\#41}}
    \put(5.5,1){\makebox(0,0){Description \& Support}}
    \end{picture}}
  \put(11,0){\framebox(3,3){}}

  \end{picture}}

\normalsize
\end{picture}
\setlength{\unitlength}{1pt}
%%\caption{Architecture of the STEP documentation.}
%%\label{fig:stepover} 
%%\end{figure}

\endinput



%%\caption{Architecture of the STEP documentation}
%%\label{fig:stepover}
\end{figure}
\bodsiz

\begin{remarks}
\remintro

\remtitle{STEP Architecture}

    The STEP architecture is centered around the Integrated Resource 
Models (IRs), which are defined using EXPRESS.

   An Application Protocol (AP) is a subset of the IRs. It includes an EXPRESS 
model mapped from the EXPRESS models in the IRs.

   The implementation methods, called Level 1, Level 2, and so on, are exchange
mechanisms for data that corresponds to an EXPRESS model. They essentially
consist of a mapping from EXPRESS to a data representation.

    As far as a typical end user is concerned, the IRs are invisible and there
are APs and exchange levels.

\remend
\end{remarks}

\vtitle{STEP ARCHITECTURE}
\normalsize
%\input{[pdes.pdes01.texfig]steparch}
%\input{/rdrc/design/pwilson/Figs/steparch}
%\input{/auto/home/pwilson/Rpi/Figs/steparch}
%\input{steparch}
\begin{figure}[hp]
\centering
%  pstparch.tex  fig:steparch        STEP Architecture with APs & MAPPING  
%\begin{figure}[htp]
\setlength{\unitlength}{0.2in}
%\center
\begin{picture}(32,25)
%  draw AP boxes
\thicklines
\put(5,22){\framebox(4,2){AP 1}}
\put(11,22){\framebox(4,2){AP 2}}
\put(17,22){\dashbox{0.5}(4,2){}}
\put(23,22){\framebox(4,2){AP n}}
%   draw Mapping boxes
\put(5,18){\framebox(4,2){Mapping}}
\put(11,18){\framebox(4,2){Mapping}}
\put(17,18){\dashbox{0.5}(4,2){}}
\put(23,18){\framebox(4,2){Mapping}}
%     draw vectors AP/IPIM
\put(7,20){\vector(0,1){2}}
\put(13,20){\vector(0,1){2}}
\put(19,20){\vector(0,1){2}}
\put(25,20){\vector(0,1){2}}
\put(7,18){\vector(0,-1){2}}
\put(13,18){\vector(0,-1){2}}
\put(19,18){\vector(0,-1){2}}
\put(25,18){\vector(0,-1){2}}
%          IPIM Box
\put(5,14){\framebox(22,2){{\bf Integrated Resource Models}}}
\put(5.05,14.05){\framebox(21.9,1.9){}}
\put(5.1,14.1){\framebox(21.8,1.8){}}
%          Mapping boxes
\multiput(8,10)(5,0){4}{\framebox(4,2){Mapping}}
%          IPIM/Mapping lines
\multiput(10,14)(5,0){4}{\line(0,-1){2}}
%          Implementation Levels
\put(8,3){\framebox(4,2){Level 1}}
\put(7,2){\dashbox{0.5}(10,4)[r]{Level 2\,}} 
\put(6,1){\dashbox{1.0}(16,6)[r]{Level 3\,}}
\put(5,0){\dashbox{2.0}(22,8)[r]{Level 4\,}}
%          Mapping/Level vectors
\put(10,10){\vector(0,-1){5}}
\put(15,10){\vector(0,-1){4}}
\put(20,10){\vector(0,-1){3}}
\put(25,10){\vector(0,-1){2}}

\end{picture}
%%\caption{STEP Architecture showing Multiple Appplication Protocols and 
%%           Exchange Implementations.}
%%\label{fig:steparch}
\setlength{\unitlength}{1pt}
%%\end{figure}

\endinput



%%\caption{STEP Architecture showing multiple Application Protocols
%%         and exchange implementations}
%% \label{fig:steparch}
\end{figure}
\bodsiz


\begin{remarks}
\remintro

\remtitle{Level 1 Exchange}

   Level 1 data exchange is file-based. Get your CAD system to create a
STEP data file then archive it and/or send it to someone else (to read into
their CAD system).

\remend
\end{remarks}

\vtitle{Level 1 Exchange}
\normalsize
%\input{[pdes.pdes01.texfig]figlevel1}
%\input{/rdrc/design/pwilson/Figs/figlevel1}
%\input{/auto/home/pwilson/Rpi/Figs/figlevel1}
%\input{figlevel1}
\begin{figure}[hp]
\centering
% plevel1.tex    fig:level1  STEP Level 1 Architecture
%%\begin{figure}[htp]
\setlength{\unitlength}{0.2in}
\begin{picture}(28,21)
%\input{grid02}
\thicklines

\put(0,9){\begin{picture}(4,3)
  \put(0,0){\framebox(4,3){}}
  \put(2,2){\makebox(0,0){CAX}}
  \put(2,1){\makebox(0,0){System}}
  \end{picture}}

\put(7,9){\begin{picture}(5,3)
  \put(0,0){\framebox(5,3){}}
  \put(2.5,2){\makebox(0,0){File}}
  \put(2.5,1){\makebox(0,0){Read/Write}}
  \end{picture}}

%\put(15.5,8.5){\begin{picture}(5,4)
%  \put(0,0){\framebox(5,4){}}
%  \put(0.05,0.05){\framebox(5,4){}}
%  \put(0.1,0.1){\framebox(5,4){}}
%%  \put(2.5,3){\makebox(0,0){DBMS I/F}}
%  \put(2.5,3){\makebox(0,0){Put/Get}}
%%n  \put(2.5,2){\makebox(0,0){Traversals}}
%  \put(2.5,1){\makebox(0,0){Read/Write}}
%  \end{picture}}

\put(24,9){\begin{picture}(4,3)
  \put(2,1.5){\oval(4,2)}
  \put(2,1.5){\oval(3.95,1.95)}
  \put(2,1.5){\oval(3.9,1.9)}
  \put(2,1.5){\makebox(0,0){Data File}}
  \end{picture}}


%  \put(15.5,0){\begin{picture}(5,6)
%    \put(0,0){\begin{picture}(5,1)
%      \bezier{50}(0,1)(0,0)(2.5,0)
%      \bezier{50}(2.5,0)(5,0)(5,1)
%      \end{picture}}
%    \put(0,4){\begin{picture}(5,2)
%      \bezier{50}(0,1)(0,0)(2.5,0)
%      \bezier{50}(2.5,0)(5,0)(5,1)
%      \bezier{50}(5,1)(5,2)(2.5,2)
%      \bezier{50}(2.5,2)(0,2)(0,1)
%      \end{picture}}
%    \multiput(0,1)(5,0){2}{\line(0,1){4}}
%%    \put(2.5,3){\makebox(0,0){Persistent}}
%    \put(2.5,3){\makebox(0,0){Temporary}}
%    \put(2.5,2){\makebox(0,0){Data}}
%%    \put(2.5,1){\makebox(0,0){Behaviour}}
%    \end{picture}}

%         CAX to Puts
%\put(10,10.5){\begin{picture}(6,1)
%  \put(0,0){\vector(-1,0){6}}
%  \put(0,0){\vector(1,0){5.5}}
%  \end{picture}}
\put(4,10.5){\begin{picture}(3,1)
  \put(0,0){\vector(-1,0){0}}
  \put(0,0){\vector(1,0){3}}
  \end{picture}}

%         Puts to file
\put(12,10.5){\begin{picture}(10,1)
  \put(0,0){\vector(-1,0){0}}
  \put(0,0){\vector(1,0){12}}
  \end{picture}}

%%         Puts to DB
%\put(18,7){\begin{picture}(1,2)
%  \put(0,0){\vector(0,1){1.5}}
%  \put(0,0){\vector(0,-1){2}}
%  \end{picture}}
%

\put(18,15){\begin{picture}(8,2)
  \put(0,0){\framebox(8,2){EXPRESS Schema}}
  \put(0.05,0.05){\framebox(8,2){}}
  \put(0.1,0.1){\framebox(8,2){}}
  % to File
  \multiput(7,0)(0,-1){4}{\line(0,-1){0.5}}
  % to DB
  \multiput(1,0)(0,-1){2}{\line(0,-1){0.5}}
  \multiput(1,-1.5)(-1,0){11}{\line(-1,0){0.5}}
  \multiput(-9.5,-1.5)(0,-1){2}{\line(0,-1){0.5}}
  \end{picture}}

\put(7,19){\makebox(0,0){{\bf PRIVATE}}}
\put(22,19){\makebox(0,0){{\bf STANDARD}}}
\multiput(14,7)(0,1){14}{\line(0,1){0.5}}

\end{picture}
\setlength{\unitlength}{1pt}
%%\caption{STEP Level 1 Implementation (File Exchange).}
%%\label{fig:level1}
%%\end{figure}

\endinput



%%\caption{STEP Level 1 implementation (File Exchange)}
%% \label{fig:level1}
\end{figure}
\bodsiz

\begin{remarks}
\remintro

\remtitle{Level 2 Exchange}

   Level 2 data exchange is memory-based. Get your CAD system to create a
(temporary) STEP database which you can then query and change. The data
can be written to a file for Level 1 use. At the end of the session
the STEP database is no longer available.

\remend
\end{remarks}

\vtitle{Level 2 Exchange}
\normalsize
%\input{[pdes.pdes01.texfig]figlevel2}
%\input{/rdrc/design/pwilson/Figs/figlevel2}
%\input{/auto/home/pwilson/Rpi/Figs/figlevel2}
%\input{figlevel2}
\begin{figure}[hp]
\centering
% plevel2.tex     fig:level2  STEP Level 2 Architecture
%%\begin{figure}[htp]
\setlength{\unitlength}{0.2in}
\begin{picture}(28,21)
%\input{grid02}
\thicklines

\put(0,9){\begin{picture}(4,3)
  \put(0,0){\framebox(4,3){}}
  \put(2,2){\makebox(0,0){CAX}}
  \put(2,1){\makebox(0,0){System}}
  \end{picture}}

\put(15.5,8.5){\begin{picture}(5,4)
  \put(0,0){\framebox(5,4){}}
  \put(0.05,0.05){\framebox(5,4){}}
  \put(0.1,0.1){\framebox(5,4){}}
%  \put(2.5,3){\makebox(0,0){DBMS I/F}}
  \put(2.5,3){\makebox(0,0){Put/Get}}
  \put(2.5,2){\makebox(0,0){Traversals}}
  \put(2.5,1){\makebox(0,0){Read/Write}}
  \end{picture}}

\put(24,9){\begin{picture}(4,3)
  \put(2,1.5){\oval(4,2)}
  \put(2,1.5){\oval(3.95,1.95)}
  \put(2,1.5){\oval(3.9,1.9)}
  \put(2,1.5){\makebox(0,0){Data File}}
  \end{picture}}


  \put(15.5,0){\begin{picture}(5,6)
    \put(0,0){\begin{picture}(5,1)
      \bezier{50}(0,1)(0,0)(2.5,0)
      \bezier{50}(2.5,0)(5,0)(5,1)
      \end{picture}}
    \put(0,4){\begin{picture}(5,2)
      \bezier{50}(0,1)(0,0)(2.5,0)
      \bezier{50}(2.5,0)(5,0)(5,1)
      \bezier{50}(5,1)(5,2)(2.5,2)
      \bezier{50}(2.5,2)(0,2)(0,1)
      \end{picture}}
    \multiput(0,1)(5,0){2}{\line(0,1){4}}
%    \put(2.5,3){\makebox(0,0){Persistent}}
    \put(2.5,3){\makebox(0,0){Temporary}}
    \put(2.5,2){\makebox(0,0){Data}}
%    \put(2.5,1){\makebox(0,0){Behaviour}}
    \end{picture}}

%         CAX to Puts
\put(10,10.5){\begin{picture}(6,1)
  \put(0,0){\vector(-1,0){6}}
  \put(0,0){\vector(1,0){5.5}}
  \end{picture}}

%         Puts to file
\put(22,10.5){\begin{picture}(2,1)
  \put(0,0){\vector(-1,0){1.5}}
  \put(0,0){\vector(1,0){2}}
  \end{picture}}

%         Puts to DB
\put(18,7){\begin{picture}(1,2)
  \put(0,0){\vector(0,1){1.5}}
  \put(0,0){\vector(0,-1){2}}
  \end{picture}}


\put(18,15){\begin{picture}(8,2)
  \put(0,0){\framebox(8,2){EXPRESS Schema}}
  \put(0.05,0.05){\framebox(8,2){}}
  \put(0.1,0.1){\framebox(8,2){}}
  % to File
  \multiput(7,0)(0,-1){4}{\line(0,-1){0.5}}
  % to DB
  \multiput(1,0)(0,-1){3}{\line(0,-1){0.5}}
  \end{picture}}

\put(7,19){\makebox(0,0){{\bf PRIVATE}}}
\put(22,19){\makebox(0,0){{\bf STANDARD}}}
\multiput(14,7)(0,1){14}{\line(0,1){0.5}}

\end{picture}
\setlength{\unitlength}{1pt}
%\caption{STEP Level 2 Implementation (Working Form Exchange)}
%%\label{fig:level2}
%%\end{figure}

\endinput


%%\caption{STEP Level 2 implementation (Working Form Exchange)}
%% \label{fig:level2}
\end{figure}
\bodsiz

\begin{remarks}
\remintro

\remtitle{Level 3 Exchange}

   Level 3 data exchange is database-based. The STEP data is maintained
in a (permanent) shared database. STEP level 1 files can be written and read
by the database.


\remend
\end{remarks}

\vtitle{LEVEL 3 EXCHANGE}
\normalsize
%\input{[pdes.pdes01.texfig]figlevel3}
%\input{/rdrc/design/pwilson/Figs/figlevel3}
%\input{/auto/home/pwilson/Rpi/Figs/figlevel3}
%\input{figlevel3}
\begin{figure}[hp]
\centering
% plevel3.tex     fig:level3  STEP Level 3 Architecture
%%\begin{figure}[htp]
\setlength{\unitlength}{0.2in}
\begin{picture}(28,21)
%\input{grid02}
\thicklines

\put(0,9){\begin{picture}(4,3)
  \put(0,0){\framebox(4,3){}}
  \put(2,2){\makebox(0,0){CAX}}
  \put(2,1){\makebox(0,0){System}}
  \end{picture}}

\put(15.5,8.5){\begin{picture}(5,4)
  \put(0,0){\framebox(5,4){}}
  \put(0.05,0.05){\framebox(5,4){}}
  \put(0.1,0.1){\framebox(5,4){}}
  \put(2.5,3){\makebox(0,0){DBMS I/F}}
%  \put(2.5,2){\makebox(0,0){Traversals}}
  \put(2.5,1){\makebox(0,0){Read/Write}}
  \end{picture}}

\put(24,9){\begin{picture}(4,3)
  \put(2,1.5){\oval(4,2)}
  \put(2,1.5){\oval(3.95,1.95)}
  \put(2,1.5){\oval(3.9,1.9)}
  \put(2,1.5){\makebox(0,0){Data File}}
  \end{picture}}


  \put(15.5,0){\begin{picture}(5,6)
    \put(0,0){\begin{picture}(5,1)
      \bezier{50}(0,1)(0,0)(2.5,0)
      \bezier{50}(2.5,0)(5,0)(5,1)
      \end{picture}}
    \put(0,4){\begin{picture}(5,2)
      \bezier{50}(0,1)(0,0)(2.5,0)
      \bezier{50}(2.5,0)(5,0)(5,1)
      \bezier{50}(5,1)(5,2)(2.5,2)
      \bezier{50}(2.5,2)(0,2)(0,1)
      \end{picture}}
    \multiput(0,1)(5,0){2}{\line(0,1){4}}
    \put(2.5,3){\makebox(0,0){Persistent}}
    \put(2.5,2){\makebox(0,0){Data}}
%    \put(2.5,1){\makebox(0,0){Behaviour}}
    \end{picture}}

%         CAX to Puts
\put(10,10.5){\begin{picture}(6,1)
  \put(0,0){\vector(-1,0){6}}
  \put(0,0){\vector(1,0){5.5}}
  \end{picture}}

%         Puts to file
\put(22,10.5){\begin{picture}(2,1)
  \put(0,0){\vector(-1,0){1.5}}
  \put(0,0){\vector(1,0){2}}
  \end{picture}}

%         Puts to DB
\put(18,7){\begin{picture}(1,2)
  \put(0,0){\vector(0,1){1.5}}
  \put(0,0){\vector(0,-1){2}}
  \end{picture}}


\put(18,15){\begin{picture}(8,2)
  \put(0,0){\framebox(8,2){EXPRESS Schema}}
  \put(0.05,0.05){\framebox(8,2){}}
  \put(0.1,0.1){\framebox(8,2){}}
  % to File
  \multiput(7,0)(0,-1){4}{\line(0,-1){0.5}}
  % to DB
  \multiput(1,0)(0,-1){3}{\line(0,-1){0.5}}
  \end{picture}}

\put(7,19){\makebox(0,0){{\bf PRIVATE}}}
\put(22,19){\makebox(0,0){{\bf STANDARD}}}
\multiput(14,7)(0,1){14}{\line(0,1){0.5}}

\end{picture}
\setlength{\unitlength}{1pt}
%%\caption{STEP Level 3 Implementation (Shared Database)}
%%\label{fig:level3}
%%\end{figure}

\endinput



%%\caption{STEP Level 3 implementation (Shared Database)}
%% \label{fig:level3}
\end{figure}
\bodsiz

\begin{remarks}
\remintro

\remtitle{Procedural Exchange}

    This allows not only data, but also commands (and their results)
to be passed into and out of a CAX program in a standardised manner.

    For example, instead of inserting the data representing, say, a block with
a hole in it, tell the system to create a block, put a hole in it, and then 
perhaps move it to another position. The end result in terms of data values
can be the same but the route is very different.

\remend
\end{remarks}

\vtitle{Procedural Exchange}
\normalsize
%\input{[pdes.pdes01.papers]figfilproc}
%\input{/rdrc/design/pwilson/Figs/figfilproc}
%\input{/auto/home/pwilson/Rpi/Figs/figfilproc}
%\input{figfilproc}
\begin{figure}[hp]
\centering
% pfilproc.tex      fig:filproc  STEP & Procedural I/F Architecture
%%\begin{figure}[htp]
\setlength{\unitlength}{0.2in}
\begin{picture}(28,17)
%\input{grid02}
\thicklines


\put(0,0){\begin{picture}(20,7)
  \put(0,0){\framebox(20,7){}}
  \put(8,1){\makebox(0,0){{\bf CAX PROGRAM}}}

  \put(1,2){\begin{picture}(5,3)
    \put(0,0){\framebox(5,3){}}
    \put(2.5,2){\makebox(0,0){algorithmic}}
    \put(2.5,1){\makebox(0,0){procedures}}
    \end{picture}}

  \put(9,2){\begin{picture}(4,3)
    \put(0,0){\framebox(4,3){}}
    \put(2,2){\makebox(0,0){data}}
    \put(2,1){\makebox(0,0){access}}
    \end{picture}}

  \put(16,1){\begin{picture}(3,5)
    \put(0,0){\begin{picture}(3,1)
      \bezier{50}(0,1)(0,0)(1.5,0)
      \bezier{50}(1.5,0)(3,0)(3,1)
      \end{picture}}
    \put(0,3){\begin{picture}(3,2)
      \bezier{50}(0,1)(0,0)(1.5,0)
      \bezier{50}(1.5,0)(3,0)(3,1)
      \bezier{50}(3,1)(3,2)(1.5,2)
      \bezier{50}(1.5,2)(0,2)(0,1)
      \end{picture}}
    \multiput(0,1)(3,0){2}{\line(0,1){3}}
    \put(1.5,1.5){\makebox(0,0){DB}}
    \end{picture}}

  %     proc/access link
  \put(7.5,3.5){\vector(-1,0){1.5}}
  \put(7.5,3.5){\vector(1,0){1.5}}
  %     access/DB link 
  \put(14.5,3.5){\vector(-1,0){1.5}}
  \put(14.5,3.5){\vector(1,0){1.5}}


  \end{picture}}

\put(0,9){\begin{picture}(20,8)
  \put(3,0){\framebox(7.75,2){{\bf Procedural I/F}}}
  \put(7,4){\vector(0,-1){2}}
  \put(7,4){\vector(0,1){2}}
  \put(7,7){\makebox(0,0){{\bf Procedures + Data}}}
  \put(11.25,0){\framebox(7.75,2){{\bf Data I/F}}}
  \put(15,4){\vector(0,-1){2}}
  \put(15,4){\vector(0,1){2}}
  \put(15,7){\makebox(0,0){{\bf Data}}}

  %  ais/proc link
  \put(4,-2){\vector(0,1){2}}
  \put(4,-2){\vector(0,-1){2}}
  %  ais/access link
  \put(10,-2){\vector(0,1){2}}
  \put(10,-2){\vector(0,-1){2}}
  %  step/proc link
  \put(12,-2){\vector(0,1){2}}
  \put(12,-2){\vector(0,-1){2}}

  \end{picture}}

%      horizontal partition
\put(0,10){\begin{picture}(28,3)
  \multiput(0,0)(1,0){3}{\line(1,0){0.5}}
  \multiput(19,0)(1,0){9}{\line(1,0){0.5}}

  \put(22,1){\vector(0,1){2}}
  \put(25,3){\makebox(0,0){PUBLIC}}
  \put(25,2){\makebox(0,0){and}}
  \put(25,1){\makebox(0,0){STANDARD}}

  \put(22,-1){\vector(0,-1){2}}
  \put(25,-1){\makebox(0,0){PRIVATE}}
  \put(25,-2){\makebox(0,0){and}}
  \put(25,-3){\makebox(0,0){SPECIFIC}}
  \end{picture}}

\end{picture}
\setlength{\unitlength}{1pt}
%%\caption{The relationship between procedural and data interfaces.}
%%\label{fig:filproc}
%%\end{figure}

\endinput



%%\caption{The relationship between procedural and data interfaces}
%% \label{fig:filproc}
\end{figure}
\bodsiz

\begin{remarks}
\remintro

\remtitle{Level 4 Exchange}

   This was the vision when STEP started --- intelligent knowledgebases as
an exchange mechanism.

   The vision has faded.

    The majority of STEP implementations are Level 1 (file exchange). 
Internally, though, they are implemented using a Level 2 or 3 architecture.

\remend
\end{remarks}

\vtitle{LEVEL 4 EXCHANGE}
\normalsize
%\input{[pdes.pdes01.texfig]figlevel4}
%\input{/rdrc/design/pwilson/Figs/figlevel4}
%\input{/auto/home/pwilson/Rpi/Figs/figlevel4}
%\input{figlevel4}
\begin{figure}[hp]
\centering
% plevel4.tex     fig:level4  STEP Level 4 Architecture
%%\begin{figure}[htp]
\setlength{\unitlength}{0.2in}
\begin{picture}(28,21)
%\input{grid02}
\thicklines

\put(0,9){\begin{picture}(4,3)
  \put(0,0){\framebox(4,3){}}
  \put(2,2){\makebox(0,0){CAX}}
  \put(2,1){\makebox(0,0){System}}
  \end{picture}}

\put(15.5,8.5){\begin{picture}(5,4)
  \put(0,0){\framebox(5,4){}}
  \put(0.05,0.05){\framebox(5,4){}}
  \put(0.1,0.1){\framebox(5,4){}}
  \put(2.5,3){\makebox(0,0){KBMS I/F}}
%  \put(2.5,2){\makebox(0,0){Traversals}}
  \put(2.5,1){\makebox(0,0){Read/Write}}
  \end{picture}}

\put(24,9){\begin{picture}(4,3)
  \put(2,1.5){\oval(4,2)}
  \put(2,1.5){\oval(3.95,1.95)}
  \put(2,1.5){\oval(3.9,1.9)}
  \put(2,1.5){\makebox(0,0){Data File}}
  \end{picture}}


  \put(15.5,0){\begin{picture}(5,6)
    \put(0,0){\begin{picture}(5,1)
      \bezier{50}(0,1)(0,0)(2.5,0)
      \bezier{50}(2.5,0)(5,0)(5,1)
      \end{picture}}
    \put(0,4){\begin{picture}(5,2)
      \bezier{50}(0,1)(0,0)(2.5,0)
      \bezier{50}(2.5,0)(5,0)(5,1)
      \bezier{50}(5,1)(5,2)(2.5,2)
      \bezier{50}(2.5,2)(0,2)(0,1)
      \end{picture}}
    \multiput(0,1)(5,0){2}{\line(0,1){4}}
    \put(2.5,3){\makebox(0,0){Data}}
    \put(2.5,2){\makebox(0,0){Rules}}
    \put(2.5,1){\makebox(0,0){Behaviour}}
    \end{picture}}

%         CAX to Puts
\put(10,10.5){\begin{picture}(6,1)
  \put(0,0){\vector(-1,0){6}}
  \put(0,0){\vector(1,0){5.5}}
  \end{picture}}

%         Puts to file
\put(22,10.5){\begin{picture}(2,1)
  \put(0,0){\vector(-1,0){1.5}}
  \put(0,0){\vector(1,0){2}}
  \end{picture}}

%         Puts to DB
\put(18,7){\begin{picture}(1,2)
  \put(0,0){\vector(0,1){1.5}}
  \put(0,0){\vector(0,-1){2}}
  \end{picture}}


\put(18,15){\begin{picture}(8,2)
  \put(0,0){\framebox(8,2){EXPRESS Schema}}
  \put(0.05,0.05){\framebox(8,2){}}
  \put(0.1,0.1){\framebox(8,2){}}
  % to File
  \multiput(7,0)(0,-1){4}{\line(0,-1){0.5}}
  % to DB
  \multiput(1,0)(0,-1){3}{\line(0,-1){0.5}}
  \end{picture}}

\put(7,19){\makebox(0,0){{\bf PRIVATE}}}
\put(22,19){\makebox(0,0){{\bf STANDARD}}}
\multiput(14,7)(0,1){14}{\line(0,1){0.5}}

\end{picture}
\setlength{\unitlength}{1pt}
%%\caption{STEP Level 4 Implementation (Knowledgebase Exchange).}
%%\label{fig:level4}
%%\end{figure}

\endinput



%%\caption{STEP Level 4 implementation (Shared Knowledgebase)}
%% \label{fig:level4}
\end{figure}
\bodsiz


\begin{remarks}
\remintro
\remtitle{EXPRESS Primitives}

These, plus literals, are the fundamental `things' of the EXPRESS language.

\begin{itemize}
\item Numbers, etc., are the most elementary

\item Schema, etc., are the most complex

\item Aggregations are collections of things

\item The procedural language is an imperitive programming language.

\end{itemize}

    These are later described in detail.

\remend
\end{remarks}


\vtitle{EXPRESS Primitives}

\begin{itemize}
\item Number, Integer, Real, Binary, String, Boolean (T/F), Logical (T/F/U)

\item Schema, Entity, Rule, Function, Procedure, Type (Defined, Select,
      Enumeration)

\item Aggregations --- Array, Set, List, Bag

\item Pascal-like procedural language
\end{itemize}

\begin{remarks}
\remintro
\remtitle{Simple Types}

\begin{itemize}
\item NUMBER is any kind of number with any value.

\item REAL is a decimal kind of NUMBER.

\item INTEGER is an integer kind of NUMBER and is a kind of REAL number.
\end{itemize}

    The numbers have infinite precision and can be as large or small
as you like.

The procedural language lets you perform operations on NUMBERs.

\remend
\end{remarks}

\vtitle{Simple Types}

\begin{itemize}
\item \verb|n : NUMBER| which has `subtypes'
  \begin{itemize}
  \item \verb|i : INTEGER|
  \item \verb|r : REAL|
  \end{itemize} 
\end{itemize}

These types may be given a `precision'. E.g \verb|REAL(6)|

Various operations such as $+, -, /, >=,$ etc. may be applied to these types.


\begin{remarks}
\remintro
\remtitle{Simple Types (cont)}

EXPRESS provides for both 2- and 3-valued logical statements and 
epressions.

The procedural language lets you perform operations on logicals.

\remend
\end{remarks}

\vtitle{Simple Types (cont)}

\begin{itemize}
\item \verb|l : LOGICAL| has values \verb|FALSE|, \verb|UNKNOWN|, and
\verb|TRUE|, with \\
  \verb|FALSE < UNKNOWN < TRUE|.
\item \verb|b : BOOLEAN| is a `subtype' of \verb|LOGICAL| having values of
  \verb|FALSE| and \verb|TRUE| only.
\end{itemize}

Comparisons on Booleans and Logicals can be performed (e.g $=, <, <=, <>$, etc.)

Other operations include \verb|NOT|, \verb|AND|, \verb|OR|, \verb|XOR|.


\begin{remarks}
\remintro
\remtitle{Simple Types (cont)}

A STRING is any sequence of any number of characters. A BINARY
is a specialisation of a STRING as it is limited to the digits 0 and 1.

The procedural language lets you perform operations (concatenation,
subsetting and comparison) on strings.

\remend
\end{remarks}
 

\vtitle{Simple Types (cont)}

\begin{itemize}
\item \verb|s : STRING| - a sequence of characters
\item\verb|bin : BINARY| - a sequence of bits (0s and 1s)
\end{itemize}
These may be dynamic or fixed with a maximum size. For example \\
  \verb|STRING(6) FIXED|.

These types may be concatenated and compared, and subsets addressed via 
indexing. For example
\begin{verbatim}
s1 : STRING := 's';
s2 : STRING := 'its';
.....
s1 := s1 + s2;
IF s1[2:3] = 'it' THEN ...
\end{verbatim}


\begin{remarks}
\remintro
\remtitle{Aggregations}

Aggregations are collections of things. A collection may be ordered or 
unordered, and fixed or expandible in size, and with or without duplicates.

\remend
\end{remarks}


\vtitle{Aggregations}

    General form is \verb?AGGR [L:H] OF ...? where L and H are the Low and High
bounds respectively ($H \geq L$), and containing N elements. Bags, Lists and 
Sets may have
an indefinite high bound denoted by `?' character.

\begin{description}
\item[ARRAY] Ordered collection of elements.
           $N = (H-L+1)$.
\item[BAG] Unordered collection with possibly duplicate elements.
           $L \leq N \leq H \mbox{ where } L \geq 0$.
\item[LIST] Ordered collection with possibly duplicate elements.
           $L \leq N \leq H \mbox{ where } L \geq 0$.
\item[SET] Unordered collection with no duplicate elements.
           $L \leq N \leq H \mbox{ where } L \geq 0$.
\end{description}

\emph{NOTE:} \verb?LIST [L:H] OF UNIQUE ...? is used for an ordered collection
with no duplicates.


\begin{remarks}
\remintro
\remtitle{Types}

    A TYPE is a user-defined extension to the EXPRESS-defined simple types
and aggregations. Every TYPE has a name chosen by the user.

\remend
\end{remarks}

\vtitle{Types}

    User defined extensions to the simple types and aggregations.

\begin{description}
\item[Defined:] A `renaming' of a simple type or aggregation.\\
  \verb|TYPE volume = REAL; END_TYPE;|
\item[Select:] A selection among some types. \\
  \verb|TYPE choose = SELECT(a,b,c); END_TYPE;|
\item[Enumeration:] An ordered set of values represented by names. \\
  \verb|TYPE enum = ENUMERATION OF (up, down);| \\
  \verb|END_TYPE;| 
\end{description}


\begin{remarks}
\remintro
\remtitle{TYPE Examples}

\texttt{things} illustrates an aggegration of an aggregation.

\texttt{gender} is an ENUMERATION because the possiblities (except for some
pathological cases) are known.

\texttt{hair\_type} is not a particularly good example, but it does imply 
a limited scope for the model.

\texttt{choose\_thing} is a selection between two alternatives.

\remend
\end{remarks}

\vtitle{TYPE Examples}

\begin{verbatim}
TYPE things = SET [1:?] OF 
              LIST [1:?] OF thing;
END_TYPE;

TYPE date = ARRAY [1:3] OF INTEGER;
END_TYPE;

TYPE gender = ENUMERATION OF 
              (male, female);
END_TYPE;

TYPE hair_type = ENUMERATION OF 
                 (blonde, black, bald);
END_TYPE;
 
TYPE choose_thing = SELECT 
                    (thing1, thing2);
END_TYPE;
\end{verbatim}


\begin{remarks}
\remintro
\remtitle{ENTITY}

An ENTITY is a user defined object, representing some thing. It has
various components which will be described. Every ENTITY has a user-defined
name.

\remend
\end{remarks}

\vtitle{ENTITY}

    An entity represents an object of interest in the model of the Universe
of Discourse.

    The characteristics (properties) of an entity are defined in terms of data
(attributes) and behaviour (constraints).

    An entity may `inherit' properties from another entity.


\begin{remarks}
\remintro
\remtitle{ENTITY Attributes}

An attribute is some kind of data element that helps characterize the ENTITY.
An attribute consists of a user-defined name and a specification of the
kind of data. 

The kind of data may be a (collection of) simple types, TYPEs or ENTITYs.

\remend
\end{remarks}

\vtitle{ENTITY Attributes}

    Attributes are either \emph{explicit} or \emph{derived}.
\begin{verbatim}
ENTITY circle;
  center : point;
  radius : length;
DERIVE
  perimeter : length := 2.0*PI*radius;
END_ENTITY;

TYPE length = REAL; END_TYPE;
\end{verbatim}
The data for calculating a derived attribute must be accessible from 
the entity.


\begin{remarks}
\remintro
\remtitle{ENTITY Constraints}

Constraints limit the kind and/or values of the attributes' data.

\textbf{UNIQUE} 
In this case no two circles can have the same center AND radius.

\textbf{WHERE} rules are logical expressions. In this case
the radius must be positive length.

\remend
\end{remarks}

\vtitle{ENTITY Constraints}

    Attribute values within entity instances may be constrained by either
uniqueness requirements or by domain rules (WHERE clauses). These apply to
\emph{every} instance of the entity.

\begin{verbatim}
ENTITY circle;
  center : point;
  radius : length;
UNIQUE
  un1 : center, radius;
WHERE
  pos_rad : radius > 0.0;
END_ENTITY;
\end{verbatim}

A WHERE (domain) rule fails if it evaluates to \verb|FALSE|.


\begin{remarks}
\remintro
\remtitle{Example ENTITY}

   The attributes are those things of interest about a person. 

Not everyone has a nickname.

Not everyone has a spouse.

No two people have the same social security number.

The WHERE rule states that if someone has a spouse then the spouse must
be of the opposite sex.

\remend
\end{remarks}

\vtitle{Example ENTITY}

\begin{verbatim}
ENTITY person;
  first_name : STRING;
  last_name  : STRING;
  nickname   : OPTIONAL STRING;
  ss_no      : INTEGER;
  sex        : gender;
  spouse     : OPTIONAL person;
  children   : SET [0:?] OF person;
UNIQUE
  un1 : ss_no;
WHERE
  w1 : (EXISTS(spouse) AND sex <> spouse.sex)
       OR NOT EXISTS(spouse);
END_ENTITY;
\end{verbatim}


\begin{remarks}
\remintro
\remtitle{Subtyping}

A Subtype is a special kind of its supertype(s). 


Forgetting about Cantor and degrees of infinity
\begin{itemize}
\item    There are fewer odd numbers than there are natural numbers.

\item    There are fewer prime numbers than there are natural numbers.
\end{itemize}

\remend
\end{remarks}

\vtitle{Subtyping}

    Subtypes inherit ther properties of their Supertypes.

\begin{verbatim}
ENTITY natural_number;
  value : INTEGER;
END_ENTITY;

ENTITY odd_number
  SUBTYPE OF (natural_number);
  ...
END_ENTITY;

ENTITY prime_number
  SUBTYPE OF (natural_number);
  ...
END_ENTITY;
\end{verbatim}


\begin{remarks}
\remintro
\remtitle{FUNCTION Example}

    These are part of EXPRESS programming language aspects.

    The particular example takes two aggregations and returns either
TRUE or FALSE depending on whether or not the first is a subset of 
the second (i.e., every member of \texttt{sub} is also in \texttt{super}).

\remend
\end{remarks}

\vtoptitle{FUNCTION Example}

    Used for constraint definition and for derived attributes.
\begin{verbatim}
FUNCTION subset(sub,super : 
         AGGREGATE OF GENERIC) : BOOLEAN;

  IF (SIZEOF(sub) > SIZEOF(super)) THEN
    RETURN(FALSE);
  END_IF;
  REPEAT i := 1 TO SIZEOF(sub);
    IF (sub[i] IN super) THEN
      super := super - sub[i];
    ELSE
      RETURN(FALSE);
    END_IF;
  END_REPEAT;
  RETURN(TRUE);

END_FUNCTION;
\end{verbatim}


\begin{remarks}
\remintro
\remtitle{Predefined Functions}

EXPRESS includes a variety of predefined functions. 

There is more on these later in the course.
\remend
\end{remarks}

\vtitle{Predefined Functions}

\begin{itemize}
\item Mathematical (e.g ABS, SIN, SQRT etc)
\item Aggregation sizes (e.g LOBOUND, HIBOUND, SIZEOF, LENGTH)
\item Number/String conversion (FORMAT, VALUE)
\item EXISTS(V) checks for existance of OPTIONAL attribute V.
\item NVL(ATTR; SUBS) if ATTR has a value, then ATTR is returned,
      else SUBS is returned.
\item TYPEOF(V) returns the set of types of V.
\item USEDIN(T; R) takes an entity T and its role R that it plays in other
      entities and returns each entity instance that uses T in role R.
\end{itemize}


\begin{remarks}
\remintro
\remtitle{Constants}

EXPRESS includes the mathematical constants $\Pi$ and $e$ (to infinite
precision).

You can also define your own constants, but this is not often done.

\remend
\end{remarks}

\vtitle{Constants}

\begin{itemize}
\item Some predefined constants (PI, e).
\item User-defined constants
\begin{verbatim}
CONSTANT
  thousand : NUMBER := 1000;
  million  : NUMBER := thousand**2;
  origin   : point := point(0.0, 0.0);
END_CONSTANT;
\end{verbatim}
\end{itemize}


\begin{remarks}
\remintro
\remtitle{SCHEMA}

The minimum EXPRESS model consists of a single empty SCHEMA.

TYPE, ENTITY, FUNCTION definitions are contained within a SCHEMA.

\remend
\end{remarks}

\vtitle{SCHEMA}

\begin{itemize}
\item A SCHEMA contains the objects, relationships and constraints for a
particular domain of interest.
\item Schemas provide a mechanism for partitioning the `real world' into
      relevant domains. 
\item There must be well defined limits to the domain represented via a Schema
      --- a single Schema should not be used to describe two different
      domains of interest.
\end{itemize}


\begin{remarks}
\remintro
\remtitle{SCHEMA (cont)}

A model usually consists of more than one SCHEMA.

From within a SCHEMA you can get at anything in any other SCHEMA
(there is no way to `hide' something).

\remend
\end{remarks}

\vtitle{SCHEMA (cont)}
\begin{itemize}
\item An EXPRESS model may contain more than one Schema.
\item Where multiple Schemas are used there is normally one `main' schema
and n `subsidiary' schemas.
\end{itemize}
\begin{verbatim}
SCHEMA main;
  REFERENCE FROM sub1 ...
  -- types, entities, rules, etc.
END_SCHEMA;

SCHEMA sub1;
  -- types, entities, rules, etc.
END_SCHEMA;
\end{verbatim}


\endinput
%%%%%%%%%%%%%%%%%%%%%%%%%%%%%%%%%%%%%%%%%%%%%%


\vtitle{EXPRESS SUMMARY}

\begin{itemize}
\item A powerful OO information modeling language
  \begin{itemize}
  \item Primary form is a computer processible text language.
  \item EXPRESS-G as a graphical subset.
  \item EXPRESS-I as an instantiation form (in review).
  \item Extension to methods is planned.
  \end{itemize}
\item In ISO standardization process.
\item Normative STEP information models.
\item Becoming widely used in the modeling communities.
\item Software tools appearing.
\item User Group forming
\end{itemize}

\endinput
