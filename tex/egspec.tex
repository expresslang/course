% egspec.tex    % ISO TR 9007 model statement
%               % based on Ebook/mmodel.tex
% Created by Peter R Wilson, 1992 -- 2004

%\documentclass[article]{memoir}
\documentclass{article}
\usepackage{ifpdf}

%\addtolength{\textheight}{4\onelineskip}
%\maxsecnumdepth{part}

\title{Example model statement \\
       Car Registration Authority}
\author{Peter Wilson}
\date{}

\raggedbottom
\begin{document}

\maketitle
\tableofcontents
\clearpage


%\chapter{Introduction}
\section{Introduction}


   The model statement, which is given in sections~\ref{scope}
and~\ref{description}, has been taken from ISO TR9007 \textit{Information
processing systems --- Concepts and terminology for the conceptual
schema and the information base}, 1987. This contains several renditions
of the model using different kinds or representations including
Entity-Relationship, A NIAM-like approach, and a formulation based on
predicate logic.


%\chapter{Scope} \label{scope}
\section{Scope} \label{scope}

The scope of the model to be described has to do with the registration of cars
and is limited to the scope of interest of the Registration Authority. The
Registration Authority exists for the purpose of:

\begin{itemize}

\item Knowing who is or was the registered owner of a car at any time from
construction to destruction of the car.

\item To monitor certain laws, for example regarding fuel consumption of cars
and their transfer of ownership.

\end{itemize}

%\chapter{Description} \label{description}
\section{Description} \label{description}

\subsection{Manufacturers of cars}

There are a number of manufacturers, each with one unique name. Manufacturers
may start operation, with the permission of the Registration Authority (which
permission cannot be withdrawn). No more than five manufacturers may be in
operation at any time. A manufacturer may cease to operate provided he owns no
cars, in which case permission to operate lapses.

\subsection{Cars}

A car is of a particular model and is given a serial number by its
manufacturer that is unique among the cars made by that manufacturer. The
manufacturer is registered as owner of the car as soon as practicable. At this
time it is given one registration number, unique for all cars and for all
time. The year of production is also recorded. During the month of January
only, a car can be declared to have been produced in the previous year.
Eventually a car is destroyed and the date of destruction is registered. The
history of the car must be kept until the end of the second calendar year
following its destruction.

\subsection{Car models}

A model of car has one universally unique name. Cars of each model are made by
only one manufacturer. New models may be introduced without limit. All cars of
one model are recorded as having the same fuel consumption.

\subsection{Fuel consumption}

Fuel consumption is the number of liters of hydrocarbon fuel per 100
kilometers, which will be between 4 and 25 liters. The fuel consumption
averaged over all registered cars produced by a particular manufacturer in a
particular year is required not to exceed a maximum value, which is the same
for each manufacturer and may change from year to year. At the end of each
January an appropriate message is sent by the Registration Authority to each
manufacturer which has failed to meet this requirement.

\subsection{Garages}

There are a number of garages\footnote{Dealerships in American English.}, each
one with a unique name. New garages may start trading. Garages may own cars,
but at any time the cars they own must have originated from no more than three
manufacturers (which three is unimportant, and may vary with time). A garage
cannot cease to trade as long as it owns cars.

\subsection{Persons}

There are a number of persons who can own one or more cars. Each person has
one unique name. Only those persons are of interest who own, or have at some
time owned, a car still known to the Registration Authority.

\subsection{Car ownership}

At any one time a car may be owned by either its manufacturer, or a trading
garage, or a person or a group of persons. If a car is owned by a group of
persons, each is regarded as an owner.

\subsection{Transfer of ownership}

Ownership of a car is transferred by registration of the actual transfer,
including the date. A manufacturer can only transfer to garages, and cannot be
a transferee. A garage can transfer only to people. After destruction of the
car it cannot be transferred anymore. Earlier transfer though still can be
recorded.

\end{document}

\clearpage
\chapter{Further exercises}

\begin{enumerate}
\item Do the following:
  \begin{enumerate}
  \item Write an information model that describes the logical content of
        a report. Assume that a report consists of a title and one or
        more authors, together with the publication date. It may have
        an abstract and may have a table of
        contents. The body of the report consists of at least two sections.
        Further divisions of the report are subsections and sub-subsections.
        Figures and tables may also be included within any sub-subsection, or
        higher level partitions. The report may have a bibliography.
  \item Write an information model that describes a book. A book is similar
        to a report with the following exceptions. A book may consist of
        two or more parts, each of which must contain two or more chapters.
        Each chapter contains at least two sections. There is always a table
        of contents and there is never an abstract, although it may have
        a preface which serves the same purpose. A book may have an index.
  \item Does the above description apply to all books?
  \item Create an information model that supports both reports and books.
        Include anything extra that you feel is necessary that is missing
        from the above descriptions.
  \end{enumerate}

\item Write an information model corresponding to the following description.

    An international company has a number of ongoing development projects.
A project has a unique name and is located in a specific city. There are
a number of suppliers to the company. The suppliers have names and may have
several branches, each in a different city. Suppliers with identical names
do not have branches in the same city. A supplier may supply one or more
different kinds of parts to the company. A part is identified by a catalogue
number, and also has a short description. Projects purchase parts from
the nearest location which stocks the part. The company keeps a record
of the purchase orders (i.e., part, supplier, and quantity) of each project.

\item Write an information model corresponding to the following description.

    A University is organised into academic, research and administrative
departments. Administrative staff may work in any kind of department, but
neither academic staff nor research staff work in the administrative
departments. Academic staff teach courses and may do research work.
Research staff are limited to research work only. Administrative staff
neither teach nor do research. All undergraduate and some graduate
students attend courses. There is a fee for each course, the amount of
which differs according to the course. Students are graded on each course
they attend, with a grade having a value between 0 and 100. It is a
tradition, however, of the University that no student has ever been
graded at either 0 or 100. Some undergraduate students may be employed
part-time to assist the administrative staff, but only if their grade
is 75 or more. All staff get paid a salary, the amount of which depends
on their position. Graduate students do research. They may teach not more than
two courses, and are paid at a fixed rate per course. No person under the age
of 18 may be paid, and the retirement age is 65.

\item  Write an information model about the delivery of items
according to the following description.

    The currency of Fluidistan is the G. This is divided into the
smaller p and z units, where G1 = 8p and 1p = 16z. The amount 190z,
for example, is written as G1-7-12. Linear measures in Fluidistan are
the inch and foot, where 1 foot equals 12 inches. The weight measures
are the pound and ounce, where 1 pound is 16 ounces.

    The government of Fluidistan operate a mail delivery service for
certain kinds of item. There are also private delivery services which
will accept any kind of item. The following are the regulations governing
the Fluidistan mail service.
\begin{description}
\item[Post Cards:] The card rate is 1p~3z. To qualify for the card rate
a postcard must be of a uniform thickness and no thinner than 0.007 inches.
It must be no larger than 4.25 by 6 inches and no smaller than 3.5 by 5 inches.

\item[Letters:] The letter rate is 1p~13z for letters weighing one ounce
or less. The rate increases by 1p~6z for each additional ounce or part
thereof. An item weighing more than 11 ounces cannot be sent at the
letter rate. Letters less than one ounce are non-standard if the length
is greater than 11.5 inches or the height is greater than 6.125 inches
or the thickness is greater than 0.25 inches or the length to height
ratio is not between 1.3 and 2.5 inclusive. Non-standard letters are
subject to a surcharge of 10z in addition to the standard rate.

\item[Parcels:] The parcel rate is G2-4-2 for items not exceeding two
pounds in weight. The rate increases by by 2p~11z for each additional
pound or part thereof, provided the weight is not greater than ten pounds.
Above ten pounds the rate increases by 2p~8z for each additional pound
or part thereof. Note: Parcels weighing less than fifteen pounds and
whose length plus girth exceeds seven feet are chargeable at the
fifteen pound rate.

\item[Size Standards:] Items whose thickness is less than 0.007 inches
are not accepted for delivery. Items less than 0.25 inches in thickness
must be rectangular in shape and at least 3.5 inches high and at
least 5 inches long. Items weighing more than 70 pounds are not accepted
for delivery.

\item[Environmental:] Neither hazardous materials nor live or dead animals
 will be accepted for delivery. All items, except cards, must be enclosed
in some wrapping. Items enclosed in environmentally sound and recycleable
wrapping are entitled to a discount of ten percent of the applicable rate;
if this results in a fraction of a z, the rate is rounded up to the
nearest z.

\end{description}

\end{enumerate}

\end{document}