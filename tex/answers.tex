% answers.tex    Some answers to Modeling exercises (exercises.tex)
% Created by Peter R Wilson, 1992 -- 2004

\documentclass{article}

\usepackage{comment}

\usepackage{egs}
\usepackage{xkw}

\newenvironment{exercises}{\begin{enumerate}}{\end{enumerate}}
\newenvironment{code}{}{}

\title{Modeling Exercises\\Some Answers}
\author{Peter Wilson}
\date{}

\setcounter{secnumdepth}{1}
\begin{document}

\maketitle

\tableofcontents

\clearpage


\section{Ambiguities}

    This set of exercises were designed to illustrate some of the
difficulties in using natural language as a means of precise communication.
Most of the exercises were open-ended, so only an illustrative set of possible
answers are provided.

\subsection{Exercise 1}

\begin{itshape}
 Write three sentences that are ambiguous along the lines of \Q{The chicken
      is ready to eat}. Can you think of more?
\end{itshape}

    Here are some which exhibit various kinds of ambiguity. Some of the 
ambiguities arise from poor word order in the sentences while others come from
the use of homonyms.

\begin{itemize}
\item Arnold's Body Works. (Sign on automotive repair shop)
\item \Q{Hmph!} snorted Major Featherstonehaugh, and grasped his monocle which was
      lying on his spare chest.
\item I can recommend this candidate for the position for which he has applied 
      with complete confidence.
\item I was thrown from my car as it left the road. I was later found in a
      ditch by some stray cows.
\item If the baby does not thrive on raw milk, boil it.
\item It was here that the Emperor liked to put on his grand spectacles.
\item Joe saw the man with the telescope.
\item Much discussion was going on about him.
\item Summarising, the study showed that males of the sixties generation kept
      their hair longer than those who had grown up in other decades.
\item The farmer talked about the stocking of pig pens with his neighbours.
\item There will be a meeting on bicycles in Conference Room 3.
\item Wanted: A rabbit for a child with floppy ears.
\end{itemize}

\subsection{Exercise 2}

\begin{itshape}
 How many ways can a date be written? Are they all in use?
      What other sorts of calendrical sytsems are there in addition to 
      Anno Domine style dates?
\end{itshape}

    The most common dating systems use a combination of day, month, year. There
is a total of six combinations among the ordering of these quantities. However,
we are not aware of any dating systems in use that put the year between the
month and the day, thus leaving four possible orderings. In the USA the ordering
is month, day, year; in the UK it is day, month, year; ISO specifies that the
ordering shall be year, month, day.

   In Anno Domine style dates years are counted from the birth of Christ; the
calender is called the Gregorian calender after Pope Gregory XIII who, in 1582,
 reformed the
previous Julian calander to take better account of the intercalated days
(leap days) and thus ensure that the start of the year remained at the same
point in the seasons of the year.

   Other systems have different starting points for the zero'th year.

   The Muslim calendar is basically a lunar calendar with months of either
29 or 30 days. Years are counted from the first day of the month preceeding
the flight of Mohammed from Mecca to Medina. In the Gregorian calendar the
starting date corresponds to Thursday 19 July, 622 {\sc ad}.

    The Jewish calendar is based on solar years and lunar months. In order to
keep these synchronised some years have an intercalary 13th month of 30 days.
Years are counted since the Creation and written as 1993 {\sc am} {\em (anno
mundi)}. In the Gregorian calendar, Creation occured on the night between Sunday
and Monday 7 October 3761 {\sc bc} at 11 hours and eleven and one third
minutes {\sc pm}.

    Most countries use the Gregorian calendar for secular purposes, but some 
countries also have a seperate religious calendar. India is one example, where the
Hindu calendar is used for the timing of religious festivals.

   Another form of calendar is a fiscal calendar. It often happens that the start
of a fiscal year does not coincide with the usual calendar year. Examples are
the income tax year in the UK which starts on 5 April, so the 1993 tax year
extends from 5 April 1993 to 4 April 1994; another is the USA Department of 
Defence budget year which starts on 1 October. In some cases, fiscal years
start on the nearest Monday to a given date, and the months are divided up into
4 and 5 week periods, so that each fiscal month starts on the same day of the
week throughout the year. In this case, the months typically are numbered from 
the start of the year rather than being named.

   Some dating systems, particularly for industrial planning and scheduling
systems, just use day and year, where the days are counted from the start of
the year. Thus, the third of February 1993 would be written as 34/1993.

\subsection{Exercise 3}

\begin{itshape}
 Ouida is said to have said \Q{All rowed fast but none so fast as stroke}.
      The word \Q{stroke} has many other meanings than in this quotation. 
      How many ways can you interpret the word \Q{stroke}? 
      Write sentences which provide examples for each meaning.
\end{itshape}

    Here is a non-exhaustive listing:
\begin{itemize}
\item He stroked her hair.
\item The blood clot eventually reached his brain and the resulting stroke 
      left his left side paralysed.
\item The advance started at the stroke of midnight.
\item Casey struck out.
\item Frieda won the backstroke competition.
\item The lightning stroke felled the tree.
\item With a stroke of his pen the immigration officer dashed all of 
      their hopes.
\item His plan was a master stroke.
\end{itemize}

\subsection{Exercise 4}

\begin{itshape}
 How many descriptions can you think of for a timepiece? List them.
\end{itshape}

    The following is a reasonably comprehensive list.
\begin{itemize}
\item One or two pointers, mounted on a common axis, rotating above a marked
      circle. (An analog clock or watch)
\item A device that measures the periods of radiation corresponding to the
      transition between the two hyperfine levels of the ground state
      of the cesium 133 atom. (An atomic clock)
\item A marked cylinder of wax with a burning central wick. (A candle)
\item A digital display driven by a microchip. (A digital clock or watch)
\item Two chambers, one above the other, connected by a thin passage, with 
      sand flowing between them. (An hourglass)
\item A rod at 45 degrees to the vertical surrounded by a graduated sector.
      (A sundial)
\item A graduated container with water dripping out of it. (A water clock)
\end{itemize}

\subsection{Exercise 5}

\begin{itshape}
Write a more general definition for \Q{cloverleaf} that makes no assumptions
      about which side of the road people drive on.
\end{itshape}

    A traffic arrangement in which one highway passes over another and the
roads connecting the two are in the pattern of a four-leaved clover so that
traffic moving from one highway to the other can merge with the traffic
flow without crossing in front of oncoming vehicles.


\clearpage


\section{Simple models}

    This set of exercises is intended to help the reader think about
categorization and classification, and to provide some experience in 
creating some small information models.

\subsection{Exercise 1}

\begin{itshape}
Develop a categorization system for non-fiction books. (Hint --- think
      how they are organised in a library).
\end{itshape}

    There are several systems in use for book classification. Two well known
ones are the Library of Congress and the Dewey Decimal Systems.

\subsubsection{Library of Congress system}

    The Library of Congress scheme partitions books into 20 classes, each class
being designated by a letter. Subclasses are designated by letter combinations
and topics by a numerical code. The following is a list of the top-level classes
and the number of subclasses within each of these is given in parentheses.
\begin{quote}
\begin{description}
\item[A:] General works (10)
\item[B:] Philosophy, Psychology, Religion (14)
\item[C:] Auxiliary Sciences of History (9)
\item[D:] History: General and Old World (19)
\item[E--F:] History: Western Hemisphere
\item[G:] Geography, Anthropology, Recreation (8)
\item[H:] Social Sciences (15)
\item[J:] Political Science (10)
\item[K:] Law (8)
\item[L:] Education (10)
\item[M:] Music (2)
\item[N:] Fine Arts (7)
\item[P:] Language and Literature (18)
\item[Q:] Science (11)
\item[R:] Medecine (16)
\item[S:] Agriculture (5)
\item[T:] Technology (16)
\item[U:] Military Science (8)
\item[V:] Naval Science (9)
\item[Z:] Bibliography, Library Science
\end{description}
\end{quote}
    The classification structure extends several levels deep. 
As an example of a 
subclass, here is the listing for General Works (Class~A).
\begin{quote}
\begin{description}
\item[AC:] Collections
\item[AE:] Encyclopedias
\item[AG:] Dictionaries
\item[AI;] Indexes
\item[AM:] Museums
\item[AN:] Newspapers
\item[AP:] Periodicals
\item[AS:] Acadamies and Societies
\item[AY:] Yearbooks, Almanacs, Directories
\item[AZ:] History of scholarship
\end{description}
\end{quote}

\subsubsection{The Dewey system}

    In 1873 Melvil Dewey (1851--1931) proposed the classification system that
bears his name. At the top level there are ten classes, numbered from 
000 to 900, in steps of one hundred. Sub-classes are numbered in steps 
of ten, sub-subclasses in steps of one, and so on. The top level of 
this structure is:
\begin{quote}
\begin{description}
\item[000:] Generalities
\item[100:] Philosophy and psychology
\item[200:] Religion
\item[300:] Social sciences
\item[400:] Language
\item[500:] Natural sciences and mathematics
\item[600:] Technology (Applied sciences)
\item[700:] The Arts
\item[800:] Literature and rhetoric
\item[900:] Geography and history
\end{description}
\end{quote}
   The system supports several levels of refinement --- a complete listing 
requiring a a book of itself. In order to provide a comparison with the 
Library of Congress system the first-level divisions under 
Generalities (Class 000) are:
\begin{quote}
\begin{description}
\item[010:] Bibliography
\item[020:] Library and information sciences
\item[030:] General encyclopedic works
\item[040:] (nothing in this sub-class)
\item[050:] General series and their indexes
\item[060:] General organisations and museology
\item[070:] News media, journalism, publishing
\item[080:] General collections
\item[090:] Manuscripts and rare books
\end{description}
\end{quote}

\subsection{Exercise 2}

\begin{itshape}
Develop a categorization scheme for the goods sold in your local
      grocery store.
\end{itshape}

    Here is one classification structure, where we use an indented listing to
show the categories and sub-categories:
\begin{itemize}
\item FOOD
  \begin{itemize}
  \item Fresh
  \item Frozen
  \item Canned and bottled
  \item Dried
  \end{itemize}
\item DRY GOODS
  \begin{itemize}
  \item Cleaning materials
  \item Paper goods
  \item Utensils
  \end{itemize}
\item BEVERAGES
  \begin{itemize}
  \item Soda
  \item Juice
  \item Dairy
  \item Alcoholic
  \end{itemize}
\end{itemize}

    Of course, there are many other ways in which we could classify these. For
example, the food category has been classified according to the storage
process. Here is another way to classify food --- by the kind of consumer:
\begin{itemize}
\item FOOD
  \begin{itemize}
  \item Baby
  \item Adult
  \item Pet
  \item Gourmet
  \end{itemize}
\end{itemize}

    Yet another way is by the kind of food itself:
\begin{itemize}
\item FOOD
  \begin{itemize}
  \item Meat
  \item Fish
  \item Vegetable
  \item Fruit
  \item Pasta
  \item Baked
  \item Dairy 
  \item etc.
  \end{itemize}
\end{itemize}

    From the food example, it rapidly becomes obvious that there are many 
ways in which things can be classified, and each of the ways is 
appropriate according to the particular view of the classifier. 
It also turns out that often we need multiple classifications. 
For example, a certain baby food may be tinned, pureed
fruit, which cuts across the classifications given above.

\subsection{Exercise 3}

\begin{itshape}
A book is written by one or more authors and is printed by a single
publisher. A book is owned by a person. Sketch a model 
that captures these statements.
\end{itshape}


    The following is one possible \Express\ model.
\begin{code}
\begin{verbatim}
*)
SCHEMA exercise_2_3;

ENTITY book;
  author       : SET [1:?] OF person;
  published_by : publisher;
END_ENTITY;

ENTITY ownership;
  owner : person;
  item  : book;
END_ENTITY;

ENTITY person;
  name : STRING;
END_ENTITY;

ENTITY publisher;
  name : STRING;
END_ENTITY;

END_SCHEMA;
(*
\end{verbatim}
\end{code}

\subsection{Exercise 4}

\begin{itshape}
Sketch a model of a bicycle. Assume that a bicycle consists of a 
frame, a saddle, handlebars, pedals, and two wheels.
\end{itshape}

    In the following models, the decompostion has only been taken to the first
level. That is, the major element in the model, namely the bicycle, has
been completely described, but the components of the bicycle have merely been
noted and not elaborated. The model is very simple as the bicycle just consists
of the noted components.

\begin{code}
\begin{verbatim}
*)
SCHEMA exercise_2_4;

ENTITY bicycle;
  body         : frame;
  seat         : saddle;
  steered_by   : handlebar;
  driven_by    : SET [2:2] OF pedal;
  supported_by : SET [2:2] OF wheel;
END_ENTITY;

ENTITY frame; (* attributes *) END_ENTITY;
ENTITY saddle; (* attributes *) END_ENTITY;
ENTITY handlebar; (* attributes *) END_ENTITY;
ENTITY pedal; (* attributes *) END_ENTITY;
ENTITY wheel; (* attributes *) END_ENTITY;

END_SCHEMA;
(*
\end{verbatim}
\end{code}


\subsection{Exercise 5}

\begin{itshape}
How does your model change if you include a chain connecting the pedals
      to the rear wheel, and also if you consider that a wheel has a hub, 
      spokes, a rim and a tire?
\end{itshape}

    Having more knowledge about the components of the bicycle leads to 
a richer model. There are several ways in which the information could 
be represented, and we have made a fairly arbitrary choice. It is 
reasonably obvious that a wheel consists of several components, which 
enables us to elaborate on the definition of this entity. We also 
chose to indicate that the pedals, chain and one wheel (usually the 
rear one) together performed the drive train function for the
bicycle. Similarly, the handlebars and the other (front) wheel 
enabled the bicycle to be steered.


\begin{code}
\begin{verbatim}
*)
SCHEMA exercise_2_5;

ENTITY bicycle;
  body       : frame;
  seat       : saddle;
  steered_by : steering_assembly;
  driven_by  : driving_assembly;
END_ENTITY;

ENTITY frame; (* attributes *) END_ENTITY;
ENTITY saddle; (* attributes *) END_ENTITY;

ENTITY steering_assembly;
  control : handlebar;
  support : wheel;
END_ENTITY;

ENTITY handlebar; (* attributes *) END_ENTITY;

ENTITY wheel;
  center     : hub;
  outer      : rim;
  support    : tire;
  connectors : SET [4:?] OF spoke;
END_ENTITY;

ENTITY hub; (* attributes *) END_ENTITY;
ENTITY rim; (* attributes *) END_ENTITY;
ENTITY tire; (* attributes *) END_ENTITY;
ENTITY spoke; (* attributes *) END_ENTITY;

ENTITY driving_assembly;
  driver     : SET [2:2] OF pedal;
  driven     : wheel;
  connection : chain;
END_ENTITY;

ENTITY pedal; (* attributes *) END_ENTITY;
ENTITY chain; (* attributes *) END_ENTITY;

END_SCHEMA;
(*
\end{verbatim}
\end{code}

%%%%%%%%%%%%%%%%%%%%%%%%%%%%%%%%%%%%%%%%%%%%%
\begin{comment}

\subsection{Exercise 2.6}

\begin{itshape}
Use any two other languages to represent the book and bicycle models.
\end{itshape}

    We present in the following sections representations in various languages 
of the book and simple bicycle models given earlier in \Express and \ExpressG\
forms.

\subsubsection{DAPLEX models}

    The following is a DAPLEX representation of the book model.
\begin{code}
\begin{verbatim}
DECLARE Book() ==>> ENTITY
DECLARE Author(Book) ==>> Person
DECLARE PublishedBy(Book) ==> Publisher

DECLARE Person() ==>> ENTITY
DECLARE Name(Person) ==> STRING

DECLARE Publisher ==>> ENTITY
DECLARE Name(Publisher) ==> STRING

DECLARE Ownership() ==>> ENTITY
DECLARE Owner(Ownership) ==> Person
DECLARE Item(Ownership) ==> Book
\end{verbatim}
\end{code}

    This is the DAPLEX representation for the simple bicycle model.
\begin{code}
\begin{verbatim}
DECLARE Bicycle() ==>> ENTITY
DECLARE Body(Bicycle) ==> Frame
DECLARE Seat(Bicycle) ==> Saddle
DECLARE SteeredBy(Bicycle) ==> Handlebar
DECLARE DrivenByLeft(Bicycle) ==> Pedal
DECLARE DrivenByRight(Bicycle) ==> Pedal
DECLARE FrontWheel(Bicycle) ==> Wheel
DECLARE RearWheel(Bicycle) ==> Wheel

DECLARE Frame() ==>> ENTITY
DECLARE Saddle() ==>> ENTITY
DECLARE Handlebar ==>> ENTITY
DECLARE Pedal() ==>> ENTITY
DECLARE Wheel() ==>> ENTITY
\end{verbatim}
\end{code}

\subsubsection{GEM models}

    First we show a GEM representation for the book model.

\begin{code}
\begin{verbatim}
BOOK(Author: {PERSON}, PublishedBy: PUBLISHER);
PERSON(Name: {c});
PUBLISHER(Name: {c});
OWNERSHIP(Owner: PERSON, Item: BOOK);
\end{verbatim}
\end{code}

    And now here is the representation for the simple bicycle model.

\begin{code}
\begin{verbatim}
BICYCLE(Body: FRAME, Seat: SADDLE, SteeredBy: HANDLEBAR,
        DrivenByLeft: PEDAL, DrivenByRight; PEDAL,
        FrontWheel: WHEEL, RearWheel: WHEEL);
FRAME(SerialNo: {c});
SADDLE(Id: {c});
HANDLEBAR(Id: {c});
PEDAL(Id: {c});
WHEEL(Id: {c});
\end{verbatim}
\end{code}

\subsubsection{SQL models}

    An SQL representation for the book model is:

\begin{code}
\begin{verbatim}
CREATE TABLE BOOK
  ( PUBLISHED_BY CHAR(50) NOT NULL,
    ISBN         CHAR(20) NOT NULL,
    PRIMARY KEY (ISBN),
    FOREIGN KEY (PUBLISHED_BY) REFERENCES PUBLISHER (NAME) )

CREATE TABLE PUBLISHER
  ( NAME         CHAR(50) NOT NULL,
    PRIMARY KEY (NAME) )

CREATE TABLE PERSON
  ( NAME         CHAR(50) NOT NULL,
    PRIMARY KEY (NAME) )

CREATE TABLE AUTHORSHIP
  ( BOOK         CHAR(20) NOT NULL,
    AUTHOR       CHAR(50) NOT NULL,
    PRIMARY KEY (BOOK, AUTHOR),
    FOREIGN KEY (BOOK) REFERENCES BOOK (ISBN),
    FOREIGN KEY (AUTHOR) REFERENCES PERSON (NAME) )

CREATE TABLE OWNERSHIP
  ( OWNER        CHAR(50) NOT NULL,
    ITEM         CHAR(20) NOT NULL,
    PRIMARY KEY (OWNER, ITEM),
    FOREIGN KEY (OWNER) REFERENCES PERSON (NAME),
    FOREIGN KEY (ITEM) REFERENCES BOOK (ISBN) )
\end{verbatim}
\end{code}

    And here is an  SQL representation for the simple bicycle model.
\begin{code}
\begin{verbatim}
CREATE TABLE BICYCLE
  ( FRAME        CHAR(50) NOT NULL,
    SEAT         CHAR(50) NOT NULL,
    HANDLEBAR    CHAR(50) NOT NULL,
    LEFT_PEDAL   CHAR(50) NOT NULL,
    RIGHT_PEDAL  CHAR(50) NOT NULL,
    FRONT_WHEEL  CHAR(50) NOT NULL,
    LEFT_WHEEL   CHAR(50) NOT NULL,
    PRIMARY KEY (FRAME),
    FOREIGN KEY (FRAME) REFERENCES FRAME (ID),
    FOREIGN KEY (SEAT) REFERENCES SADDLE (ID),
    FOREIGN KEY (HANDLEBAR) REFERENCES HANDLEBAR (ID),
    FOREIGN KEY (LEFT_PEDAL) REFERENCES PEDAL (ID),
    FOREIGN KEY (RIGHT_PEDAL) REFERENCES PEDAL (ID),
    FOREIGN KEY (FRONT_WHEEEL) REFERENCES WHEEL (ID),
    FOREIGN KEY (REAR_WHEEL) REFERENCES WHEEL (ID) )

CREATE TABLE FRAME
  ( ID          CHAR(50),
    PRIMARY KEY (ID) )

CREATE TABLE SADDLE
  ( ID          CHAR(50),
    PRIMARY KEY (ID) )

CREATE TABLE HANDLEBAR
  ( ID          CHAR(50),
    PRIMARY KEY (ID) )

CREATE TABLE PEDAL
  ( ID          CHAR(50),
    PRIMARY KEY (ID) )

CREATE TABLE WHEEL
  ( ID          CHAR(50),
    PRIMARY KEY (ID) )
\end{verbatim}
\end{code}

\end{comment}
%%%%%%%%%%%%%%%%%%%%%%%%%%%%%%%%%%%%%%%%%%%%%%%%%%%%


\clearpage

\section{Regular models} 

    The exercises in this set were designed to provide a variety of
modeling tasks.

\subsection{Exercise 1}

\begin{itshape}
Do the following:
  \begin{enumerate}
  \item Write an information model that describes the logical content of a report.
        Assume that a report consists of a title and one or more authors, together 
        with the publication date. It may have an abstract and may have a table of
        contents. The body of the report consists of at least two sections.
        Further divisions of the report are subsections and sub-subsections.
        Figures and tables may also be included within any sub-subsection, or
        higher level partitions. The report may have a bibliography.
  \item Write an information model that describes a book. A book is similar to
        a report with the following exceptions. A book may consist of two or more
        parts, each of which must contain two or more chapters. Each chapter
        contains at least two sections. There is always a table of contents and 
        there is never an abstract, although it may have a preface which serves the
        same purpose. A book may have an index.
  \item Does the above description apply to all books?
  \item Create an information model that supports both reports and books. Include
        anything extra that you feel is necessary that is missing from the above
        descriptions.
  \end{enumerate}
\end{itshape}

    This is a start at the first part of the exercise.

\begin{code}
\begin{verbatim}
*)
SCHEMA exercise_4_1_1;

TYPE title = STRING; END_TYPE;
TYPE subject = STRING; END_TYPE;
TYPE page_id = STRING; END_TYPE;
TYPE author = STRING; END_TYPE;
TYPE date = STRING; END_TYPE;
TYPE text = LIST [1:?] OF STRING; END_TYPE;
TYPE table = LIST [1:?] OF BINARY; END_TYPE;
TYPE figure = LIST [1:?] OF BINARY; END_TYPE;

ENTITY report;
  start : front_material;
  body  : LIST [2:?] OF section;
  finish : OPTIONAL bibliography;
END_ENTITY;

ENTITY front_material;
  title : subject;
  authors : SET [1:?] OF author;
  issued   : date;
  abstract : OPTIONAL text;
  contents : OPTIONAL table_of_contents;
END_ENTITY;

ENTITY table_of_contents_entry;
  subject : subject;
  placement : OPTIONAL page_id;
END_ENTITY;

ENTITY table_of_contents;
  entries : LIST [1:?] OF UNIQUE table_of_contents_entry;
END_ENTITY;

ENTITY section;
  title : subject;
  body  : OPTIONAL text_andor_insert;
  subdivision : LIST OF sub_section;
END_ENTITY;

ENTITY sub_section;
  title : subject;
  body : OPTIONAL text_andor_insert;
  subdivision : LIST OF subsub_section;
END_ENTITY;

ENTITY text_andor_insert;
  words         : LIST OF UNIQUE text;
  tabulars      : LIST OF UNIQUE table;
  illustrations : LIST OF UNIQUE figure;
END_ENTITY;

-- and so on

END_SCHEMA;
(*
\end{verbatim}
\end{code}

%%%%%%%%%%%%%%%%%%%%%%%%%%%%%%%%%%%%%%%%%%%%
\begin{comment}

\subsection{Exercise 4.2}

\begin{itshape}
Write an information model corresponding to the following description.

    An international company has a number of ongoing development projects. A project
 has a unique name and is located in a specific city.
There are a number of suppliers to the company. The suppliers have names and may have
several branches, each in a different city. Suppliers with identical names do not
have branches in the same city. A supplier may supply one or more different kinds of
parts to the company. A part is identified by a catalogue number, and also has
a short description. Projects purchase parts from the nearest location which stocks
the part. The company keeps a record of the purchase orders (i.e., part, supplier,
and quantity) of each project.
\end{itshape}


\subsection{Exercise 4.3}

\begin{itshape}
Write an information model corresponding to the following description.

    A University is organised into academic, research and administrative departments.
Administrative staff may work in any kind of department, but neither academic staff
nor research staff work in the administrative departments. Academic staff teach
courses and may do research work. Research staff are limited to research work only.
Administrative staff neither teach nor do research.
All undergraduate and some graduate students attend courses.
There is a fee for each course, the amount of which differs
according to the course. Students are graded on each course they attend, with a grade
having a value between 0 and 100. It is a tradition, however, of the University that
 no student has ever been graded at either 0 or 100. Some undergraduate students may
 be employed part-time to
assist the administrative staff, but only if their grade is 75 or more. 
  All staff get paid a salary, the amount of which depends
on their position. Graduate students do research. They may teach not more than
two courses, and are paid at a fixed rate per course. No person under the age
of 18 may be paid, and the retirement age is 65. 

\end{itshape}

\subsection{Exercise 4.4} \label{ex:fluid}

\begin{itshape}
Write an information model about the delivery of items 
according to the
following description.

    The currency of Fluidistan is the G. This is divided into the smaller p and
z units, where G1 = 8p and 1p = 16z. The amount 190z, for example, is written
as G1-7-12. Linear measures in Fluidistan are the inch and foot, where 1 foot equals
12 inches. The weight measures are the pound and ounce, where 1 pound is 16 ounces.

    The government of Fluidistan operate a mail delivery service for certain kinds of
item. There are also private delivery services which will accept any kind of item.
The following are the regulations governing the Fluidistan mail service.
\begin{description}
\item[Post Cards:] The card rate is 1p~3z. To qualify for the card rate a postcard
must be of a uniform thickness and no thinner than 0.007 inches. It must be no
larger than 4.25 by 6 inches and no smaller than 3.5 by 5 inches.

\item[Letters:] The letter rate is 1p~13z for letters weighing one ounce or less.
The rate increases by 1p~6z for each additional ounce or part thereof. An item
weighing more than 11 ounces cannot be sent at the letter rate. Letters less than
one ounce are non-standard if the length is greater than 11.5 inches or the
height is greater than 6.125 inches or the thickness is greater than 0.25 inches
or the length to height ratio is not between 1.3 and 2.5 inclusive. Non-standard
letters are subject to a surcharge of 10z in addition to the standard rate.

\item[Parcels:] The parcel rate is G2-4-2 for items not exceeding two pounds in 
weight. The rate increases by by 2p~11z for each additional pound or part thereof,
provided the weight is not greater than ten pounds. Above ten pounds the rate 
increases by 2p~8z for each additional pound or part thereof. Note: Parcels
weighing less than fifteen pounds and whose length plus girth exceeds seven feet
are chargeable at the fifteen pound rate.

\item[Size Standards:] Items whose thickness is less than 0.007 inches are not
accepted for delivery. Items less than 0.25 inches in thickness must be rectangular
in shape and at least 3.5 inches high and at least 5 inches long. Items
weighing more than 70 pounds are not accepted for delivery.

\item[Environmental:] Neither hazardous materials nor live or dead animals
 will be accepted for delivery. All 
items, except cards, must be enclosed in some wrapping. Items enclosed in
environmentally sound and recycleable wrapping are entitled to a discount of ten
percent of the applicable rate; if this results in a fraction of a z, the
rate is rounded up to the nearest z.

\end{description}

\end{itshape}

\end{comment}
%%%%%%%%%%%%%%%%%%%%%%%%%%%%%%%%%%%%%%%%%%%%%%%%%%






\end{document}


%\chapter{Ambiguities}
\section{Ambiguities}

\begin{exercises}

\item Write three sentences that are ambiguous along the lines of
      `The chicken is ready to eat'. Can you think of more?

\item How many ways can a date be writen? Are they all in use? What other
      sorts of calendrical systems are there in addition to Anno Domine style 
      dates?

\item Ouida is said to have said
    \begin{quote} `All rowed fast but none so fast as stroke.' \end{quote}
      The word `stroke' has many other meanings than in this quotation.
      How many ways can you interpret `stroke'? Write sentences which provide
      examples for each meaning.

\item A definition of a clock from Webster's (Ninth New Collegiate 
      Dictionary, 1985) is: \begin{quote} A registering device with
      a dial and indicator attached to a mechanism to measure or gauge
      its functioning or to record its output. \end{quote} 
      How many descriptions can you
      think of for a timepiece? List them.

\item Also from Webster's is this definition of a cloverleaf:
     \begin{quote}
     A road plan passing one highway over another and routing turning
     traffic onto connecting roadways which branch only to the right
     and lead round in a circle to enter the other highway from the 
     right and thus merge traffic withut left-hand turns or direct
     crossings.
     \end{quote}
     Write a more general description that makes no assumption about 
     which side of the road people are allowed (or required) to drive on.

\end{exercises}

\clearpage
%\chapter{Simple models}
\section{Simple models}

\begin{exercises}

\item Develop a categorization system for non-fiction books. (Hint ---
      think how they are organized in a library).

\item Develop a categorization scheme for the goods sold in your local
      grocery store.

\item A book is written by one or more authors and is printed by a
      single publisher. Sketch a model that captures these statements.

\item Sketch a model of a bicycle. Assume that a bicycle consists of a frame,
      a saddle, handelbars, pedals, and two wheels.

\item How does your model change if you include a chain connecting the 
      pedals to the rear wheel, and also if you consider that a wheel has a 
      hub, spokes, a rim and a tire?


\end{exercises}

\clearpage
%\chapter{Regular models}  \label{chap:regular}
\section{Regular models}  \label{chap:regular}


\begin{exercises}
\item Do the following:
  \begin{enumerate}
  \item Write an information model that describes the logical content of 
        a report. Assume that a report consists of a title and one or 
        more authors, together with the publication date. It may have 
        an abstract and may have a table of
        contents. The body of the report consists of at least two sections.
        Further divisions of the report are subsections and sub-subsections.
        Figures and tables may also be included within any sub-subsection, or
        higher level partitions. The report may have a bibliography.
  \item Write an information model that describes a book. A book is similar 
        to a report with the following exceptions. A book may consist of 
        two or more parts, each of which must contain two or more chapters. 
        Each chapter contains at least two sections. There is always a table 
        of contents and there is never an abstract, although it may have 
        a preface which serves the same purpose. A book may have an index.
  \item Does the above description apply to all books?
  \item Create an information model that supports both reports and books. 
        Include anything extra that you feel is necessary that is missing 
        from the above descriptions.
  \end{enumerate}

\item Write an information model corresponding to the following description.

    An international company has a number of ongoing development projects. 
A project has a unique name and is located in a specific city. There are 
a number of suppliers to the company. The suppliers have names and may have
several branches, each in a different city. Suppliers with identical names 
do not have branches in the same city. A supplier may supply one or more 
different kinds of parts to the company. A part is identified by a catalogue 
number, and also has a short description. Projects purchase parts from 
the nearest location which stocks the part. The company keeps a record 
of the purchase orders (i.e., part, supplier, and quantity) of each project.

\item Write an information model corresponding to the following description.

    A University is organised into academic, research and administrative 
departments. Administrative staff may work in any kind of department, but 
neither academic staff nor research staff work in the administrative 
departments. Academic staff teach courses and may do research work. 
Research staff are limited to research work only. Administrative staff 
neither teach nor do research. All undergraduate and some graduate 
students attend courses. There is a fee for each course, the amount of 
which differs according to the course. Students are graded on each course 
they attend, with a grade having a value between 0 and 100. It is a 
tradition, however, of the University that no student has ever been 
graded at either 0 or 100. Some undergraduate students may be employed 
part-time to assist the administrative staff, but only if their grade 
is 75 or more. All staff get paid a salary, the amount of which depends
on their position. Graduate students do research. They may teach not more than
two courses, and are paid at a fixed rate per course. No person under the age
of 18 may be paid, and the retirement age is 65. 

\item \label{ex:fluid} Write an information model about the delivery of items 
according to the following description.

    The currency of Fluidistan is the G. This is divided into the 
smaller p and z units, where G1 = 8p and 1p = 16z. The amount 190z, 
for example, is written as G1-7-12. Linear measures in Fluidistan are 
the inch and foot, where 1 foot equals 12 inches. The weight measures 
are the pound and ounce, where 1 pound is 16 ounces.

    The government of Fluidistan operate a mail delivery service for 
certain kinds of item. There are also private delivery services which 
will accept any kind of item. The following are the regulations governing 
the Fluidistan mail service.
\begin{description}
\item[Post Cards:] The card rate is 1p~3z. To qualify for the card rate 
a postcard must be of a uniform thickness and no thinner than 0.007 inches. 
It must be no larger than 4.25 by 6 inches and no smaller than 3.5 by 5 inches.

\item[Letters:] The letter rate is 1p~13z for letters weighing one ounce 
or less. The rate increases by 1p~6z for each additional ounce or part 
thereof. An item weighing more than 11 ounces cannot be sent at the 
letter rate. Letters less than one ounce are non-standard if the length 
is greater than 11.5 inches or the height is greater than 6.125 inches 
or the thickness is greater than 0.25 inches or the length to height 
ratio is not between 1.3 and 2.5 inclusive. Non-standard letters are 
subject to a surcharge of 10z in addition to the standard rate.

\item[Parcels:] The parcel rate is G2-4-2 for items not exceeding two 
pounds in weight. The rate increases by by 2p~11z for each additional 
pound or part thereof, provided the weight is not greater than ten pounds. 
Above ten pounds the rate increases by 2p~8z for each additional pound 
or part thereof. Note: Parcels weighing less than fifteen pounds and 
whose length plus girth exceeds seven feet are chargeable at the 
fifteen pound rate.

\item[Size Standards:] Items whose thickness is less than 0.007 inches 
are not accepted for delivery. Items less than 0.25 inches in thickness 
must be rectangular in shape and at least 3.5 inches high and at 
least 5 inches long. Items weighing more than 70 pounds are not accepted 
for delivery.

\item[Environmental:] Neither hazardous materials nor live or dead animals
 will be accepted for delivery. All items, except cards, must be enclosed 
in some wrapping. Items enclosed in environmentally sound and recycleable 
wrapping are entitled to a discount of ten percent of the applicable rate; 
if this results in a fraction of a z, the rate is rounded up to the 
nearest z.

\end{description}

\item \label{ex:bmd} Create an information model for the following.

    The BMD authority is responsible for recording births, marriages, divorces
    and deaths. At birth the name of the child, its sex, its 
    date of birth, and its parents are recorded. The spouses and the date of
    the marriage are recorded. A similar record is kept for each divorce.
    The divorced couple and date are recorded. Deaths are
    recorded after the issuance of a death certificate. The 
    date of death and the signatory of the death certificate are recorded.
    The legal age for marriage is eighteen, but minors between the ages of 
    sixteen
    and eighteen may marry with their parents' consent. Upon request, the BMD
    authority will provide information on the marital status of anybody (i.e.,
    whether they are single, married, divorced, widowed or deceased). They will
    also provide, to the person concerned, a listing of all their ancestors.

\end{exercises}

\clearpage
%\chapter{Mathematical models}
\section{Mathematical models}

\begin{exercises}

\item Produce a model of the following cartesian geometry items.
  
    A \emph{point} is a location in space and is defined by its location
with respect to the origin of a coordinate system. The \emph{location}
is represented by the $x$, $y$ and $z$ coordinate values.

    A \emph{vector} is a direction ad is represented in terms of three numbers
corresponding to its relative extent in the $x$, $y$ and $z$ coordinate 
directions.

    A \emph{straight line} can be respresented by a point on the line
and a vector denoting its direction.

    A \emph{plane} can be represented by a point through which it passes, 
and the direction of the normal to the plane surface.

    A \emph{circle} is a planar curve (i.e., it lies in a plane). It can be
represented by a center point, the normal to the plane in which it lies,
and a non-negative radius value.

    An \emph{ellipse} is a planar curve. It can be
represented by a center point, the normal to the plane in which it lies,
major and minor non-negative radius values, and the direction of the major
radius.

    A \emph{parabola} is a planar curve. It can be
represented by a vertex point, the normal to the plane in which it lies,
a non-negative focal distance, and the direction of the focus from
the vertex.

\item Write a model that captures the following information about
      a very simple bridge.

    Simplistically, a bridge can be considered to be a simply supported beam, of
    length $l$, with width $b$ and depth $h$. The beam is of uniform material
    having density $d$ and modulus of elasticity $E$.
    As well as its own weight, a bridge must support a 
    uniformly distributed load $L$, and a point load $P$ at the center of the
    span. There are limits on the maximum deflection, $y$, of the span under load
    and also limits on the maximum stress, $s$, in the beam.

    The moment of inertia, $I$, of the beam cross-section is given by
    \begin{displaymath} I = \frac{bh^{3}}{12}  \end{displaymath}
    and the maximum stress at any beam cross-section is given by
    \begin{displaymath} s = \frac{Mh}{2I}  \end{displaymath}
    where $M$ is the bending moment.

    For a beam of length $l$ with a total uniformly distributed load of $W$,
    the maximum  bending moment is
    \begin{displaymath} M = \frac{Wl}{8}    \end{displaymath}
    while for a point load $W$ at the center of the beam it is
    \begin{displaymath} M = \frac{Wl}{4}    \end{displaymath}
    The maximum deflection of a uniformly loaded beam is
    \begin{displaymath} y = \frac{5Wl^{3}}{384EI}  \end{displaymath}
    and for a center loaded beam is
    \begin{displaymath} y = \frac{Wl^{3}}{48EI}    \end{displaymath}

    Bending moments, deflections and stresses are additive with respect to 
    loading conditons. That is, the total bending moment is the sum of the
    bending moments for the uniform load case and the point load case.

\end{exercises}

\clearpage
%\chapter{Scope changes}
\section{Scope changes}

\begin{exercises}

\item In order to respond to increasing budget deficits and voter antipathy
      to increased taxes, it has been decided to combine the Car Registration
      Authority (see the example model) and the BMD Authority 
      (see exercise~\ref{ex:bmd} in \S\ref{chap:regular}).
      Integrate the two information models to represent
      the combined Authority.

\item Because of the rising unemployment rate, the government of Fluidistan
      is planning to increase the number of bureaucrats it employs by 
      splitting its postal service into three parts. One will be responsible 
      for setting the rules and regulations, another will be responsible 
      for delivering cards and letters, while the third will be responsible 
      for parcel delivery.

      Starting with the model resulting from exercise~\ref{ex:fluid}
      in \S\ref{chap:regular}, produce two models, 
      one for the letter and card branch and the other for
      the parcel branch. Try and minimise changes to the starting model 
      and also try and minimise the overall amount of work to produce
      (i.e., create and document) the new models.

      What happens when the rules and regulations change? How would you
      cater for the possibility that a third model might be required
      for the regulatory branch?

\end{exercises}


\end{document}

