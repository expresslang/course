% advexp.tex   EXPRESS Course  "Advanced stuff"
% Created by Peter R Wilson, 1992 -- 2004
%\documentclass[11pt]{article}  
%\usepackage{vgraph}
%\usepackage{headfoot}

%%%%%\notestrue
%%%%%\notesfalse

%%\begin{document}

\pagestyle{empty}
\bodsiz

\vspace*{\beftit}
\begin{center}
{\titsiz\bfseries Information Modeling Using EXPRESS --- \\ Advanced Concepts}
\end{center}
\vspace{\beftit}
\vspace{\beftit}
\begin{center}
%%%%\textbf{Peter R. Wilson}
\end{center}

\clearpage

\ifnotes
  \pagestyle{plain}
\else
%%  \pagefooter{}{}{{\small Peter R. Wilson}}
  \pagestyle{empty}
\fi

\begin{remarks}
\remintro
NB. NOTES ARE ON EVEN NUMBERED (LEFT HAND) PAGES AND THE PRESENTATION ON
ODD NUMBERED (RIGHT HAND) PAGES.

\remtitle{Extensibles}

EXPRESS Edition 2 (2004) introduced EXTENSIBLE types. These are
SELECT and ENUMERATION types whose item lists can be extended.
This provides increased flexiblity and reuse in modeling. 

    In some sense, EXTENSIBLE types are analagous to SUBTYPEs, except that 
subtyping reduces the domain while extending increases the domain.

\remend
\end{remarks}


\vtitle{Extensibles}

\begin{itemize}
\item SELECT specifies a list of things to choose from
\item ENUMERATION specifies a list of items (names)
\item It is possible to extend these lists
\end{itemize}

\begin{verbatim}
TYPE approval = EXTENSIBLE ENUMERATION OF
     (approved, rejected); END_TYPE;

TYPE your_approval = ENUMERATION BASED_ON
     approval WITH (cancelled); END_TYPE;

TYPE my_approval = EXTENSIBLE ENUMERATION 
     BASED_ON approval WITH (pending); 
END_TYPE;
\end{verbatim}

\begin{remarks}
\remintro

\remtitle{Relationships}

If one entity serves as an attribute of another entity, then the two
entities are related, or more precisely there is a relationship
between the two entities.

A \texttt{door} must have at least two \texttt{hinge}s.

A \texttt{door} must have a \texttt{handle}.


\remend
\end{remarks}



\vtitle{Relationships}

    When an attribute of an entity is of type entity, then a relationship is
established between the two entity types.
\begin{verbatim}
ENTITY door;
  hinges : SET [2:?] OF hinge;
  knob   : handle;
  ...
END_ENTITY;

ENTITY hinge;
  id : name;
  ...
END_ENTITY;

ENTITY handle;
 ...
END_ENTITY;
\end{verbatim}

\begin{remarks}
\remintro


\remtitle{Cardinality}

Cardinality is how many of one thing is needed/used by another thing.
(`Cardinality' is related to `cardinal number' --- a number used for 
counting).

    The cardinality of the `attribute' entity with respect to the `owning'
entity is defined in the owning entity.

    By default the cardinality of the owning entity with respect to the
attribute entity is zero to many.

    The attribute entity can change this via an INVERSE attribute.

\remend
\end{remarks}

\vtitle{Cardinality}

\begin{itemize}
\item Cardinality constraint from the referring entity to the referenced
entity is specified via the attribute definition (e.g OPTIONAL, aggregation).
\item Default cardinality constraint from the referenced to the referring
entity is zero to many \verb|[0:?]|. 
\item \verb|INVERSE| attributes can be used to specify other cardinalities.
\end{itemize}

\begin{remarks}
\remintro

\remtitle{Cardinality (cont)}

In the last model a \texttt{door} required one \texttt{handle} but from 
the point of view of a \texttt{handle}, any number could be used on
a \texttt{door} (or yhis could be restated as \texttt{handles} know
nothing about \texttt{doors}).

In this model a \texttt{handle} can be used on either one or no doors.

\remend
\end{remarks}

\vtitle{Cardinality (cont)}

\begin{verbatim}
ENTITY door;
  hinges : SET [2:?] OF hinge;
  knob   : handle;
  ...
END_ENTITY;

ENTITY handle;
 ...
INVERSE
  knob_for : SET [0:1] OF door FOR knob;
END_ENTITY;
\end{verbatim}
Here, the INVERSE clause states that a handle is `used' by zero or one doors.

\begin{remarks}
\remintro

\remtitle{ABSTRACT ENTITY}

    This was introduced in EXPRESS Edition 2.

    It has to be instantiated (e.g., appear in a Part 21 file)
via a SUBTYPE in which the specific type of each attribute is specified.

    The model here says that a \texttt{general\_approval} has a \texttt{status}
which is of type \texttt{approval\_status} (whatever that is), and 
\texttt{approved\_items} (which is a SET of (unnamed) ENTITY). An instantiable
SUBTYPE would have to replace \texttt{GENERIC\_ENTITY} by a named entity,
say \texttt{plan} (whatever that might be), to have a set of 
approved \texttt{plan}s.
 

\remend
\end{remarks}

\vtitle{ABSTRACT ENTITY}

    An ENTITY that can only be instantiated via its SUBTYPEs.

\begin{verbatim}
ENTITY general_approval ABSTRACT;
  approved_items : SET OF GENERIC_ENTITY;
  status         : aproval_status;
END_ENTITY;
\end{verbatim}

\begin{itemize}
\item GENERIC\_ENTITY stands for any ENTITY
\item All attributes must have specific types before
      instantiation is possible.
\item An attribute type can be redeclared to a more
      specific kind in a SUBTYPE
\end{itemize}


\begin{remarks}
\remintro

\remtitle{Example person Entity}

   We are going to use this as the basis for the next few slides.

\remend
\end{remarks}

\vtitle{Example person ENTITY}

\begin{verbatim}
ENTITY person;
  first_name : STRING;
  last_name  : STRING;
  nickname   : OPTIONAL STRING;
  ss_no      : INTEGER;
  gender     : sex;
  spouse     : OPTIONAL person;
  children   : SET [0:?] OF person;
UNIQUE
  un1 : ss_no;
WHERE
  w1 : (EXISTS(spouse) AND 
        gender <> spouse.gender)
       OR NOT EXISTS(spouse);
END_ENTITY;
\end{verbatim}

\begin{remarks}
\remintro

\remtitle{IS-A Relationship}

    The database world talks about IS-A relationships, for instance
THIS IS-A THAT, or A CAR IS-A (kind of) VEHICLE.

In EXPRESS a SUBTYPE IS-A (more special kind of its) SUPERTYPE(s).

Conversely a SUPERTYPE IS-A (more general kind of its) SUBTYPE(s).


\remend
\end{remarks}

\vtitle{IS-A Relationship}

\begin{itemize}
\item EXPRESS supports the IS-A relationship via subtyping.
\item Entities S1, S2, \ldots can be declared to be SUBTYPES of entity E.
This also effectively declares E to be a SUPERTYPE of S1, S2, etc.

That is, S1 is-a E, S2 is-a E, etc. Also, E may-be an S1, E may-be an S2.
\item An entity may be both a SUB- and a SUPERTYPE.
\item An entity may be a SUBTYPE of more than one entity.
\item SUPER/SUBTYPING may be used for many purposes.
\end{itemize}

\begin{remarks}
\remintro

\remtitle{Inheritance}

    A SUBTYPE is a special kind of its SUPERTPE(s). There are fewer instances
of a SUBTYPE than of its SUPERTYPE. For example, there are fewer CARS than 
there are VEHICLES. 

    A SUBTYPE inherits all the attributes and constraints of its SUPERTYPE(s).

    A SUBTYPE can have additional attributes and constraints.

    This revised \texttt{person} model eliminates the original WHERE rule
about spouses being of opposite sex. We can also talk about a \texttt{person}
without having to identify the person's gender.

\remend
\end{remarks}

\vtoptitle{Inheritance}

A SUBTYPE inherits all the attributes and constraints of its SUPERTYPE(s).
\begin{verbatim}
ENTITY person;
  first_name : STRING;
  last_name  : STRING;
  ss_no      : INTEGER;
  children   : SET [0:?] OF person;
UNIQUE
  un1 : ss_no;
END_ENTITY;

ENTITY male
  SUBTYPE OF (person);
  wife : OPTIONAL female;
END_ENTITY;

ENTITY female
  SUBTYPE OF (person);
  husband : OPTIONAL male;
END_ENTITY;
\end{verbatim}

\begin{remarks}
\remintro

\remtitle{SUBTYPE instance constraints}

We can use this model to talk about a 
\begin{itemize}
\item A person
\item A person who is an employee
\item A person who is a student
\item A person who is an employee and who is also a student
\end{itemize}

\remend
\end{remarks}

\vtitle{SUBTYPE instance constraints}

\begin{itemize}
\item In general, an instance of a Supertype may involve instances of zero or 
more of its Subtypes.
\begin{verbatim}
ENTITY person;
  ...
END_ENTITY;

ENTITY employee
  SUBTYPE OF person;
  ...
END_ENTITY;

ENTITY student
  SUBTYPE OF person;
  ...
END_ENTITY;
\end{verbatim}
\item If this is not the required behaviour, then the `instance set' can be
constrained.
\end{itemize}

\begin{remarks}
\remintro

\remtitle{SUBTYPE\_CONSTRAINT}

The SUBTYPE\_CONSTRAINT construct was introduced in EXPRESS Edition 2.

    In Edition 1 the constraint specification was lexically embedded in
the definition of the Supertpye entity. If a new subtytpe was introduced
in a different Schema that imported the Supertype there was no convenient 
method, apart from changing the original Supertype definition, of constraining
the use of the new Subtype.

   Multiple SUBTYPE\_CONSTRAINTs can be applied to a Supertype. 
The constraints are additive. (In EXPRESS you cannot eliminate a constraint).


\remend
\end{remarks}

\vtitle{SUBTYPE\_CONSTRAINT}

\begin{verbatim}
SUBTYPE_CONSTRAINT sc FOR ent;
-- constraints
END_SUBTYPE_CONSTRAINT;
\end{verbatim}
specifies SUBTYPE constraints for ENTITY ent.

Several SUBTYPE\_CONSTRAINTs can be specified for any
given ENTITY. The constraints are additive.

\begin{remarks}
\remintro

\remtitle{SUBTYPE Constraint Summary}

In general, an instance of a Supertype can involve any of its Subtypes.

The constraints are used to eliminate certain combinations of Subtypes.

The particulars are described later.

\remend
\end{remarks}

\vtitle{SUBTYPE Constraint Summary}

\begin{itemize}
\item No constraints: An instance of the Supertype involves zero or 
      more Subtype instances.
\item ABSTRACT SUPERTYPE: An instance of the Supertype must involve 
      one or more Subtype instances.
\item TOTAL\_OVER(x,y) means that every instance of the Supertype must 
      involve an instance of at least one of the listed Subtypes.
\item ONEOF(x,y,z) means that one and only \emph{one of} the listed Subtypes
can be instanced with an instance of the Supertype.
\item (x ANDOR y) means that an instance of the Supertype may be accompanied by
instances of the Subtypes x \emph{and/or} y (the default condition).
\item (x AND y)  means that an instance of the Supertype may be accompanied by
instances of the Subtypes x \emph{and} y.
\end{itemize}

\begin{remarks}
\remintro

\remtitle{ABSTRACT SUPERTYPE}

An ABSTRACT SUPERTYPE can only be instantiated in conjunction with non-ABSTRACT
subtype(s).

\remend
\end{remarks}

\vtoptitle{ABSTRACT SUPERTYPE}

\begin{itemize}
\item An entity does not have to declare itself to be a SUPERTYPE. It is a
SUPERTYPE if it is mentioned by a SUBTYPE.
\item In some cases, a Supertype is not to be instantiated without one of
its Subtypes. The entity can be constrained to be an ABSTRACT SUPERTYPE.
\begin{verbatim}
ENTITY mammal
  ...
END_ENTITY;

SUBTYPE_CONSTRAINT sc_abs FOR mammal;
  ABSTRACT SUPERTYPE;
END_SUBTYPE_CONSTRAINT;

ENTITY dog
  SUBTYPE OF mammal;
  ...
END_ENTITY;
\end{verbatim}
\end{itemize}

\begin{remarks}
\remintro

\remtitle{TOTAL\_OVER}

This was introduced in Edition 2 (I have failed to find any use for it).

It means (I think) that the listed Subtypes completely cover the domain
of the Supertype. Further, every instance of the Supertype that includes
Subtype instances must include an instance of one of the listed subtypes.


\remend
\end{remarks}

\vtitle{TOTAL\_OVER}

\begin{verbatim}
ENTITY person;
...
END_ENTITY;

SUBTYPE_CONSTRAINT adultchild FOR person;
  TOTAL_OVER(adult,child);
END_SUBTYPE_CONSTRAINT;

ENTITY child SUBTYPE OF (person);
END_ENTITY;

ENTITY adult SUBTYPE OF (person);
END_ENTITY;

ENTITY student SUBTYPE OF (person);
END_ENTITY;
\end{verbatim}

Every person is either a child or an adult. A student
is also either a child or an adult.



\begin{remarks}
\remintro

\remtitle{ONEOF}

A ONEOF constraint means that one and only ONE OF the listed subtypes
can be used in an instance of the Supertype.

    Here the constraint is that a person cannot be simultaneously 
a male and a female. Note that if the constraint was not there (as 
in the earlier model) it
would mean that the model catered for hermaphrodites, which would introduce
a new set of problems.

\remend
\end{remarks}

\vmintitle{ONEOF}

\begin{verbatim}
ENTITY person;
  first_name : STRING;
  last_name  : STRING;
  ss_no      : INTEGER;
  children   : SET [0:?] OF person;
UNIQUE
  un1 : ss_no;
END_ENTITY;

SUBTYPE_CONSTRAINT mf FOR person;
  ONEOF(male, female);
END_SUBTYPE_CONSTRAINT;

ENTITY male
  SUBTYPE OF (person);
  wife : OPTIONAL female;
END_ENTITY;

ENTITY female
  SUBTYPE OF (person);
  husband : OPTIONAL male;
END_ENTITY;
\end{verbatim}

\begin{remarks}
\remintro

\remtitle{ANDOR}

P ANDOR Q means that the following combinations of subtypes are allowed:
\begin{itemize}
\item P only
\item Q only
\item P and Q together.
\end{itemize}
That is P and/or Q are allowed.

The unconstrained relationship between Subtypes (the default) is ANDOR.

In the example model the constraint might as well not be there.

\remend
\end{remarks}

\vmintitle{ANDOR}

\begin{verbatim}
ENTITY person;
  first_name : STRING;
  last_name  : STRING;
  ss_no      : INTEGER;
  children   : SET [0:?] OF person;
UNIQUE
  un1 : ss_no;
END_ENTITY;

SUBTYPE_CONSTRAINT es FOR person;
  employee ANDOR student;
END_SUBTYPE_CONSTRAINT;

ENTITY employee
  SUBTYPE OF (person);
  salary : REAL;
END_ENTITY;

ENTITY student
  SUBTYPE OF (person);
  fees : REAL;
END_ENTITY;
\end{verbatim}

\begin{remarks}
\remintro

\remtitle{AND}

P AND Q means that if there is an instance of P it must be accompanied
by an instance of Q, and vice-versa --- either both or none.

The example shows that the constraints may be complex (logical) expressions.

Unconstrained there are 15 possible combinations 
(from Person to a male, female, citizen, alien person).

With the given constraints there are only 5 
(Person, (fe)male citizen, (fe)male alien).


\remend
\end{remarks}

%\vuptitle{AND}
\vmintitle{AND}

\begin{verbatim}
ENTITY person;
  ...
END_ENTITY;

SUBTYPE_CONSTRAINT mf_and_ca FOR person;
  ONEOF(male, female) AND
  ONEOF(citizen, alien);
END_SUBTYPE_CONSTRAINT;

ENTITY male SUBTYPE OF (person);
 ...
END_ENTITY;

ENTITY female SUBTYPE OF (person);
 ...
END_ENTITY;

ENTITY citizen SUBTYPE OF (person);
END_ENTITY;

ENTITY alien SUBTYPE OF (person);
END_ENTITY;
\end{verbatim}

\begin{remarks}
\remintro

\remtitle{SUBTYPEs}

Much of this list has already been touched on. The first
item is part of the `meaning' of SUBTYPE. 

The following
example includes examples of the last 3 elements in the list.

\remend
\end{remarks}

\vtitle{SUBTYPEs}

\begin{itemize}
\item A Subtype is a specialisation of its Supertype(s).
\item New attributes may be added.
\item New constraints may be added.
\item Attributes may be `retyped' (i.e their domains may be specialised in a
      compatible manner).
\end{itemize}

\begin{remarks}
\remintro

\remtitle{SUBTYPEs (cont)}

 A simple example showing:
\begin{itemize}
\item Attribute redeclaration
\item Adding attribute(s)
\item Adding constraint(s)
\end{itemize}

\remend
\end{remarks}

\vtitle{SUBTYPEs (cont)}

\begin{verbatim}
ENTITY circle;
  radius : NUMBER;
  center : point;
END_ENTITY;

ENTITY specialised_circle
  SUBTYPE OF (circle);
  SELF\circle.radius : REAL;    -- retyped 
  shade  : colour; -- additional attribute
WHERE
  SELF\circle.radius > 3.0; -- add constraint 
END_ENTITY;
\end{verbatim}

\begin{remarks}
\remintro

\remtitle{QUERY Expression}

    Now we are getting away from structural modeling.

\remend
\end{remarks}

\vtitle{QUERY Expression}

    The query expression evaluates a logical expression against each element
of an aggregation, returning an aggregation of all the elements for which
the logical expression is TRUE.

The syntax is roughly:
\begin{verbatim}
QUERY( temp <* agg | lexp)
\end{verbatim}
where \verb|temp| is the name of a temporary variable, \verb|agg| is the 
aggregation, and \verb|lexp| is the logical expression.

For example, assuming that a person's attributes included the age of the person,
\begin{verbatim}
QUERY(t <* persons | t.age >= 21)
\end{verbatim}
would return all the people whose age was 21 or greater. 

\begin{remarks}
\remintro

\remtitle{QUERY (cont)}

You can't actually write this function in EXPRESS (if you could the
QUERY expression would probably not have been invented), as there is 
no LOGICAL\_EXPRESSION type in the language.

    An example of its use follows.

\remend
\end{remarks}

\vtitle{QUERY (cont)}

    The effect of QUERY is similar to the pseudo-function below.
\begin{verbatim}
FUNCTION q(agg  : AGGREGATE OF GENERIC;
           lexp : LOGICAL_EXPRESSION;) 
          : AGGREGATE OF GENERIC;
LOCAL
  result : AGGREGATE OF GENERIC := [];
END_LOCAL;
  REPEAT i := 1 TO SIZEOF(agg);
    IF (lexp = TRUE) THEN
      result := result + agg[i];
    END_IF;
  END_REPEAT;
RETURN(result);
END_FUNCTION;
\end{verbatim}

\begin{remarks}
\remintro

\remtitle{RULE}

A WHERE rule in an ENTITY applies to each and every instance of the ENTITY.

A RULE is a constraint that can be applied to either some instances of
a particular ENTITY or to combinations of instances of different ENTITY 
(types).


    Given a database of instances, each RULE is applied to every applicable
instance in the database to determine if the instance conforms to the 
constraint.

    EXPRESS assumes that every (ENTITY) instance has a unique identifier, 
although it does not specify what that might be. You could have two (or more)
instances of a \texttt{point} with the same coordinate values but they are 
still distinguisable fronm each other in the storage system.

\remend
\end{remarks}

\vtoptitle{RULE}

\begin{itemize}
\item Local constraints (WHERE, UNIQUE, INVERSE) are applied to each and every
instance of the entity.
\item Global constraints (RULEs) are applied between entities or across a
subset of entity instances.
\end{itemize}
The following rule states that there shall be one and only one point at the
origin in the objectbase.
\begin{verbatim}
RULE unique_origin FOR (point);
LOCAL
  origin : BAG OF point;
END_LOCAL;
  origin := QUERY(temp <* point |
                  (temp.x = 0.0) AND
                  (temp.y = 0.0) ); 
WHERE
  r1 : SIZEOF(origin) = 1;
END_RULE;
\end{verbatim}


\begin{remarks}
\remintro

\remtitle{RULE (cont)}

Creating a robust EXPRESS model is not necessarily easy.

Going back to the Person/male/female model it does say that wifes are
females and husbands are males. It doesn't say that if Adam claims his wife 
to be Eve then Eve's husband must be Adam.

In some communities that might not be a problem. But, if it is in the bit 
of the real world that the model represents then the rather complicated RULE 
fixes that relationship problem.

It looks at every male and checks to see if he is his wife's husband.
It also has to look at every female to see if she is her husband's
wife.

    The double check is needed for the cases when one of a pair claims 
to be single. 

    NOTE: EXPRESS does not specify when the RULEs should be checked. 

\remend
\end{remarks}

\vtoptitle{RULE (cont)}

This RULE states that husbands and wives must be married to each other.
\begin{verbatim}
RULE married FOR (male,female);
  LOCAL
    ok1, ok2 : BOOLEAN := TRUE;
  END_LOCAL;
  IF (EXISTS(male.wife) AND 
      male :<>: male.wife.husband) THEN
    ok1 := FALSE;
  END_IF;
  IF (EXISTS(female.husband) AND
      female :<>: female.husband.wife) THEN
    ok2 := FALSE;
  END_IF;
WHERE
  r1 : ok1;
  r2 : ok2;
END_RULE;
\end{verbatim}


\begin{remarks}
\remintro

\remtitle{SCHEMA Interfacing}

An EXPRESS model typically consists of several SCHEMAs, each
dealing with a distinguishable subtopic.

   Anything in a SCHEMA can be utilised by any other SCHEMA
--- you can't hide anything --- but you have to specify what
you want.

    The contents of a SCHEMA are ENTITY, TYPE, RULE, SUBTYPE\_CONSTRAINT,
FUNCTION, PROCEDURE and CONSTANT declarations, each of which has a name.

    Within a SCHEMA all the names must be unique.

    When importing something from another SCHEMA it may be necessary 
to rename it if its name is already declared, or it may convey the 
semantics better if it was called by a different name.

\remend
\end{remarks}

\vtitle{SCHEMA Interfacing}

\begin{itemize}
\item Definitions within a Schema are potentially available to all Schemas.
\item Definitions have to be `imported' from the original Schema into the
`current' Schema.
\item An imported definition implicitly imports all the necessary definitions
to complete the definition.
\end{itemize}

EXPRESS syntax is roughly
\begin{verbatim}
import FROM schema_ref (def1 AS newname1,
                        def2 AS newname2);
\end{verbatim}

\begin{remarks}
\remintro

\remtitle{USE Import}

Only ENTITYs and TYPEs can be USEd into a SCHEMA.

A USEd ENTITY is a \emph{first class} item. That means that in the object
base instances do not need to be referenced by other instances.

It is as though the ENTITY had been declared in the using schema. Following
from this, USEs can be chained.

Any items needed to complete the definitions of USEd items are
implicitly REFERENCEd into the schema.


\remend
\end{remarks}

\vtitle{USE Import}

\begin{itemize}
\item Only ENTITYs and TYPEs can be imported via a USE statement.
\item USEd ENTITYs are `first class' items (i.e they can be independently
      instantiated). 
\item The `stuff' required to complete the definitions of an imported item
      are implicitly REFERENCEd into the schema.
\item If no list is given, \emph{all} ENTITYs and TYPEs in the SCHEMA are
      imported.
\item USEs can be chained.
\end{itemize}

\textbf{NOTE:} If \verb|fc| is a first-class entity, then the statement
\begin{verbatim}
SIZEOF(USEROF(fc)) >= 0;
\end{verbatim}
holds.

\begin{remarks}
\remintro

\remtitle{USE (cont)}

Here is a demonstration 2 schema model where an enity declared
in one schema is USEd by the other.

 Following this is an equivalent model expanding out the USE.

\remend
\end{remarks}

\vtitle{USE (cont)}

\begin{verbatim}
SCHEMA source;
  ENTITY e1;
    attr : t1;
  END_ENTITY;

  TYPE t1 = REAL; END_TYPE;
END_SCHEMA;

SCHEMA using;
  USE FROM source (e1);

  ENTITY e2;
    attr : SET OF e1;
  END_ENTITY;
END_SCHEMA;
\end{verbatim}
gives effectively:

\begin{remarks}
\remintro

\remtitle{USE (cont)}

In the expanded model, SCHEMA \texttt{source} is unchanged.

SCHEMA \texttt{using} is changed with the USE being replaced by:
\begin{itemize}
\item ENTITY \texttt{e1} is declared 
\item TYPE \texttt{t1} is REFERENCED from SCHEMA \texttt{source}
      to provide for the \texttt{attr} attribute of \texttt{e1}
      (which was originally implicitly referenced).
\end{itemize}

\remend
\end{remarks}

\vtitle{USE (cont)}

\begin{verbatim}
SCHEMA source;
  ENTITY e1;
    attr : t1;
  END_ENTITY;

  TYPE t1 = REAL; END_TYPE;
END_SCHEMA;

SCHEMA using;
  REFERENCE FROM source (t1);

  ENTITY e1;
    attr : t1;
  END_ENTITY;

  ENTITY e2;
    attr : SET OF e1;
  END_ENTITY;
END_SCHEMA;
\end{verbatim}

\begin{remarks}
\remintro

\remtitle{RERERENCE Import}

Effectively, any kind of item can be REFERENCEd --- 
ENTITY, TYPE, FUNCTION \ldots

REFERENCEd ENTITYs are second class items (only instances that are used as
attribute(s) in other ENTITYs are allowed).

Items required to complete declarations are implicitly REFERENCEd, but
there is no chaining.

A REFERENCE with just the SCHEMA name references everything in the SCHEMA.

If an item is both USEd and REFERENCEd, it is treated as being USEd.

\remend
\end{remarks}

\vtoptitle{REFERENCE Import}

\begin{itemize}
\item Any kind of item can be imported via a REFERENCE statement.
\item A REFERENCE is necessary to resolve references (links) to declarations
      in other schemas.
\item REFERENCEDd items are `second class' items (i.e they can not be 
      independently instantiated).
\item The `stuff' required to complete the definitions of an imported entity
      are implicitly REFERENCEd into the schema.
\end{itemize}

\textbf{NOTE:} If \verb|sc| is a second-class entity, then the statement
\begin{verbatim}
SIZEOF(USEROF(sc)) >= 1;
\end{verbatim}
holds.

\begin{remarks}
\remintro

\remtitle{REFERENCE (cont)}

This model is the same as the earlier one except that USE is
replaced by REFERENCE.

An expanded version follows.


\remend
\end{remarks}

\vtitle{REFERENCE (cont)}

\begin{verbatim}
SCHEMA source;
  ENTITY e1;
    attr : t1;
  END_ENTITY;

  TYPE t1 = REAL; END_TYPE;
END_SCHEMA;

SCHEMA referencing;
  REFERENCE FROM source (e1);

  ENTITY e2;
    attr : SET OF e1;
  END_ENTITY;
END_SCHEMA;
\end{verbatim}
gives effectively:

\begin{remarks}
\remintro

\remtitle{REFERENCE (cont)}

In the expanded model, SCHEMA \texttt{source} is unchanged.

SCHEMA \texttt{using} is changed with the REFERENCE list expanded
to include the TYPE \texttt{t1} (which was originally implicitly
referenced).

\remend
\end{remarks}

\vtitle{REFERENCE (cont)}

\begin{verbatim}
SCHEMA source;
  ENTITY e1;
    attr : t1;
  END_ENTITY;

  TYPE t1 = REAL; END_TYPE;
END_SCHEMA;

SCHEMA referencing;
  REFERENCE FROM source (e1, t1);

  ENTITY e2;
    attr : SET OF e1;
  END_ENTITY;
END_SCHEMA;
\end{verbatim}


\begin{remarks}
\remintro

\remtitle{SCHEMA Interfacing}

A SCHEMA can extend and/or constrain a model in another SCHEMA.

In SCHEMA \texttt{second}, \texttt{bbb} (which is \texttt{aaa} under 
another name)and \texttt{constrained} are first class entities.
Entity \texttt{original}, which is now a SUPERTYPE of \texttt{constrained},
is second class (every instance of \texttt{original} must also be an
instance of \texttt{constrained}).

Within SCHEMA \texttt{first}, entity \texttt{original} does not know it is
a SUPERTYPE as \texttt{first} knows nothing about the \texttt{second} SCHEMA.

\remend
\end{remarks}

\vtoptitle{SCHEMA Interfacing}

\begin{verbatim}
SCHEMA first;
  ENTITY aaa;
    -- attributes
  END_ENTITY;

  ENTITY original;
    attr : NUMBER;
  END_ENTITY;
END_SCHEMA; -- first

SCHEMA second;
  USE FROM first (aaa AS bbb);
  REFERENCE FROM first (original);

  ENTITY constrained
    SUBTYPE OF (original);
    attr : INTEGER(7);
    WHERE
      positive : attr > 0;
    END_ENTITY;
END_SCHEMA; -- second
\end{verbatim}

%\vtitle{EXPRESS-G Example Model}
%
%\input{[pdes.pdes01.step]gfigfull1}

\begin{remarks}
\remintro

\remtitle{EXPRESS Summary}

It's a great family of languages.

\remend
\end{remarks}

\vtitle{EXPRESS Summary}

\begin{itemize}
\item A powerful OO information modeling language
  \begin{itemize}
  \item Primary form is a computer processible text language.
  \item EXPRESS-G as a graphical subset.
  \item EXPRESS-I as an instantiation form 
  \item EXPRESS-X transformation specification
  \end{itemize}
\item Is an ISO standard language.
\item Normative STEP information models.
\item Becoming widely used in the modeling communities.
\item Software tools available.
\end{itemize}

\endinput


