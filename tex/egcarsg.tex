% egcarsg.tex   % Example EXPRESS-G car registration model
%               % based on Ebook/Text/acarsg.tex
% Craeted by Peter R Wilson, 1992 -- 2004

%\documentclass[article]{memoir}
\documentclass{article}
\usepackage{ifpdf}

\ifpdf
  \pdfoutput=1
\fi

\usepackage{egs}
\usepackage{xkw}

%\maxsecnumdepth{subsubsection}
%\maxtocdepth{subsubsection}

\title{EXPRESS-G example model \\ Car Registration Authority}
\author{Peter Wilson}
\date{}

%\renewcommand{\thesection}{\arabic{section}}
\raggedbottom
\begin{document}

\maketitle

\tableofcontents
\listoffigures
\clearpage

% \chapter{Model Example in EXPRESS-G}  \label{anx:gmodel}

\makeatletter \@topnum\z@ \makeatother
\section{Introduction}

This document contains a complete \ExpressG\ representation of the 
Car Registration Authority example model. There are details which cannot
be expressed using the \ExpressG\ iconic language but which can be described
using the \Express\ lexical language.

    An \Express\ model is a primary representation and an \ExpressG\ model
is a secondary representation.

\section{Model overview}

The model consists of three schemas, as shown in Figure~\ref{fig:cargschema}.

%\input{(pdes.pdes01.ebook.bkfig)cargschema}
% cargschema.tex  BOOK  EXPRESS-G car schemas  fig:cargschema
\begin{figure}[htp]
\center
\setlength{\unitlength}{1mm}
\begin{picture}(107,55)
%\begin{picture}(110,160)\input{gridmm}
\thicklines

\put(40,40){\begin{picture}(40,12)
  \put(0,0){\framebox(40,12){}}
  \put(0,6){\framebox(40,6){registration\_authority}}
  \end{picture}}

\put(15,0){\begin{picture}(30,12)
  \put(0,0){\framebox(30,12){}}
  \put(0,6){\framebox(30,6){auxiliary}}
  \end{picture}}

\put(75,0){\begin{picture}(30,12)
  \put(0,0){\framebox(30,12){}}
  \put(0,6){\framebox(30,6){calendar}}
  \end{picture}}

% auxiliary/registration
\put(30,12){\begin{picture}(10,40)
  \put(0,1){\circle{2}}
  \multiput(0,2)(0,4){8}{\line(0,1){2}}
  \multiput(0,32)(4,0){3}{\line(1,0){2}}
  \end{picture}}
\put(0,22){\begin{picture}(20,12)
  \put(0,0){fuel\_consumption}
  \put(0,3){manufacturer}
  \put(0,6){transfer}
  \put(0,9){car}
  \end{picture}}
\put(0,20){\vector(1,0){30}}

% registration/calendar
\put(80,12){\begin{picture}(10,40)
  \multiput(0,32)(4,0){3}{\line(1,0){2}}
  \multiput(10,2)(0,4){8}{\line(0,1){2}}
  \put(10,1){\circle{2}}
  \end{picture}}

% auxiliary/calendar
\put(45,6){\begin{picture}(30,10)
  \multiput(0,0)(4,0){7}{\line(1,0){2}}
  \put(29,0){\circle{2}}
  \end{picture}}
\put(55,20){\vector(0,-1){14}}
\put(57,15){months}
\put(57,18){date}

\end{picture}
\setlength{\unitlength}{1pt}
\caption{Complete schema-level model for Registration Authority example
         (Page 1 of 1).}
\label{fig:cargschema}
\end{figure}

The schema \nexp{registration\_authority} is the primary schema. It references
the noted items from the \nexp{auxiliary} schema; some functions, rules or
procedures may also be referenced from this schema as well, but the diagram
cannot show these. This schema also references one or more items from the
\nexp{calendar} schema. These may be rules, functions or procedures but again
the diagram cannot show these; alternatively, the entire schema may be
referenced.

The \nexp{auxiliary} schema references at least two items from the
\nexp{calendar} schema. Rules, functions and procedures may also be
referenced, but this cannot be discerned via \ExpressG.

The \nexp{calendar} schema is self-contained as it neither uses nor references
any other schema.

An entity-level model for the \nexp{registration\_authority} schema is given
in Figure~\ref{fig:cargreg}. The entity-level model for the \nexp{auxiliary}
schema is displayed in Figures~\ref{fig:cargaux1} and~\ref{fig:cargaux2}. The
internals of the \nexp{calendar} schema are shown in Figure~\ref{fig:cargcal}.

\clearpage

%\chapter{The registration authority schema}
\section{The registration authority schema}

Figure~\ref{fig:cargreg} shows the \nexp{registration\_authority} schema and 
the entities and types in it.

%\input{(pdes.pdes01.ebook.bkfig)cargreg}
% cargreg.tex  BOOK  EXPRESS-G of Reg_auth schema  fig:cargreg
\begin{figure}[htp]
\center
\setlength{\unitlength}{1mm}
\begin{picture}(110,110)
%\begin{picture}(110,160)\input{gridmm}
\thicklines

%  history
\put(5,80){\begin{picture}(100,30)

  \put(0,14){\begin{picture}(20,10)
    \put(0,0){\framebox(20,10){*history}}
    \end{picture}}

  \put(70,20){\begin{picture}(30,8)
    \put(0,0){\dashbox{2}(30,8){auxiliary.car}}
    \put(15,4){\oval(30,4)}
    \end{picture}}

  \put(70,10){\begin{picture}(30,8)
    \put(0,0){\dashbox{2}(30,8){auxiliary.transfer}}
    \put(15,4){\oval(30,4)}
    \end{picture}}

  \put(76,0){\begin{picture}(24,4)
    \put(0,0){\framebox(24,4){BOOLEAN}}
    \put(22,0){\line(0,1){4}}
    \end{picture}}

  % history/car and transfer
  \put(20,14){\begin{picture}(50,10)
    \multiput(0,2.5)(0,5){2}{\line(1,0){48}}
    \multiput(49,2.5)(0,5){2}{\circle{2}}
    \put(25,8.5){\makebox(0,0)[b]{*item}}
    \put(25,1.5){\makebox(0,0)[t]{*transfers L(0:?)}}
    \end{picture}}

  % history/boolean
  \put(0,0){\begin{picture}(76,14)
    \put(10,2){\line(0,1){12}}
    \put(10,2){\line(1,0){64}}
    \put(75,2){\circle{2}}
    \put(48,3){\makebox(0,0)[b]{(DER) to\_be\_deleted}}
    \end{picture}}

  \end{picture}}  % end history

% send message
\put(0,0){\begin{picture}(110,100)

  \put(0,19){\begin{picture}(30,10)
    \put(0,0){\framebox(30,10){send\_message}}
    \end{picture}}

  \put(65,60){\begin{picture}(40,8)
    \put(0,0){\dashbox{2}(40,8){auxiliary.manufacturer}}
    \put(20,4){\oval(40,4)}
    \end{picture}}

  \put(60,40){\begin{picture}(50,10)
    \put(0,0){\framebox(50,10){*authorized\_manufacturer}}
    \end{picture}}

  \put(86,22){\begin{picture}(24,4)
    \put(0,0){\framebox(24,4){INTEGER}}
    \put(22,0){\line(0,1){4}}
    \end{picture}}

  \put(60,0){\begin{picture}(50,8)
    \put(0,0){\dashbox{2}(50,8){auxiliary.fuel\_consumption}}
    \put(25,4){\oval(50,4)}
    \end{picture}}

  % message/integer
  \put(30,19){\begin{picture}(56,10)
    \put(0,5){\line(1,0){54}}
    \put(55,5){\circle{2}}
    \put(28,6){\makebox(0,0)[b]{year}}
    \end{picture}}

  % message/consumption
  \put(0,0){\begin{picture}(60,19)
    \put(15,4){\line(0,1){15}}
    \put(15,4){\line(1,0){43}}
    \put(59,4){\circle{2}}
    \put(37.5,5){\makebox(0,0)[b]{max\_consumption}}
    \end{picture}}

  % message/manufacturer
  \put(0,29){\begin{picture}(60,20)
    \put(20,16){\line(0,-1){16}}
    \put(20,16){\line(1,0){38}}
    \put(59,16){\circle{2}}
    \put(40,17){\makebox(0,0)[b]{makers}}
    \end{picture}}

  % mnfs/auth
  \put(60,50){\begin{picture}(50,10)
    \put(25,1){\circle{2}}
    \multiput(24.75,2)(0.25,0){3}{\line(0,1){8}}
    \end{picture}}

  % message/mnfs
  \put(0,29){\begin{picture}(60,40)
    \put(10,35){\line(0,-1){35}}
    \put(10,35){\line(1,0){53}}
    \put(64,35){\circle{2}}
    \put(37.5,36){\makebox(0,0)[b]{(DER) excessives}}
    \end{picture}}

  \end{picture}}  % end send message

\end{picture}
\setlength{\unitlength}{1pt}
\caption{Complete entity-level model of the Registration Authority schema
         (Page 1 of 1).}
\label{fig:cargreg}
\end{figure}

There are three entities altogether, namely \nexp{send\_message},
\nexp{history} and \nexp{authorized\_manufacturer}. There are constraints on
the two latter entities. The entity \nexp{authorized\_manufacturer} is a
subtype of the entity \nexp{manufacturer} which is defined in the
\nexp{auxiliary} schema.

There are no types defined within the schema.

\clearpage

%\chapter{The auxiliary schema}
\section{The auxiliary schema}

The nexp{auxiliary} schema contains too many items to be displayed on a single
page. Figures~\ref{fig:cargaux1} and~\ref{fig:cargaux2} show the entities and
types that are contained in the schema.

%\input{(pdes.pdes01.ebook.bkfig)cargaux1}
% cargaux1.tex  BOOK  EXPRESS-G of Auxiliary (1) schema  fig:cargaux1
\begin{figure}[htp]
\center
\setlength{\unitlength}{1mm}
\begin{picture}(110,110)
%\begin{picture}(110,160)\input{gridmm}
\thicklines

% ident #
\put(10,0){\begin{picture}(90,4)

  \put(0,0){\begin{picture}(15,4)
    \put(7.5,2){\oval(15,4)}
    \put(7.5,2){\makebox(0,0){1,3 (2)}}
    \end{picture}}

  \put(25,0){\dashbox{1}(30,4){identification\_no}}

  \put(65,0){\begin{picture}(24,4)
    \put(0,0){\framebox(24,4){STRING}}
    \put(22,0){\line(0,1){4}}
    \end{picture}}

  % ref/id #
  \put(15,2){\line(1,0){8}}
  \put(24,2){\circle{2}}

  % no/string
  \put(55,2){\line(1,0){8}}
  \put(64,2){\circle{2}}

  \end{picture}} % end ident #

%  car model
\put(0,20){\begin{picture}(110,45)

  \put(50,0){\begin{picture}(30,4)
    \put(15,2){\oval(30,4)}
    \put(15,2){\makebox(0,0){2,2 manufacturer}}
    \end{picture}}

  \put(0,10){\framebox(20,10){car\_model}}

  \put(50,10){\dashbox{1}(30,4){*fuel\_consumption}}

  \put(86,10){\begin{picture}(24,4)
    \put(0,0){\framebox(24,4){REAL}}
    \put(22,0){\line(0,1){4}}
    \end{picture}}

  \put(70,16){\dashbox{1}(10,4){name}}

  \put(86,16){\begin{picture}(24,4)
    \put(0,0){\framebox(24,4){STRING}}
    \put(22,0){\line(0,1){4}}
    \end{picture}}

  % ref into model
  \put(2.5,30){\begin{picture}(15,4)
    \put(7.5,2){\oval(15,4)}
    \put(7.5,2){\makebox(0,0){1,1 (2)}}
    \put(7.5,0){\line(0,-1){8}}
    \put(7.5,-9){\circle{2}}
    \end{picture}}

  % ref into name
  \put(67.5,30){\begin{picture}(15,4)
    \put(7.5,2){\oval(15,4)}
    \put(7.5,2){\makebox(0,0){1,2 (2)}}
    \put(7.5,0){\line(0,-1){8}}
    \put(7.5,-9){\circle{2}}
    \end{picture}}

  % model/mnfr
  \put(0,0){\begin{picture}(50,10)
    \put(10,2){\line(0,1){8}}
    \put(10,2){\line(1,0){40}}
    \put(30,3){\makebox(0,0)[b]{made\_by}}
    \end{picture}}

  % model/fuel/real
  \put(20,10){\begin{picture}(70,10)
    \put(0,2){\line(1,0){28}}
    \put(29,2){\circle{2}}
    \put(15,1){\makebox(0,0)[t]{consumption}}
    \put(60,2){\line(1,0){4}}
    \put(65,2){\circle{2}}
    \end{picture}}

  % model/name/string
  \put(20,10){\begin{picture}(70,10)
    \put(0,8){\line(1,0){48}}
    \put(49,8){\circle{2}}
    \put(15,9){\makebox(0,0)[b]{*called}}
    \put(60,8){\line(1,0){4}}
    \put(65,8){\circle{2}}
    \end{picture}}

  \end{picture}} % end car model

% transfer
\put(5,80){\begin{picture}(100,30)

  \put(70,0){\begin{picture}(30,8)
    \put(0,0){\dashbox{2}(30,8){calendar.date}}
    \put(15,4){\oval(30,4)}
    \end{picture}}

  \put(0,10){\framebox(20,10){transfer}}

  \put(70,13){\begin{picture}(20,4)
    \put(10,2){\oval(20,4)}
    \put(10,2){\makebox(0,0){2,1 owner}}
    \end{picture}}

  \put(70,23){\begin{picture}(20,4)
    \put(10,2){\oval(20,4)}
    \put(10,2){\makebox(0,0){2,3 car}}
    \end{picture}}

  % transfer/date
  \put(0,0){\begin{picture}(70,10)
    \put(10,4){\line(0,1){6}}
    \put(10,4){\line(1,0){60}}
    \put(40,3){\makebox(0,0)[t]{on}}
    \end{picture}}

  % transfer/owner
  \put(20,10){\begin{picture}(50,10)
    \multiput(0,2)(0,6){2}{\line(1,0){40}}
    \put(40,2){\line(0,1){6}}
    \put(40,5){\line(1,0){10}}
    \put(20,1){\makebox(0,0)[t]{*prior}}
    \put(20,9){\makebox(0,0)[b]{*new}}
    \end{picture}}

  % transfer/car
  \put(0,20){\begin{picture}(70,10)
    \put(10,5){\line(0,-1){5}}
    \put(10,5){\line(1,0){60}}
    \put(40,6){\makebox(0,0)[b]{*item}}
    \end{picture}}

  \end{picture}}  % end transfer

\end{picture}
\setlength{\unitlength}{1pt}
\caption{Complete entity-level model of the Auxiliary schema
         (Page 1 of 2).}
\label{fig:cargaux1}
\end{figure}

%\input{(pdes.pdes01.ebook.bkfig)cargaux2}
% cargaux2.tex  BOOK  EXPRESS-G of Auxiliary (2) schema  fig:cargaux2
\begin{figure}[htp]
\center
\setlength{\unitlength}{1mm}
\begin{picture}(110,160)
%\begin{picture}(110,160)\input{gridmm}
\thicklines

% Owner tree
\put(0,0){\begin{picture}(110,100)

  % Integer/garage
  \put(38,0){\begin{picture}(24,15)
    \put(0,0){\framebox(24,4){INTEGER}}
    \put(22,0){\line(0,1){4}}
    \put(12,6){\line(0,1){9}}
    \put(12,5){\circle{2}}
    \put(13,11){(DER)}
    \put(13,8){no\_of\_mnfs}
    \end{picture}}

  % ref into mnf
  \put(7.5,0){\begin{picture}(15,15)
    \put(7.5,2){\oval(15,4)}
    \put(7.5,2){\makebox(0,0){2,2 (1)}}
    \put(7.5,4){\line(0,1){11}}
    \end{picture}}

  \put(0,15){\framebox(30,10){manufacturer}}

  \put(40,15){\framebox(20,10){garage}}

  \put(70,15){\framebox(20,10){person}}

  \put(25,40){\framebox(40,10){(ABS) named\_owner}}

  \put(30,65){\framebox(30,10){(ABS) owner}}

  \put(80,55){\framebox(20,10){group}}

  % owner/name ref
  \put(65,40){\begin{picture}(40,10)
    \put(15,3){\begin{picture}(20,4)
      \put(10,2){\oval(20,4)}
      \put(10,2){\makebox(0,0){1,2 name}}
      \end{picture}}
    \put(0,5){\line(1,0){15}}
    \put(7.5,6){\makebox(0,0)[b]{*called}}
    \end{picture}}

  % named subtypes connections
  \put(20,25){\begin{picture}(70,15)
    \multiput(0,0)(30,0){3}{\begin{picture}(1,10)
      \put(0,1){\circle{2}}
      \multiput(-0.25,2)(0.25,0){3}{\line(0,1){8}}
      \end{picture}}
    \multiput(0,9.75)(0,0.25){3}{\line(1,0){60}}
    \multiput(24.75,10)(0.25,0){3}{\line(0,1){5}}
    \put(26,11){1}
    \end{picture}}

  % owner/named/group subtype connections
  \put(45,50){\begin{picture}(40,15)
    \put(0,1){\circle{2}}
    \multiput(-0.25,2)(0.25,0){3}{\line(0,1){13}}
    \multiput(0,9.75)(0,0.25){3}{\line(1,0){33}}
    \put(34,10){\circle{2}}
    \put(1,11){1}
    \end{picture}}

  % group/person connection
  \put(90,15){\begin{picture}(15,50)
    \put(1,5){\circle{2}}
    \put(15,5){\line(-1,0){13}}
    \put(15,5){\line(0,1){40}}
    \put(15,45){\line(-1,0){5}}
    \put(14,22){\makebox(0,0)[br]{members}}
    \put(14,20){\makebox(0,0)[tr]{S(1:?)}}

    \end{picture}}

%    \multiput(24.75,2)(0.25,0){3}{\line(0,1){8}}

  \end{picture}} % end owner tree

%  car tree
  \put(0,90){\begin{picture}(70,110)

    \put(0,10){\framebox(30,28){car}}

    % destroyed car and sub
    \put(0,60){\begin{picture}(30,10)
      \put(0,0){\framebox(30,10){destroyed\_car}}
      \multiput(19.75,-2)(0.25,0){3}{\line(0,-1){20}}
      \put(20,-1){\circle{2}}
      \end{picture}}

    \put(70,61){\begin{picture}(30,8)
      \put(0,0){\dashbox{2}(30,8){calendar.date}}
      \put(15,4){\oval(30,4)}
      \end{picture}}

    \put(80,0){\begin{picture}(30,4)
      \put(15,2){\oval(30,4)}
      \put(15,2){\makebox(0,0){1,1 car\_model}}
      \end{picture}}

    \put(70,16){\begin{picture}(40,4)
      \put(20,2){\oval(40,4)}
      \put(20,2){\makebox(0,0){1,3 identification\_no}}
      \end{picture}}

    \put(86,28){\begin{picture}(24,4)
      \put(0,0){\framebox(24,4){INTEGER}}
      \put(22,0){\line(0,1){4}}
      \end{picture}}

    % ref into car
    \put(0,38){\begin{picture}(15,15)
      \put(7.5,12){\oval(15,4)}
      \put(7.5,12){\makebox(0,0){2,3 (1)}}
      \put(7.5,10){\line(0,-1){10}}
      \end{picture}}

    % car/model
    \put(0,0){\begin{picture}(80,10)
      \put(20,2){\line(0,1){8}}
      \put(20,2){\line(1,0){60}}
      \put(50,3){\makebox(0,0)[b]{model\_type}}
      \end{picture}}

    % car/id
    \put(30,10){\begin{picture}(60,20)
      \multiput(0,2)(0,12){2}{\line(1,0){60}}
      \multiput(60,2)(0,8){2}{\line(0,1){4}}
      \put(20,3){\makebox(0,0)[b]{*registration\_no}}
      \put(20,13){\makebox(0,0)[t]{*mnfg\_no}}
      \end{picture}}

    % car/integer
    \put(30,10){\begin{picture}(60,28)
      \put(0,20){\line(1,0){54}}
      \put(55,20){\circle{2}}
      \put(20,21){\makebox(0,0)[b]{*production\_year}}
      \end{picture}}

    % destroyed/date
    \put(30,60){\begin{picture}(40,10)
      \put(0,5){\line(1,0){40}}
      \put(20,6){\makebox(0,0)[b]{*destroyed\_on}}
      \end{picture}}

    % car/date
    \put(30,10){\begin{picture}(50,60)
      \put(0,26){\line(1,0){50}}
      \put(50,26){\line(0,1){25}}
      \put(20,27){\makebox(0,0)[b]{*production\_date}}
      \end{picture}}

    \end{picture}} % end car tree

  % car/manufacturer
  \put(0,25){\begin{picture}(30,75)
     \put(10,1){\circle{2}}
     \put(10,2){\line(0,1){73}}
     \put(11,60){(DER)}
     \put(11,57){*made\_by}
     \end{picture}}

\end{picture}
\setlength{\unitlength}{1pt}
\caption{Complete entity-level model of the Auxiliary schema
         (Page 2 of 2).}
\label{fig:cargaux2}
\end{figure}

The schema contains the following entities:

\begin{itemize}

\item \nexp{car}

\item \nexp{car\_model}

\item \nexp{destroyed\_car}, which is a subtype of \nexp{car}

\item \nexp{garage}, which is a subtype of \nexp{named\_owner}

\item \nexp{group}, which is a subtype of \nexp{owner}

\item \nexp{manufacturer}, which is a subtype of \nexp{named\_owner}

\item \nexp{named\_owner}, which is an abstract supertype and also a subtype
of \nexp{owner}. It puts a \zoneof\ constraint on its subtypes.

\item \nexp{owner}, which is an abstract supertype and puts a \zoneof\
constraint on its subtypes.

\item \nexp{person}, which is a subtype of \nexp{named\_owner}

\item \nexp{transfer}

\end{itemize}

It also contains the following types:

\begin{itemize}

\item \nexp{identification\_no}, a defined data type with a base-type of
\zstring\

\item \nexp{fuel\_consumption}, a defined data type with a base-type of
\zreal\

\item \nexp{name}, a defined data type with a base-type of \zstring\

\end{itemize}

\clearpage
%\chapter{The calendar schema}
\section{The calendar schema}

Figure~\ref{fig:cargcal} shows the entities and types within the
\nexp{calendar} schema.

%\input{(pdes.pdes01.ebook.bkfig)cargcal}
% cargcal.tex  BOOK EXPRESS-G calendar schema  fig:cargcal
\begin{figure}[htp]
\center
\setlength{\unitlength}{1mm}
\begin{picture}(85,22)
%\begin{picture}(110,160)\input{gridmm}
\thicklines

\put(0,10){\begin{picture}(20,10)
  \put(0,0){\framebox(20,10){*date}}
  \end{picture}}

\put(60,13){\begin{picture}(24,4)
  \put(0,0){\framebox(24,4){INTEGER}}
  \put(22,0){\line(0,1){4}}
  \end{picture}}

\put(60,0){\begin{picture}(24,4)
  \put(0,0){\dashbox{1}(24,4){months}}
  \multiput(22,0.5)(0,2){2}{\line(0,1){1}}
  \end{picture}}

% date/months
\put(10,0){\begin{picture}(50,10)
  \put(0,2){\line(0,1){8}}
  \put(0,2){\line(1,0){48}}
  \put(49,2){\circle{2}}
  \put(25,3){\makebox(0,0)[b]{month}}
  \end{picture}}

% date/integer
\put(20,10){\begin{picture}(40,12)
  \multiput(0,2.5)(0,5){2}{\line(1,0){30}}
  \put(30,2.5){\line(0,1){5}}
  \put(30,5){\line(1,0){8}}
  \put(39,5){\circle{2}}
  \put(15,1.5){\makebox(0,0)[t]{*year}}
  \put(15,8.5){\makebox(0,0)[b]{*day}}
  \end{picture}}

\end{picture}
\setlength{\unitlength}{1pt}
\caption{Complete entity-level model of Calendar schema (Page 1 of 1).}
\label{fig:cargcal}
\end{figure}

There is one entity called \nexp{date} and one enumeration type called 
\nexp{months}. 

\end{document}

