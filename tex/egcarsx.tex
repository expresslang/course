% egcarsx.tex   % Example EXPRESS(-G) car registration model
%               % based on Ebook/Text/acarsx.tex
% Created by Peter R Wilson, 1992 -- 2004

%\documentclass[article]{memoir}
\documentclass{article}
\usepackage{ifpdf}

\ifpdf
  \pdfoutput=1
\fi

\usepackage{egs}

%\maxsecnumdepth{subsubsection}
%\maxtocdepth{subsubsection}

\title{EXPRESS example model \\ Car Registration Authority}
\author{Peter Wilson}
\date{}

%\renewcommand{\thesection}{\arabic{section}}
\providecommand*{\fref}[1]{Figure~\ref{#1}}

\raggedbottom
\begin{document}

\maketitle

\tableofcontents
\listoffigures
\clearpage


%\section{EXPRESS example model}  \label{anx:emodel}

\makeatletter \@topnum\z@ \makeatother
\section{Introduction}

 This document contains a complete and documented \Express\ model for the
Car Registration Authority example; an \ExpressG\ version of the model is also 
included. 

\section{Scope}

The model has to do with the registration of cars and is limited to the scope
of interest of the Registration Authority. This Authority exists for the
purpose of:

\begin{itemize}

\item Knowing who is or was the registered owner of a car at any time from
construction to destruction of the car;

\item To monitor laws regarding the transfer of ownership of cars;

\item To monitor laws regarding the fuel consumption of cars;

\item To monitor laws regarding manufacturers of cars.

\end{itemize}

\section{Model overview}

    The model is described using both \Express\ and \ExpressG. The \Express\ 
definitions are primary and the \ExpressG\ diagrams are to assist in 
understanding the primary model. If there is any conflict between the 
\Express\ and \ExpressG, then the \Express\ takes precedence.

%\input{/rdrc/design/pwilson/Pbooks/Ebook/Bkfigs/cargschema}
%%%%%%%%%%%%
% cargschema.tex  BOOK  EXPRESS-G car schemas  fig:cargschema
\begin{figure}[tbp]
\center
\setlength{\unitlength}{1mm}
\begin{picture}(107,55)
%\begin{picture}(110,160)\input{gridmm}
\thicklines

\put(40,40){\begin{picture}(40,12)
  \put(0,0){\framebox(40,12){}}
  \put(0,6){\framebox(40,6){authority}}
  \end{picture}}

\put(15,0){\begin{picture}(30,12)
  \put(0,0){\framebox(30,12){}}
  \put(0,6){\framebox(30,6){support}}
  \end{picture}}

\put(75,0){\begin{picture}(30,12)
  \put(0,0){\framebox(30,12){}}
  \put(0,6){\framebox(30,6){calender}}
  \end{picture}}


% auxiliary/registration
\put(30,12){\begin{picture}(10,40)
  \put(0,1){\circle{2}}
  \multiput(0,2)(0,4){8}{\line(0,1){2}}
  \multiput(0,32)(4,0){3}{\line(1,0){2}}
  \end{picture}}
\put(0,22){\begin{picture}(20,12)
  \put(0,0){fuel\_consumption}
  \put(0,3){manufacturer}
  \put(0,6){transfer}
  \put(0,9){car}
  \end{picture}}
\put(0,20){\vector(1,0){30}}

% registration/calender
\put(80,12){\begin{picture}(10,40)
  \multiput(0,32)(4,0){3}{\line(1,0){2}}
  \multiput(10,2)(0,4){8}{\line(0,1){2}}
  \put(10,1){\circle{2}}
  \end{picture}}

% auxiliary/calender
\put(45,6){\begin{picture}(30,10)
  \multiput(0,0)(4,0){7}{\line(1,0){2}}
  \put(29,0){\circle{2}}
  \end{picture}}
\put(55,20){\vector(0,-1){14}}
\put(57,15){months}
\put(57,18){date}

\end{picture}
\setlength{\unitlength}{1pt}
\caption{Complete schema-level model for Registration Authority example
         (Page 1 of 1).}
\label{fig:cargschema}
\end{figure}



    The model consists of three schemas, as shown in \fref{fig:cargschema}. 
The schema \nexp{authority} is the primary schema. It references items from 
the two ancilliary schemas, namely \nexp{support} and \nexp{calendar}. 
The \nexp{support} schema also references items from the \nexp{calendar} 
schema.

\section{Authority schema}

\begin{Mnamedesc}{authority}

\begin{Mdesctext}

This schema is the primary one in the model and is principally concerned
with the main functions of the Registration Authority.

The schema imports definitions from two sources, namely the \nexp{support} 
and the \nexp{calendar} schemas. 

     Figure~\ref{fig:cargreg} is an \ExpressG\ complete entity-level model
 for this schema.

 \end{Mdesctext}

 \begin{Mexp}
 \begin{verbatim}
 *)
 SCHEMA authority;
   REFERENCE FROM support (car,
			   transfer,
			   manufacturer,
			   fuel_consumption,
			   mnfg_average_consumption);
   REFERENCE FROM calendar (current_date);
 (*
 \end{verbatim}
 \end{Mexp}
 \end{Mnamedesc}

 \subsection{Entity definitions}

% \subsubsection{Entity history}
 \subsubsection{history}

 \begin{Mnamedesc}{history}

 \begin{Mdesctext}

 A \nexp{history} records the transfers of ownership of a \nexp{car} over its
 lifetime. A \nexp{history} must be kept for a certain period after the
 \nexp{car} is destroyed, after which the ownership records may be destroyed.

 \end{Mdesctext}


%\input{/rdrc/design/pwilson/Pbooks/Ebook/Bkfigs/cargreg}
%%%%%%%%%%%%%%%%%%%%%%%
 % cargreg.tex  BOOK  EXPRESS-G of Reg_auth schema  fig:cargreg
 \begin{figure}[tbp]
 \center
 \setlength{\unitlength}{1mm}
 \begin{picture}(110,110)
 %\begin{picture}(110,160)\input{gridmm}
 \thicklines

 %  history
 \put(5,80){\begin{picture}(100,30)

   \put(0,14){\begin{picture}(20,10)
     \put(0,0){\framebox(20,10){*history}}
     \end{picture}}

   \put(70,20){\begin{picture}(30,8)
     \put(0,0){\dashbox{2}(30,8){support.car}}
     \put(15,4){\oval(30,4)}
     \end{picture}}

   \put(70,10){\begin{picture}(30,8)
     \put(0,0){\dashbox{2}(30,8){support.transfer}}
     \put(15,4){\oval(30,4)}
     \end{picture}}

   \put(76,0){\begin{picture}(24,4)
     \put(0,0){\framebox(24,4){BOOLEAN}}
     \put(22,0){\line(0,1){4}}
     \end{picture}}

   % history/car and transfer
   \put(20,14){\begin{picture}(50,10)
     \multiput(0,2.5)(0,5){2}{\line(1,0){48}}
     \multiput(49,2.5)(0,5){2}{\circle{2}}
     \put(25,8.5){\makebox(0,0)[b]{*item}}
     \put(25,1.5){\makebox(0,0)[t]{*transfers L[0:?]}}
     \end{picture}}

   % history/boolean
   \put(0,0){\begin{picture}(76,14)
     \put(10,2){\line(0,1){12}}
     \put(10,2){\line(1,0){64}}
     \put(75,2){\circle{2}}
     \put(48,3){\makebox(0,0)[b]{(DER) to\_be\_deleted}}
     \end{picture}}

   \end{picture}}  % end history

 % send message
 \put(0,0){\begin{picture}(110,100)

   \put(0,19){\begin{picture}(30,10)
     \put(0,0){\framebox(30,10){send\_message}}
     \end{picture}}

   \put(65,60){\begin{picture}(40,8)
     \put(0,0){\dashbox{2}(40,8){support.manufacturer}}
     \put(20,4){\oval(40,4)}
     \end{picture}}

   \put(60,40){\begin{picture}(50,10)
     \put(0,0){\framebox(50,10){*authorized\_manufacturer}}
     \end{picture}}

   \put(86,22){\begin{picture}(24,4)
     \put(0,0){\framebox(24,4){INTEGER}}
     \put(22,0){\line(0,1){4}}
     \end{picture}}

   \put(60,0){\begin{picture}(50,8)
     \put(0,0){\dashbox{2}(50,8){support.fuel\_consumption}}
     \put(25,4){\oval(50,4)}
     \end{picture}}

   % message/integer
   \put(30,19){\begin{picture}(56,10)
     \put(0,5){\line(1,0){54}}
     \put(55,5){\circle{2}}
     \put(28,6){\makebox(0,0)[b]{year}}
     \end{picture}}

   % message/consumption
   \put(0,0){\begin{picture}(60,19)
     \put(15,4){\line(0,1){15}}
     \put(15,4){\line(1,0){43}}
     \put(59,4){\circle{2}}
     \put(37.5,5){\makebox(0,0)[b]{max\_consumption}}
     \end{picture}}

   % message/manufacturer
   \put(0,29){\begin{picture}(60,20)
     \put(20,16){\line(0,-1){16}}
     \put(20,16){\line(1,0){38}}
     \put(59,16){\circle{2}}
     \put(40,17){\makebox(0,0)[b]{makers S[0:?]}}
     \end{picture}}

   % mnfs/auth
   \put(60,50){\begin{picture}(50,10)
     \put(25,1){\circle{2}}
     \multiput(24.75,2)(0.25,0){3}{\line(0,1){8}}
     \end{picture}}

   % message/mnfs
   \put(0,29){\begin{picture}(60,40)
     \put(10,35){\line(0,-1){35}}
     \put(10,35){\line(1,0){53}}
     \put(64,35){\circle{2}}
     \put(37.5,36){\makebox(0,0)[b]{(DER) excessives S[0:?]}}
     \end{picture}}

   \end{picture}}  % end send message

 \end{picture}
 \setlength{\unitlength}{1pt}
 \caption{Complete entity-level model of the Authority schema (Page 1 of 1).}
 \label{fig:cargreg}
 \end{figure}


 \begin{Mexp}
 \begin{verbatim}
 *)
 ENTITY history;
   item : car;
   transfers : LIST [0:?] OF UNIQUE transfer;
 DERIVE
   to_be_deleted : BOOLEAN := too_old(SELF);
 UNIQUE
   un1 : item;
 WHERE
   one_car  : single_car(SELF);
   ordering : exchange_ok(transfers);
 END_ENTITY;
 (*
 \end{verbatim}
 \end{Mexp}

 \begin{Matts}

 \item[item:] The \nexp{car} whose ownership history is being tracked.

 \item[transfers:] The ownership \nexp{transfer} records of the \nexp{item}.

 \item[to\_be\_deleted:] A flag which indicates that this \nexp{history} record
 may be deleted because the \nexp{item} has been destroyed (TRUE), or that the
 record shall not be deleted (FALSE).

 \end{Matts}

 \begin{Mprops}

 \item[un1:] The value of \nexp{item} shall be unique across all instances of
 \nexp{history}.

 \item[one\_car:] Each \nexp{transfer} collected in a \nexp{history} shall be
 of the same \nexp{car}.

 \item[ordering:] The list of \nexp{transfer} shall be in increasing historical
 order.

 \end{Mprops}

 \end{Mnamedesc}

% \subsubsection{Entity authorized\_manufacturer}
 \subsubsection{authorized\_manufacturer}

 \begin{Mnamedesc}{authorized_manufacturer}

 \begin{Mdesctext}

 An \nexp{authorized manufacturer} is a \nexp{manufacturer} who has been given
 permission by the Registration Authority to make cars.

 \end{Mdesctext}

 %\Mexp
 \begin{Mexp}
 \begin{verbatim}
 *)
 ENTITY authorized_manufacturer
   SUBTYPE OF (manufacturer);
 END_ENTITY;
 (*
 \end{verbatim}
 \end{Mexp}
 \end{Mnamedesc}


% \subsubsection{Entity send\_message}
 \subsubsection{send\_message}

 \begin{Mnamedesc}{send_message}

 \begin{Mdesctext}

 In January each year the Registration Authority shall send a message to each
 \nexp{manufacturer} whose cars' average fuel consumption exceeds a certain
 limit, which may vary from year to year.

 \end{Mdesctext}

 \begin{Mexp}
 \begin{verbatim}
 *)
 ENTITY send_message;
   max_consumption : fuel_consumption;
   year            : INTEGER;
   makers          : SET [0:?] OF authorized_manufacturer;
 DERIVE
   excessives : SET [0:?] OF manufacturer := guzzlers(SELF);
 END_ENTITY;
 (*
 \end{verbatim}
 \end{Mexp}

 \begin{Matts}

 \item[max\_consumption:] The legal maximum average fuel consumption.

 \item[year:] The year for which the \nexp{max consumption} value applies.

 \item[makers:] The \nexp{authorized manufacturers} operating during the
 \nexp{year}.

 \item[excessives:] The \nexp{manufacturers} whose cars exceed the consumption 
 limit. 

 \end{Matts}

 \end{Mnamedesc}

 \subsection{Rule definitions}

% \subsubsection{Rule max\_number}
 \subsubsection{max\_number}

 \begin{Mnamedesc}{max_number}

 \begin{Mdesctext}

 No more than five \nexp{authorized manufacturers} are permitted at any one
 time.

 \end{Mdesctext}

 \begin{Mexp}
 \begin{verbatim}
 *)
 RULE max_number FOR (authorized_manufacturer);
 WHERE
   max_of_5 : SIZEOF(authorized_manufacturer) <= 5;
 END_RULE;
 (*
 \end{verbatim}
 \end{Mexp}

 \begin{Mprops}

 \item[max\_of\_5:] The rule is violated if there are more than five
 \nexp{authorized manufacturers} at any time.

 \end{Mprops}

 \end{Mnamedesc}


 \subsection{Function and procedure definitions}

% \subsubsection{Function guzzlers}
 \subsubsection{guzzlers}

 \begin{Mnamedesc}{guzzlers}

 \begin{Mdesctext}

 This function returns the set of \nexp{manufacturers} whose cars exceed an
 average fuel consumption limit.

 \end{Mdesctext}

 \begin{Ipars}

 \item[par:] An instance of a \nexp{send message} entity.

 \item[RESULT:] A set of instances of \nexp{manufacturer} whose cars' average
 fuel consumption is excessive.

 \end{Ipars}

 \begin{Mexp}
 \begin{verbatim}
 *)
 FUNCTION guzzlers(par : send_message) : SET OF manufacturer;
 LOCAL
   result : SET OF manufacturer := [];
   mnfs   : SET OF manufacturer := par.makers;
   limit  : fuel_consumption := par.max_consumption;
   time   : INTEGER := par.year;
 END_LOCAL;
   REPEAT i := 1 TO SIZEOF(mnfs);
     IF (mnfg_average_consumption(mnfs[i],time) > limit) THEN
       result := result + mnfs[i];
     END_IF;
   END_REPEAT;
 RETURN(result);
 END_FUNCTION;
 (*
 \end{verbatim}
 \end{Mexp}
 \end{Mnamedesc}

% \subsubsection{Function too\_old}
 \subsubsection{too\_old}

 \begin{Mnamedesc}{too_old}

 \begin{Mdesctext}

 This function calculates whether the \nexp{car} in a \nexp{history} was
 destroyed more than two years ago.

 \end{Mdesctext}

 \begin{Ipars}

 \item[par:] An instance of a \nexp{history}.

 \item[RESULT:] A Boolean value. TRUE if the \nexp{car} in the input
 \nexp{history} was destroyed two or more years ago; otherwise FALSE.

 \end{Ipars}

 \begin{Mexp}
 \begin{verbatim}
 *)
 FUNCTION too_old(par : history) : BOOLEAN;
   (* The function returns TRUE if the input history is 
      outdated. That is, if it is of an item that was destroyed 
      more than 2 years ago. *) 
   IF ('SUPPORT.DESTROYED_CAR' IN par.item) THEN
     IF (current_date.year-par.item.destroyed_on.year >= 2) THEN
       RETURN(TRUE);
     END_IF;
   END_IF;
   RETURN(FALSE);
 END_FUNCTION;
 (*
 \end{verbatim}
 \end{Mexp}
 \end{Mnamedesc}

% \subsubsection{Function exchange\_ok}
 \subsubsection{exchange\_ok}

 \begin{Mnamedesc}{exchange_ok}

 \begin{Mdesctext}

 This function checks whether or not the \nexp{transfers} in a list are
 ordered.

 \end{Mdesctext}

 \begin{Ipars}

 \item[par] A list of \nexp{transfer} instances.

 \item[RESULT] A Boolean value. TRUE if the recipient in the $N^{th}$ transfer
 is the same as the giver in the $(N+1)^{th}$ transfer.

 \end{Ipars}

 \begin{Mexp}
 \begin{verbatim}
 *)
 FUNCTION exchange_ok(par : LIST OF transfer) : BOOLEAN;
   (* returns TRUE if the "to owner" in the N'th transfer of a 
      car is the "from owner" in the N+1'th transfer *) 
   REPEAT i := 1 TO (SIZEOF(par) - 1);
     IF (par[i].new :<>: par[i+1].prior) THEN
       RETURN (FALSE);
     END_IF;
   END_REPEAT;
   RETURN (TRUE);
 END_FUNCTION;
 (*
 \end{verbatim}
 \end{Mexp}
 \end{Mnamedesc}

% \subsubsection{Function single\_car}
 \subsubsection{single\_car}

 \begin{Mnamedesc}{single_car}
 \begin{Mdesctext}
     This function checks whether or not the \nexp{car} in a transfer
 \nexp{history} is the same \nexp{car} specified in each individual
 \nexp{transfer}.
 \end{Mdesctext}

 \begin{Ipars}

 \item[par:] A \nexp{history} instance.

 \item[RESULT:] A Boolean value. TRUE if the \nexp{history} and all its
 \nexp{transfers} are of the same \nexp{car}, otherwise FALSE.
 \end{Ipars}

 \begin{Mexp}
 \begin{verbatim}
 *)
 FUNCTION single_car(par : history) : BOOLEAN;
   (* returns TRUE if a history is of a single car *) 
   REPEAT i := 1 TO SIZEOF(par.transfers);
     IF (par.item :<>: par.transfers[i].item) THEN
       RETURN (FALSE);
     END_IF;
   END_REPEAT;
   RETURN (TRUE);
 END_FUNCTION;
 (*
 \end{verbatim}
 \end{Mexp}
 \end{Mnamedesc}

 \subsection{Entity classification structure}

 The following indented listing shows the entity classification structure.
 Entities in upper case characters are defined in this schema. Entities in
 lower case characters are defined in other schemas.

 \begin{small} 
 \begin{verbatim}
 HISTORY
 manufacturer (in schema support)
   AUTHORIZED_MANUFACTURER
 SEND_MESSAGE
 \end{verbatim} 
 \end{small}

 \begin{small} 
 \begin{verbatim}
 *)
 END_SCHEMA;  -- end of authority schema
 (*
 \end{verbatim} 
 \end{small}

 \section{Support schema}

 \begin{Mnamedesc}{support}
 \begin{Mdesctext}
     This schema contains supporting definitions for the primary
 \nexp{authority} schema.

     An \ExpressG\ model of the contents of this schema is given in 
 \fref{fig:cargaux1} and in \fref{fig:cargaux2}.


    The schema imports definitions from the \nexp{calendar} schema.
\end{Mdesctext}

\begin{Mexp}
\begin{verbatim}
*)
SCHEMA support;
  REFERENCE FROM calendar (date, months, days_between);
(*
\end{verbatim}
\end{Mexp}
\end{Mnamedesc}

\subsection{Type definitions}

%\subsubsection{Type name}
\subsubsection{name}

\begin{Mnamedesc}{name}
\begin{Mdesctext}

The \Q{name} of something. A human interpretable name which may identify some
object, thing or person, etc. For example, \nexp{Widget Company, Inc.}.

\end{Mdesctext}

\begin{Mexp}
\begin{verbatim}
*)
TYPE name = STRING;
END_TYPE;
(*
\end{verbatim}
\end{Mexp}
\end{Mnamedesc}

%\subsubsection{Type identification\_no}
\subsubsection{identification\_no}

\begin{Mnamedesc}{identification_no}
\begin{Mdesctext}
A character string which may be used as the \Q{identification number} for a
particular instance of some object. This is typically a mixture of
alphanumeric characters and other symbols. For example, \nexp{D20-736597WP23}.
\end{Mdesctext}

\begin{Mexp}
\begin{verbatim}
*)
TYPE identification_no = STRING;
END_TYPE;
(*
\end{verbatim}
\end{Mexp}
\end{Mnamedesc}

 %\input{/rdrc/design/pwilson/Pbooks/Ebook/Bkfigs/cargaux1}
 %\input{/rdrc/design/pwilson/Pbooks/Ebook/Bkfigs/cargaux2}
 %%%%%%%%%%%%%%%%
% cargaux1.tex  BOOK  EXPRESS-G of Auxiliary (1) schema  fig:cargaux1
\begin{figure}[tbp]
\center
\setlength{\unitlength}{1mm}
\begin{picture}(110,110)
%\begin{picture}(110,160)\input{gridmm}
\thicklines

% ident #
\put(10,0){\begin{picture}(90,4)

  \put(0,0){\begin{picture}(15,4)
    \put(7.5,2){\oval(15,4)}
    \put(7.5,2){\makebox(0,0){1,3 (2)}}
    \end{picture}}

  \put(25,0){\dashbox{1}(30,4){identification\_no}}

  \put(65,0){\begin{picture}(24,4)
    \put(0,0){\framebox(24,4){STRING}}
    \put(22,0){\line(0,1){4}}
    \end{picture}}

  % ref/id #
  \put(15,2){\line(1,0){8}}
  \put(24,2){\circle{2}}

  % no/string
  \put(55,2){\line(1,0){8}}
  \put(64,2){\circle{2}}

  \end{picture}} % end ident #

%  car model
\put(0,20){\begin{picture}(110,45)

  \put(50,0){\begin{picture}(30,4)
    \put(15,2){\oval(30,4)}
    \put(15,2){\makebox(0,0){2,2 manufacturer}}
    \end{picture}}

  \put(0,10){\framebox(20,10){car\_model}}

  \put(50,10){\dashbox{1}(30,4){*fuel\_consumption}}

  \put(86,10){\begin{picture}(24,4)
    \put(0,0){\framebox(24,4){REAL}}
    \put(22,0){\line(0,1){4}}
    \end{picture}}

  \put(70,16){\dashbox{1}(10,4){name}}

  \put(86,16){\begin{picture}(24,4)
    \put(0,0){\framebox(24,4){STRING}}
    \put(22,0){\line(0,1){4}}
    \end{picture}}

  % ref into model
  \put(2.5,30){\begin{picture}(15,4)
    \put(7.5,2){\oval(15,4)}
    \put(7.5,2){\makebox(0,0){1,1 (2)}}
    \put(7.5,0){\line(0,-1){8}}
    \put(7.5,-9){\circle{2}}
    \end{picture}}

  % ref into name
  \put(67.5,30){\begin{picture}(15,4)
    \put(7.5,2){\oval(15,4)}
    \put(7.5,2){\makebox(0,0){1,2 (2)}}
    \put(7.5,0){\line(0,-1){8}}
    \put(7.5,-9){\circle{2}}
    \end{picture}}

  % model/mnfr
  \put(0,0){\begin{picture}(50,10)
    \put(10,2){\line(0,1){8}}
    \put(10,2){\line(1,0){40}}
    \put(30,3){\makebox(0,0)[b]{made\_by}}
    \end{picture}}

  % model/fuel/real
  \put(20,10){\begin{picture}(70,10)
    \put(0,2){\line(1,0){28}}
    \put(29,2){\circle{2}}
    \put(15,1){\makebox(0,0)[t]{consumption}}
    \put(60,2){\line(1,0){4}}
    \put(65,2){\circle{2}}
    \end{picture}}

  % model/name/string
  \put(20,10){\begin{picture}(70,10)
    \put(0,8){\line(1,0){48}}
    \put(49,8){\circle{2}}
    \put(15,9){\makebox(0,0)[b]{*called}}
    \put(60,8){\line(1,0){4}}
    \put(65,8){\circle{2}}
    \end{picture}}

  \end{picture}} % end car model

% transfer
\put(5,80){\begin{picture}(100,30)

  \put(70,0){\begin{picture}(30,8)
    \put(0,0){\dashbox{2}(30,8){calendar.date}}
    \put(15,4){\oval(30,4)}
    \end{picture}}

  \put(0,10){\framebox(20,10){transfer}}

  \put(70,13){\begin{picture}(20,4)
    \put(10,2){\oval(20,4)}
    \put(10,2){\makebox(0,0){2,1 owner}}
    \end{picture}}
  
  \put(70,23){\begin{picture}(20,4)
    \put(10,2){\oval(20,4)}
    \put(10,2){\makebox(0,0){2,3 car}}
    \end{picture}}
  
  % transfer/date
  \put(0,0){\begin{picture}(70,10)
    \put(10,4){\line(0,1){6}}
    \put(10,4){\line(1,0){60}}
    \put(40,3){\makebox(0,0)[t]{on}}
    \end{picture}}

  % transfer/owner
  \put(20,10){\begin{picture}(50,10)
    \multiput(0,2)(0,6){2}{\line(1,0){40}}
    \put(40,2){\line(0,1){6}}
    \put(40,5){\line(1,0){10}}
    \put(20,1){\makebox(0,0)[t]{*prior}}
    \put(20,9){\makebox(0,0)[b]{*new}}
    \end{picture}}

  % transfer/car
  \put(0,20){\begin{picture}(70,10)
    \put(10,5){\line(0,-1){5}}
    \put(10,5){\line(1,0){60}}
    \put(40,6){\makebox(0,0)[b]{*item}}
    \end{picture}}

  \end{picture}}  % end transfer

\end{picture}
\setlength{\unitlength}{1pt}
\caption{Complete entity-level model of the Support schema
         (Page 1 of 2).}
\label{fig:cargaux1}
\end{figure}


%\subsubsection{Type fuel\_consumption}
\subsubsection{fuel\_consumption}

\begin{Mnamedesc}{fuel_consumption}
\begin{Mdesctext}
    A measure of the fuel consumption of some powered device.
\end{Mdesctext}

\begin{Mexp}
\begin{verbatim}
*)
TYPE fuel_consumption = REAL;
WHERE
  range : {4.0 <= SELF <= 25.0};
END_TYPE;
(*
\end{verbatim}
\end{Mexp}

\begin{Mprops}

\item[range:] The value is limited to lie in the range 4 to 25 inclusive.
\end{Mprops}
\end{Mnamedesc}


% cargaux2.tex  BOOK  EXPRESS-G of Auxiliary (2) schema  fig:cargaux2
\begin{figure}[tbp]
\center
\setlength{\unitlength}{1mm}
\begin{picture}(110,160)
%\begin{picture}(110,160)\input{gridmm}
\thicklines


% Owner tree
\put(0,0){\begin{picture}(110,100)

  % Integer/garage
  \put(38,0){\begin{picture}(24,15)
    \put(0,0){\framebox(24,4){INTEGER}}
    \put(22,0){\line(0,1){4}}
    \put(12,6){\line(0,1){9}}
    \put(12,5){\circle{2}}
    \put(13,11){(DER)}
    \put(13,8){no\_of\_mnfs}
    \end{picture}}

  % ref into mnf
  \put(7.5,0){\begin{picture}(15,15)
    \put(7.5,2){\oval(15,4)}
    \put(7.5,2){\makebox(0,0){2,2 (1)}}
    \put(7.5,4){\line(0,1){11}}
    \end{picture}}


  \put(0,15){\framebox(30,10){manufacturer}}

  \put(40,15){\framebox(20,10){garage}}

  \put(70,15){\framebox(20,10){person}}

  \put(25,40){\framebox(40,10){(ABS) named\_owner}}

  \put(30,65){\framebox(30,10){(ABS) owner}}

  \put(80,55){\framebox(20,10){group}}

  % owner/name ref
  \put(65,40){\begin{picture}(40,10)
    \put(15,3){\begin{picture}(20,4)
      \put(10,2){\oval(20,4)}
      \put(10,2){\makebox(0,0){1,2 name}}
      \end{picture}}
    \put(0,5){\line(1,0){15}}
    \put(7.5,6){\makebox(0,0)[b]{*called}}
    \end{picture}}

  % named subtypes connections
  \put(20,25){\begin{picture}(70,15)
    \multiput(0,0)(30,0){3}{\begin{picture}(1,10)
      \put(0,1){\circle{2}}
      \multiput(-0.25,2)(0.25,0){3}{\line(0,1){8}}
      \end{picture}} 
    \multiput(0,9.75)(0,0.25){3}{\line(1,0){60}}
    \multiput(24.75,10)(0.25,0){3}{\line(0,1){5}}
    \put(26,11){1}
    \end{picture}}

  % owner/named/group subtype connections
  \put(45,50){\begin{picture}(40,15)
    \put(0,1){\circle{2}}
    \multiput(-0.25,2)(0.25,0){3}{\line(0,1){13}}
    \multiput(0,9.75)(0,0.25){3}{\line(1,0){33}}
    \put(34,10){\circle{2}}
    \put(1,11){1}
    \end{picture}}

  % group/person connection
  \put(90,15){\begin{picture}(15,50)
    \put(1,5){\circle{2}}
    \put(15,5){\line(-1,0){13}}
    \put(15,5){\line(0,1){40}}
    \put(15,45){\line(-1,0){5}}
    \put(14,22){\makebox(0,0)[br]{members}}
    \put(14,20){\makebox(0,0)[tr]{S[1:?]}}

    \end{picture}}  

%    \multiput(24.75,2)(0.25,0){3}{\line(0,1){8}}

  \end{picture}} % end owner tree

%  car tree
  \put(0,90){\begin{picture}(70,110)

    \put(0,10){\framebox(30,28){car}}

    % destroyed car and sub
    \put(0,60){\begin{picture}(30,10)
      \put(0,0){\framebox(30,10){destroyed\_car}}
      \multiput(19.75,-2)(0.25,0){3}{\line(0,-1){20}}
      \put(20,-1){\circle{2}}
      \end{picture}}

    \put(70,61){\begin{picture}(30,8)
      \put(0,0){\dashbox{2}(30,8){calendar.date}}
      \put(15,4){\oval(30,4)}
      \end{picture}}

    \put(80,0){\begin{picture}(30,4)
      \put(15,2){\oval(30,4)}
      \put(15,2){\makebox(0,0){1,1 car\_model}}
      \end{picture}}

    \put(70,16){\begin{picture}(40,4)
      \put(20,2){\oval(40,4)}
      \put(20,2){\makebox(0,0){1,3 identification\_no}}
      \end{picture}}

    \put(86,28){\begin{picture}(24,4)
      \put(0,0){\framebox(24,4){INTEGER}}
      \put(22,0){\line(0,1){4}}
      \end{picture}}

    % ref into car
    \put(0,38){\begin{picture}(15,15)
      \put(7.5,12){\oval(15,4)}
      \put(7.5,12){\makebox(0,0){2,3 (1)}}
      \put(7.5,10){\line(0,-1){10}}
      \end{picture}}

    % car/model
    \put(0,0){\begin{picture}(80,10)
      \put(20,2){\line(0,1){8}}
      \put(20,2){\line(1,0){60}}
      \put(50,3){\makebox(0,0)[b]{model\_type}}
      \end{picture}}

    % car/id
    \put(30,10){\begin{picture}(60,20)
      \multiput(0,2)(0,12){2}{\line(1,0){60}}
      \multiput(60,2)(0,8){2}{\line(0,1){4}}
      \put(20,3){\makebox(0,0)[b]{*registration\_no}}
      \put(20,13){\makebox(0,0)[t]{*mnfg\_no}}
      \end{picture}}

    % car/integer
    \put(30,10){\begin{picture}(60,28)
      \put(0,20){\line(1,0){54}}
      \put(55,20){\circle{2}}
      \put(20,21){\makebox(0,0)[b]{*production\_year}} 
      \end{picture}}

    % destroyed/date
    \put(30,60){\begin{picture}(40,10)
      \put(0,5){\line(1,0){40}}
      \put(20,6){\makebox(0,0)[b]{*destroyed\_on}}
      \end{picture}}

    % car/date
    \put(30,10){\begin{picture}(50,60)
      \put(0,26){\line(1,0){50}}
      \put(50,26){\line(0,1){25}}
      \put(20,27){\makebox(0,0)[b]{*production\_date}}
      \end{picture}}

    \end{picture}} % end car tree

  % car/manufacturer
  \put(0,25){\begin{picture}(30,75)
     \put(10,1){\circle{2}}
     \put(10,2){\line(0,1){73}}
     \put(11,60){(DER)}
     \put(11,57){*made\_by}
     \end{picture}}

\end{picture}
\setlength{\unitlength}{1pt}
\caption{Complete entity-level model of the Support schema
         (Page 2 of 2).}
\label{fig:cargaux2}
\end{figure}


\subsection{Entity definitions}

%\subsubsection{Entity transfer}
\subsubsection{transfer}

\begin{Mnamedesc}{transfer}
\begin{Mdesctext}
    A record of a transfer of a \nexp{car} from one owner to a new owner.
\end{Mdesctext}

\begin{Mexp}
\begin{verbatim}
*)
ENTITY transfer;
  item  : car;
  prior : owner;
  new   : owner;
  on    : date;
WHERE
  wr1 : NOT ('SUPPORT.MANUFACTURER' IN TYPEOF(new));
  wr2 : (NOT ('SUPPORT.MANUFACTURER' IN TYPEOF(prior))) XOR
     (('SUPPORT.MANUFACTURER' IN TYPEOF(prior)) AND
     ('SUPPORT.GARAGE' IN TYPEOF (new)));
  wr3 : (NOT ('SUPPORT.GARAGE' IN TYPEOF(prior))) XOR
     (('SUPPORT.GARAGE' IN TYPEOF(prior)) AND
     (('SUPPORT.PERSON' IN TYPEOF(new)) XOR
     ('SUPPORT.GROUP' IN TYPEOF(new))));
  wr4 : (NOT ('SUPPORT.DESTROYED_CAR' IN TYPEOF(item)) XOR
     (('SUPPORT.DESTROYED_CAR' IN TYPEOF(item)) AND
     (days_between(on, item\destroyed_car.destroyed_on) > 0)));
END_ENTITY;
(*
\end{verbatim}
\end{Mexp}

\begin{Matts}

\item[item:] The \nexp{car} being transferred.

\item[prior:] The prior owner of the \nexp{item}.

\item[new:] The new owner of the \nexp{item}.

\item[on:] The \nexp{date} of the \nexp{transfer}.
\end{Matts}

\begin{Mprops}

\item[wr1:] A \nexp{car} cannot be transferred to a \nexp{manufacturer}.

\item[wr2:] A \nexp{manufacturer} can only transfer a \nexp{car} to a
\nexp{garage}.

\item[wr3:] A \nexp{garage} can only transfer a \nexp{car} to either a
\nexp{person} of a \nexp{group} of people.

\item[wr4:] A \nexp{car} which has been destroyed cannot be
transferred.
\end{Mprops}
\end{Mnamedesc}

%\subsubsection{Entity car}
\subsubsection{car}

\begin{Mnamedesc}{car}
\begin{Mdesctext}
   A \nexp{car}.
\end{Mdesctext}

\begin{Mexp}
\begin{verbatim}
*)
ENTITY car;
  model_type      : car_model;
  mnfg_no         : identification_no;
  registration_no : identification_no;
  production_date : date;
  production_year : INTEGER;
DERIVE
  made_by : manufacturer := model_type.made_by;
UNIQUE
  joint  : made_by, mnfg_no;
  single : registration_no;
WHERE
  jan_prod : (production_year = production_date.year) XOR
             ((production_date.month = months.January) AND
              (production_year = production_date.year - 1));
END_ENTITY;
(*
\end{verbatim}
\end{Mexp}

\begin{Matts}

\item[model\_type:] The \nexp{car model}.

\item[mnfg\_no:] An identification number of the \nexp{car} assigned by the
car's manufacturer.

\item[registration\_no:] An identification number for the \nexp{car} assigned
by the Registration Authority.

\item[production\_date:] The date on which the car was produced.

\item[production\_year:] The registered year of production of the \nexp{car}.

\item[made\_by:] The \nexp{manufacturer} of the \nexp{car}.
\end{Matts}

\begin{Mprops}

\item[joint:] The \nexp{mnfg no} given to a \nexp{car} is unique for the given
car manufacturer.

\item[single:] Each car is given a unique \nexp{registration no} by the
Registration Authority.

\item[jan\_prod:] The registered \nexp{production year} is the same as the year
in which the car was produced, except that cars produced in January may be
registered as having been produced in the previous year.
\end{Mprops}
\end{Mnamedesc}

%\subsubsection{Entity destroyed\_car}
\subsubsection{destroyed\_car}

\begin{Mnamedesc}{destroyed_car}
\begin{Mdesctext}
    A \nexp{car} may be destroyed, in which case its date of destruction is
recorded.
\end{Mdesctext}

\begin{Mexp}
\begin{verbatim}
*)
ENTITY destroyed_car
  SUBTYPE OF (car);
  destroyed_on : date;
WHERE
  dates_ok : days_between(production_date, destroyed_on) >= 0;
END_ENTITY;
(*
\end{verbatim}
\end{Mexp}

\begin{Matts}

\item[destroyed\_on:] The date on which the \nexp{car} was destroyed.
\end{Matts}

\begin{Mprops}

\item[dates\_ok:] A \nexp{car} cannot be destroyed before it has been made.
\end{Mprops}
\end{Mnamedesc}

%\subsubsection{Entity car\_model}
\subsubsection{car\_model}

\begin{Mnamedesc}{car_model}
\begin{Mdesctext}
    A particular type of \nexp{car}.
\end{Mdesctext}

\begin{Mexp}
\begin{verbatim}
*)
ENTITY car_model;
  called      : name;
  made_by     : manufacturer;
  consumption : fuel_consumption;
UNIQUE
  un1 : called;
END_ENTITY;
(*
\end{verbatim}
\end{Mexp}

\begin{Matts}

\item[called:] The name of the model.

\item[made\_by:] The \nexp{manufacturer} of the model.

\item[consumption:] The average fuel consumption of all cars of this model
type.

\end{Matts}

\begin{Mprops}

\item[un1:] Each \nexp{car model} has a distinct name.
\end{Mprops}
\end{Mnamedesc}

%\subsubsection{Entity owner}
\subsubsection{owner}

\begin{Mnamedesc}{owner}
\begin{Mdesctext}
    An owner of a \nexp{car}. Owners are categorized into \nexp{named owner}
and \nexp{group}.
\end{Mdesctext}

\begin{Mexp}
\begin{verbatim}
*)
ENTITY owner
  ABSTRACT SUPERTYPE OF (ONEOF(named_owner,
                               group));
END_ENTITY;
(*
\end{verbatim}
\end{Mexp}
\end{Mnamedesc}

%\subsubsection{Entity named\_owner}
\subsubsection{named\_owner}

\begin{Mnamedesc}{named_owner}
\begin{Mdesctext}
    An \nexp{owner} who has a name. These are categorized into
\nexp{manufacturer}, \nexp{garage} and \nexp{person}.
\end{Mdesctext}

\begin{Mexp}
\begin{verbatim}
*)
ENTITY named_owner
  ABSTRACT SUPERTYPE OF (ONEOF(manufacturer,
                               garage,
                               person))
  SUBTYPE OF (owner);
  called : name;
UNIQUE
  un1 : called;
END_ENTITY;
(*
\end{verbatim}
\end{Mexp}

\begin{Matts}

\item[called:] The name of the \nexp{owner}.
\end{Matts}

\begin{Mprops}

\item[un1:] Owner's names are unique.
\end{Mprops}
\end{Mnamedesc}

%\subsubsection{Entity manufacturer}
\subsubsection{manufacturer}

\begin{Mnamedesc}{manufacturer}
\begin{Mdesctext}
    A type of named car owner. Manufacturers may also manufacture cars.
\end{Mdesctext}

\begin{Mexp}
\begin{verbatim}
*)
ENTITY manufacturer
  SUBTYPE OF (named_owner);
END_ENTITY;
(*
\end{verbatim}
\end{Mexp}
\end{Mnamedesc}

%\subsubsection{Entity garage}
\subsubsection{garage}

\begin{Mnamedesc}{garage}
\begin{Mdesctext}
    A type of named car owner.
\end{Mdesctext}

\begin{Mexp}
\begin{verbatim}
*)
ENTITY garage
  SUBTYPE OF (named_owner);
DERIVE
  no_of_mnfs : INTEGER := dealer_for_mnfs(SELF);
WHERE
  wr1 : {1 <= no_of_mnfs <= 3};
END_ENTITY;
(*
\end{verbatim}
\end{Mexp}

\begin{Matts}

\item[no\_of\_mnfs:] The number of different manufacturers of the cars owned by
the \nexp{garage}.
\end{Matts}

\begin{Mprops}

\item[wr1:] At any particular time, a \nexp{garage} shall not own cars made by
more than three manufacturers.
\end{Mprops}
\end{Mnamedesc}

%\subsubsection{Entity person}
\subsubsection{person}


\begin{Mnamedesc}{person}
\begin{Mdesctext}
    A type of named car owner.
\end{Mdesctext}

\begin{Mexp}
\begin{verbatim}
*)
ENTITY person
  SUBTYPE OF (named_owner);
END_ENTITY;
(*
\end{verbatim}
\end{Mexp}
\end{Mnamedesc}

%\subsubsection{Entity group}
\subsubsection{group}

\begin{Mnamedesc}{group}
\begin{Mdesctext}
    A type of car owner consisting of a group of people.
\end{Mdesctext}

\begin{Mexp}
\begin{verbatim}
*)
ENTITY group
  SUBTYPE OF (owner);
  members : SET [1:?] OF person;
END_ENTITY;
(*
\end{verbatim}
\end{Mexp}

\begin{Matts}

\item[members:] The people who form the \nexp{group}.
\end{Matts}
\end{Mnamedesc}

\subsection{Function and procedure definitions}

%\subsubsection{Function dealer\_for\_mnfs}
\subsubsection{dealer\_for\_mnfs}

\begin{Mnamedesc}{dealer_for_mnfs}
\begin{Mdesctext}
    This function calculates the total number of distinct manufacturers of cars
owned by a \nexp{garage}.
\end{Mdesctext}

\begin{Ipars}

\item[dealer:] An instance of a \nexp{garage}.

\item[RESULT:] The number of distinct manufacturers of the cars owned by the
\nexp{garage}.
\end{Ipars}

\begin{Mexp}
\begin{verbatim}
*)
FUNCTION dealer_for_mnfs(dealer : garage) : INTEGER;
  LOCAL
    cars : SET OF car := [];
    transfers : SET OF transfer := [];
    makers : SET OF manufacturer := [];
  END_LOCAL;
  transfers := USEDIN(dealer, 'TRANSFER.NEW');
  REPEAT i := 1 TO SIZEOF(transfers);
    cars := cars + transfers[i].item;
  END_REPEAT;
  transfers := USEDIN(dealer, 'TRANSFER.PRIOR');
  REPEAT i := 1 TO SIZEOF(transfers);
    cars := cars - transfers[i].item;
  END_REPEAT;
  REPEAT i := 1 TO SIZEOF(cars);
    makers := makers + cars[i].model_type.made_by;
  END_REPEAT;
  RETURN (SIZEOF(makers));
END_FUNCTION;
(*
\end{verbatim}
\end{Mexp}
\end{Mnamedesc}

%\subsubsection{Function mnfg\_average\_consumption}
\subsubsection{mnfg\_average\_consumption}

\begin{Mnamedesc}{mnfg_average_consumption}
\begin{Mdesctext}
    This function calculates the average fuel consumption in a given year
of all the cars made by a particular manufacturer.
\end{Mdesctext}

\begin{Ipars}

\item[mnfg:] A \nexp{manufacturer}.

\item[when:] An INTEGER representing a particular year.

\item[RESULT:] A REAL giving the average fuel consumption of the manufacturer's
cars during a particular year.
\end{Ipars}

\begin{Mexp}
\begin{verbatim}
*)
FUNCTION mnfg_average_consumption(mnfg : manufacturer;
                                  when : INTEGER) : REAL;
  (* returns the average fuel consumption of the given 
     manufacturer's cars produced in the given year *) 
  LOCAL
    models : SET OF car_model := [];
    cars   : SET OF car := [];
    num    : INTEGER := 0;
    tot    : INTEGER := 0;
    fuel   : REAL := 0;
    result : REAL := 0.0;
  END_LOCAL;
     -- set of mnfg's models
  models := USEDIN(mnfg, 'MODEL.MADE_BY'); 
  REPEAT i := 1 TO SIZEOF(models);
     -- cars of particular model year
    cars := QUERY(temp <* USEDIN(models[i], 'CAR.MODEL_TYPE')
            | temp.production_year = when);                                 
    num := SIZEOF(cars);
    fuel := fuel + num*models[i].consumption;
    tot := tot + num;
  END_REPEAT;
  IF tot > 0.0 THEN
    result := fuel/tot;
  END_IF;
  RETURN (result);
END_FUNCTION;
(*
\end{verbatim}
\end{Mexp}
\end{Mnamedesc}

\subsection{Entity classification structure}

The following indented listing shows the entity classification structure.
Entities in upper case characters are defined in this schema. Entities in
lower case characters are defined in other schemas.

\begin{small} 
\begin{verbatim}
CAR
    DESTROYED_CAR
CAR_MODEL
OWNER
    GROUP
    NAMED_OWNER
        GARAGE
        MANUFACTURER
        PERSON
TRANSFER
\end{verbatim} 
\end{small}

\begin{small} 
\begin{verbatim}
*)
END_SCHEMA;  -- end of support schema
(*
\end{verbatim} 
\end{small}

\section{Calendar schema}

\begin{Mnamedesc}{calendar}
\begin{Mdesctext}
This schema contains definitions related to dates and other calendrical items.

%\input{/rdrc/design/pwilson/Pbooks/Ebook/Bkfigs/cargcal}
%%%%%%%%%%%%%%%%
% cargcal.tex  BOOK EXPRESS-G calender schema  fig:cargcal
\begin{figure}[tbp]
\center
\setlength{\unitlength}{1mm}
\begin{picture}(85,22)
%\begin{picture}(110,160)\input{gridmm}
\thicklines

\put(0,10){\begin{picture}(20,10)
  \put(0,0){\framebox(20,10){*date}}
  \end{picture}}

\put(60,13){\begin{picture}(24,4)
  \put(0,0){\framebox(24,4){INTEGER}}
  \put(22,0){\line(0,1){4}}
  \end{picture}}

\put(60,0){\begin{picture}(24,4)
  \put(0,0){\dashbox{1}(24,4){months}}
  \multiput(22,0.5)(0,2){2}{\line(0,1){1}}
  \end{picture}}

% date/months
\put(10,0){\begin{picture}(50,10)
  \put(0,2){\line(0,1){8}}
  \put(0,2){\line(1,0){48}}
  \put(49,2){\circle{2}}
  \put(25,3){\makebox(0,0)[b]{month}}
  \end{picture}}

% date/integer
\put(20,10){\begin{picture}(40,12)
  \multiput(0,2.5)(0,5){2}{\line(1,0){30}}
  \put(30,2.5){\line(0,1){5}}
  \put(30,5){\line(1,0){8}}
  \put(39,5){\circle{2}}
  \put(15,1.5){\makebox(0,0)[t]{*year}}
  \put(15,8.5){\makebox(0,0)[b]{*day}}
  \end{picture}}

\end{picture}
\setlength{\unitlength}{1pt}
\caption{Complete entity-level model of Calendar schema (Page 1 of 1).}
\label{fig:cargcal}
\end{figure}


    Figure~\ref{fig:cargcal} is an \ExpressG\ model showing the contents
of this schema.

\end{Mdesctext}

\begin{Mexp}
\begin{verbatim}
*)
SCHEMA calendar;
(*
\end{verbatim}
\end{Mexp}
\end{Mnamedesc}

\subsection{Type definitions}

%\subsubsection{Type months}
\subsubsection{months}

\begin{Mnamedesc}{months}
\begin{Mdesctext}
    An enumeration of the months of the year. \nexp{January} is the first month
in a year and \nexp{December} is the last month in a year.

\end{Mdesctext}

\begin{Mexp}
\begin{verbatim}
*)
TYPE months = ENUMERATION OF
    (January, February, March,
     April,   May,      June,
     July,    August,   September,
     October, November, December);
END_TYPE;
(*
\end{verbatim}
\end{Mexp}
\end{Mnamedesc}

\subsection{Entity definitions}

%\subsubsection{Entity date}
\subsubsection{date}

\begin{Mnamedesc}{date}
\begin{Mdesctext}
    A \nexp{date} AD in the Gregorian calendar.
\end{Mdesctext}

\begin{Mexp}
\begin{verbatim}
*)
ENTITY date;
  day   : INTEGER;
  month : months;
  year  : INTEGER;
WHERE
  days_ok : {1 <= day <= 31};
  year_ok : year > 0;
  date_ok : valid_date(SELF);
END_ENTITY;
(*
\end{verbatim}
\end{Mexp}

\begin{Matts}

\item[day:] The day of the \nexp{month}.

\item[month:] The month of the \nexp{year}

\item[year:] The year.
\end{Matts}

\begin{Mprops}

\item[days\_ok:] The \nexp{day} shall be numbered between 1 and 31 inclusive.

\item[year\_ok:] The year shall be greater than zero.

\item[date\_ok:] The combination of \nexp{day}, \nexp{month} and \nexp{year}
shall form a valid date, taking into account the differing numbers of days in
particular months, and also the effect of leap years.

\end{Mprops}

\end{Mnamedesc}

\subsection{Function and procedure definitions}

%\subsubsection{Function valid\_date}
\subsubsection{valid\_date}

\begin{Mnamedesc}{valid_date}
\begin{Mdesctext}
This function checks a \nexp{date} for valid day, month, year combinations.
\end{Mdesctext}

\begin{Ipars}

\item[par:] A \nexp{date}.

\item[RESULT:] A Boolean. TRUE if the \nexp{date} has a valid day, month, year
combination, FALSE otherwise.
\end{Ipars}

\begin{Mexp}
\begin{verbatim}
*)
FUNCTION valid_date (par : date) : BOOLEAN;
  (* returns FALSE if its input is not a valid date *)
  CASE par.month OF
    April     : RETURN (par.day <= 30);
    June      : RETURN (par.day <= 30);
    September : RETURN (par.day <= 30);
    November  : RETURN (par.day <= 30);
    February  : IF (leap_year(par.year)) THEN
                  RETURN (par.day <= 29);
                ELSE
                  RETURN (par.day <= 28);
                END_IF;
    OTHERWISE : RETURN (TRUE);
  END_CASE;
END_FUNCTION;
(*
\end{verbatim}
\end{Mexp}
\end{Mnamedesc}

%\subsubsection{Function leap\_year}
\subsubsection{leap\_year}

\begin{Mnamedesc}{leap_year}
\begin{Mdesctext}
    This function checks whether a given integer could represent a leap year.
\end{Mdesctext}

\begin{Ipars}

\item[year:] An INTEGER.

\item[RESULT:] A Boolean. TRUE if \nexp{year} is a leap year, otherwise FALSE.
\end{Ipars}

\begin{Mexp}
\begin{verbatim}
*)
FUNCTION leap_year(year : INTEGER) : BOOLEAN;
  (* returns TRUE if its input is a leap year *)
  IF ((((year MOD 4) = 0) AND ((year MOD 100) <> 0)) OR
      ((year MOD 400) = 0)) THEN
    RETURN (TRUE);
  ELSE
    RETURN (FALSE);
  END_IF;
END_FUNCTION;
(*
\end{verbatim}
\end{Mexp}
\end{Mnamedesc}

%\subsubsection{Function current\_date}
\subsubsection{current\_date}

\begin{Mnamedesc}{current_date}
\begin{Mdesctext}
    This function returns the current date.
\end{Mdesctext}

\begin{Ipars}

\item[RESULT:] The current \nexp{date}.
\end{Ipars}

\begin{Mexp}
\begin{verbatim}
*)
FUNCTION current_date : date;
  (* This function returns the date when it is called. 
     Typically, it will be implemented via a system provided 
     procedure within the information base *) 
END_FUNCTION;
(*
\end{verbatim}
\end{Mexp}
\end{Mnamedesc}

%\subsubsection{Function days\_between}
\subsubsection{days\_between}

\begin{Mnamedesc}{days_between}

\begin{Mdesctext}

This function returns the number of days between any two \nexp{date}s.

\end{Mdesctext}

\begin{Ipars}

\item[d1:] A \nexp{date}.

\item[d2:] A \nexp{date}.

\item[RESULT:] An Integer. The number of days between the two input
\nexp{dates}. If \nexp{d1} is earlier than \nexp{d2} a positive integer is
returned; if \nexp{d1} is later than \nexp{d2} a negative integer is returned;
otherwise zero is returned.

\end{Ipars}

\begin{Mexp}
\begin{verbatim}
*)
FUNCTION days_between(d1, d2 : date) : INTEGER;
  (* returns the number of days between two input dates. If d1 
     is earlier than d2, a positive number is returned. *) 
END_FUNCTION;
(*
\end{verbatim}
\end{Mexp}
\end{Mnamedesc}

\subsection{Entity classification structure}

The following indented listing shows the entity classification structure.
Entities in upper case characters are defined in this schema. Entities in
lower case characters are defined in other schemas.

\begin{small} 
\begin{verbatim}
DATE
\end{verbatim} 
\end{small}

\begin{small} 
\begin{verbatim}
*)
END_SCHEMA; -- end of calendar schema
(*
\end{verbatim} 
\end{small}

\end{document}